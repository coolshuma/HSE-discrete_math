\documentclass[a4paper,12pt]{article}

%% Работа с русским языком
\usepackage{cmap}					% поиск в PDF
\usepackage{mathtext} 				% русские буквы в формулах
\usepackage[T2A]{fontenc}			% кодировка
\usepackage[utf8]{inputenc}			% кодировка исходного текста
\usepackage[english,russian]{babel}	% локализация и переносы
\usepackage{amsmath, amsfonts, amsthm, mathtools, amssymb, icomma, units}
\usepackage{algorithmicx, algorithm}
\usepackage{algpseudocode}

%% Отступы между абзацами и в начале абзаца 
\setlength{\parindent}{0pt}
\setlength{\parskip}{\medskipamount}

%% Изменяем размер полей
\usepackage[top=0.5in, bottom=0.75in, left=0.825in, right=0.825in]{geometry}

%% Графика
\usepackage[pdftex]{graphicx}
\graphicspath{{images/}}

%% Различные пакеты для работы с математикой
\usepackage{mathtools}				% Тот же amsmath, только с некоторыми поправками

%\usepackage{amssymb}				% Математические символы
\usepackage{amsthm}					% Пакет для написания теорем
\usepackage{amstext}
\usepackage{array}
\usepackage{amsfonts}
\usepackage{icomma}					% "Умная" запятая: $0,2$ --- число, $0, 2$ --- перечисление
\usepackage{bbm}				    % Для красивого (!) \mathbb с  буквами и цифрами
\usepackage{enumitem}               % Для выравнивания itemise (\begin{itemize}[align=left])

% Номера формул
\mathtoolsset{showonlyrefs=true} % Показывать номера только у тех формул, на которые есть \eqref{} в тексте.

% Ссылки
\usepackage[colorlinks=true, urlcolor=blue]{hyperref}

% Шрифты
\usepackage{euscript}	 % Шрифт Евклид
\usepackage{mathrsfs}	 % Красивый матшрифт

% Свои команды\textbf{}
\DeclareMathOperator{\sgn}{\mathop{sgn}}

% Перенос знаков в формулах (по Львовскому)
\newcommand*{\hm}[1]{#1\nobreak\discretionary{}
	{\hbox{$\mathsurround=0pt #1$}}{}}

% Графики
\usepackage{tikz}
\usepackage{pgfplots}
%\pgfplotsset{compat=1.12}

% Изменим формат \section и \subsection:
%\usepackage{titlesec}
%\titleformat{\section}
%{\vspace{1cm}\centering\LARGE\bfseries}	% Стиль заголовка
%{}										% префикс
%{0pt}									% Расстояние между префиксом и заголовком
%{} 										% Как отображается префикс
%\titleformat{\subsection}				% Аналогично для \subsection
%{\Large\bfseries}
%{}
%{0pt}
%{}

% Информация об авторах
\title{Лекции по предмету \\
	\textbf{Линейная алгебра и геометрия}}

\newtheorem*{Def}{Определение}
\newtheorem*{Lemma}{Лемма}
\newtheorem*{Suggestion}{Предложение}
\newtheorem*{Examples}{Пример}
%\newtheorem*{Comment}{Замечание}
\newtheorem*{Consequence}{Следствие}
\newtheorem*{Theorem}{Теорема}
\newtheorem*{Statement}{Утверждение}
\newtheorem*{Task}{Упражнение}
\newtheorem*{Designation}{Обозначение}
\newtheorem*{Generalization}{Обобщение}
\newtheorem*{Thedream}{Предел мечтаний}
\newtheorem*{Properties}{Свойства}


\renewcommand{\Re}{\mathrm{Re\:}}
\renewcommand{\Im}{\mathrm{Im\:}}
\newcommand{\Arg}{\mathrm{Arg\:}}
\renewcommand{\arg}{\mathrm{arg\:}}
\newcommand{\Mat}{\mathrm{Mat}}
\newcommand{\id}{\mathrm{id}}
\newcommand{\isom}{\xrightarrow{\sim}} 
\newcommand{\leftisom}{\xleftarrow{\sim}}
\newcommand{\Hom}{\mathrm{Hom}}
\newcommand{\Ker}{\mathrm{Ker}\:}
\newcommand{\rk}{\mathrm{rk}\:}
\newcommand{\diag}{\mathrm{diag}}
\newcommand{\ort}{\mathrm{ort}}
\newcommand{\pr}{\mathrm{pr}}
\newcommand{\vol}{\mathrm{vol\:}}
\def\limref#1#2{{#1}\negmedspace\mid_{#2}}
\newcommand{\eps}{\varepsilon}

\renewcommand{\epsilon}{\varepsilon}
\renewcommand{\phi}{\varphi}
\newcommand{\e}{\mathbb{e}}
\renewcommand{\l}{\lambda}
\renewcommand{\C}{\mathbb{C}}
\newcommand{\R}{\mathbb{R}}
\newcommand{\E}{\mathbb{E}}

\newcommand{\vvector}[1]{\begin{pmatrix}{#1}_1 \\\vdots\\{#1}_n\end{pmatrix}}
\renewcommand{\vector}[1]{({#1}_1, \ldots, {#1}_n)}

%Теоремы
%11.01.2016
\newtheorem*{standartbase}{Теорема о стандартном базисе}
\newtheorem*{fulllemma}{Лемма}
\newtheorem*{sl1}{Следствие 1}
\newtheorem*{sl2}{Следствие 2}
\newtheorem*{monotonousbase}{Теорема о монотонном базисе}
\newtheorem*{scheme}{Утверждение 1}
\newtheorem*{n2}{Утверждение 2}
\newtheorem*{usp-rais}{Теорема Успенского-Райса}
\newtheorem*{rec}{Свойство рекурсии}
\newtheorem*{point}{Теорема о неподвижной точке}
\newtheorem*{zhegalkin}{Теорема Жегалкина}
\newtheorem*{poste}{Теорема Поста}
\newtheorem*{algo1}{Первое свойство алгоритмов}

%18.01.2016
\newtheorem*{theorem}{Теорема}

\renewcommand{\qedsymbol}{\textbf{Q.E.D.}}
\newcommand{\definition}{\underline{Определение:} }
\newcommand{\definitions}{\underline{Определения:} }
\newcommand{\definitionone}{\underline{Определение 1:} }
\newcommand{\definitiontwo}{\underline{Определение 2:} }
\newcommand{\statement}{\underline{Утверждение:} }
\newcommand{\note}{\underline{Замечание:} }
\newcommand{\sign}{\underline{Обозначения:} }
\newcommand{\statements}{\underline{Утверждения:} }

\newcommand{\Z}{\mathbb{Z}}
\newcommand{\N}{\mathbb{N}}
\newcommand{\Q}{\mathbb{Q}}
\begin{document}
	\section{Домашнее задание 15\\ Шумилкин Андрей, группа 163} 
	\subsection{Задача 1}
	Как мы знаем, если некоторое множество $U$ бесконечно, а множество $V$ конечно или счетно, то $U \cup V$ равномощно $U$. \\
	Обозначим $C = A\setminus B$. Тогда мы можем записать $A = C \cup B$. Тогда по свойству, упомянотому выше мощность $C$ равна мощности $A$ и $C$ не может быть не бесконечным, поскольку объединение счетного множества со счетным будет не более, чем счетным. \\
	Значит все утверждение верно. 
	
	\subsection{Задача 2}
	Заметим, что $A \bigtriangleup B = (A \setminus B) \cup (B \setminus A)$. Как мы доказали в прошлой задаче $A \setminus B$ равномощно $A$. А $(B \setminus A)$ будет не более чем счетно, так как само $B$ счетно. \\
	И тогда из свойства, упомянутого в первой задаче и из транзитивности равномощности заметим, что раз $A \setminus B$ равномощно $A$ и  объединение бесконечного множества $U$ и счетного множества $V$ будет равномощно $U$, что $A \bigtriangleup B$ равномощно $A$. \\
	Значит все утверждение верно.
	
	\subsection{Задача 3}
	Как мы знаем, если некоторое множество $U$ бесконечно, а множество $V$ конечно или счетно, то $U \cup V$ равномощно $U$. \\
	Обозначим $C = A\setminus B$. Тогда мы можем записать $A = C \cup B$. Тогда по свойству, упомянотому выше мощность $C$ равна мощности $A$ и $C$ не может быть не бесконечным, поскольку объединение конечного множества со счетным будет не более, чем счетным. \\
	Значит все утверждение верно. 
	
	\subsection{Задача 4}
	Как нам известно множество рациональных чисел счетно и в каждом интервале найдется хотя бы одно рациональное число по аксиоме полноты. \\
	Тогда мы можем сопоставить кажому интервалу минимальную рациональную точку, находящуюся в нем. Такое сопоставление будет взаимно-однозначным, поскольку интервалы не пересекаются. \\
	Раз мы смогли найти такое сопоставление, то мощность множества данных интервалов не больше, чем мощность множества рациональных чисел, а оно счетно. Значит множество интервалов будет не более чем счетно.
	
	\subsection{Задача 5}
	Как мы знаем, всякое бесконечное множество содержит счетное подмножество. \\
	Это подмножество равномощно $\mathbb{N}$. Мы можем заметить, что и само $\mathbb{N}$ содержит бесконечное счетное число счетных подмножеств. Такими подмножествами будут, например, являться степени простых чисел, поскольку, как известно, простых чисел бесконечное число, а их степень так же принадлежит $\mathbb{N}$. И при этом они не будут пересекаться по основной теореме арифметики.\\
	Значит из первичной биекции некоторого подмножества нашего множества в $\mathbb{N}$ мы можем выделить сопоставления со множествами различных простых чисел в степени, которые являются счетными и которых бесконечное число, что и будет означать, что всякое бесконечное множество содержит бесконечное число непересекающихся счетных подмножеств.

	
\end{document}