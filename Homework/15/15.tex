\documentclass[a4paper,12pt]{article}
\usepackage{amsmath}
\usepackage{cmap}					% поиск в PDF
\usepackage{mathtext} 				% русские буквы в формулах
\usepackage[T2A]{fontenc}			% кодировка
\usepackage[utf8]{inputenc}			% кодировка исходного текста
\usepackage[english,russian]{babel}	% локализация и переносы

% Изменим формат \section и \subsection:
\usepackage{titlesec}
\titleformat{\section}
{\vspace{1cm}\centering\LARGE\bfseries}	% Стиль заголовка
{}										% префикс
{0pt}									% Расстояние между префиксом и заголовком
{} 										% Как отображается префикс
\titleformat{\subsection}				% Аналогично для \subsection
{\Large\bfseries}
{}
{0pt}
{}

%% Отступы между абзацами и в начале абзаца 
\setlength{\parindent}{0pt}
\setlength{\parskip}{\medskipamount}

% Перенос знаков в формулах (по Львовскому)
\newcommand*{\hm}[1]{#1\nobreak\discretionary{}
	{\hbox{$\mathsurround=0pt #1$}}{}}

%% Изменяем размер полей
\usepackage[top=1in, bottom=1in, left=1in, right=1in]{geometry}
\begin{document}
	\section{Домашнее задание 15\\ Шумилкин Андрей, группа 163} 
	\subsection{Задача 1}
	Как мы знаем, если некоторое множество $U$ бесконечно, а множество $V$ конечно или счетно, то $U \cup V$ равномощно $U$. \\
	Обозначим $C = A\setminus B$. Тогда мы можем записать $A = C \cup B$. Тогда по свойству, упомянотому выше мощность $C$ равна мощности $A$ и $C$ не может быть не бесконечным, поскольку объединение счетного множества со счетным будет не более, чем счетным. \\
	Значит все утверждение верно. 
	
	\subsection{Задача 2}
	Заметим, что $A \bigtriangleup B = (A \setminus B) \cup (B \setminus A)$. Как мы доказали в прошлой задаче $A \setminus B$ равномощно $A$. А $(B \setminus A)$ будет не более чем счетно, так как само $B$ счетно. \\
	И тогда из свойства, упомянутого в первой задаче и из транзитивности равномощности заметим, что раз $A \setminus B$ равномощно $A$ и  объединение бесконечного множества $U$ и счетного множества $V$ будет равномощно $U$, что $A \bigtriangleup B$ равномощно $A$. \\
	Значит все утверждение верно.
	
	\subsection{Задача 3}
	Как мы знаем, если некоторое множество $U$ бесконечно, а множество $V$ конечно или счетно, то $U \cup V$ равномощно $U$. \\
	Обозначим $C = A\setminus B$. Тогда мы можем записать $A = C \cup B$. Тогда по свойству, упомянотому выше мощность $C$ равна мощности $A$ и $C$ не может быть не бесконечным, поскольку объединение конечного множества со счетным будет не более, чем счетным. \\
	Значит все утверждение верно. 
	
	\subsection{Задача 4}
	Как нам известно множество рациональных чисел счетно и в каждом интервале найдется хотя бы одно рациональное число по аксиоме полноты. \\
	Тогда мы можем сопоставить кажому интервалу минимальную рациональную точку, находящуюся в нем. Такое сопоставление будет взаимно-однозначным, поскольку интервалы не пересекаются. \\
	Раз мы смогли найти такое сопоставление, то мощность множества данных интервалов не больше, чем мощность множества рациональных чисел, а оно счетно. Значит множество интервалов будет не более чем счетно.
	
	\subsection{Задача 5}
	Как мы знаем, всякое бесконечное множество содержит счетное подмножество. \\
	Это подмножество равномощно $\mathbb{N}$. Мы можем заметить, что и само $\mathbb{N}$ содержит бесконечное счетное число счетных подмножеств. Такими подмножествами будут, например, являться степени простых чисел, поскольку, как известно, простых чисел бесконечное число, а их степень так же принадлежит $\mathbb{N}$. И при этом они не будут пересекаться по основной теореме арифметики.\\
	Значит из первичной биекции некоторого подмножества нашего множества в $\mathbb{N}$ мы можем выделить сопоставления со множествами различных простых чисел в степени, которые являются счетными и которых бесконечное число, что и будет означать, что всякое бесконечное множество содержит бесконечное число непересекающихся счетных подмножеств.

	
\end{document}