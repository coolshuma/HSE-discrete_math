\documentclass[a4paper,12pt]{article}

%% Работа с русским языком
\usepackage{cmap}					% поиск в PDF
\usepackage{mathtext} 				% русские буквы в формулах
\usepackage[T2A]{fontenc}			% кодировка
\usepackage[utf8]{inputenc}			% кодировка исходного текста
\usepackage[english,russian]{babel}	% локализация и переносы
\usepackage{amsmath, amsfonts, amsthm, mathtools, amssymb, icomma, units}
\usepackage{algorithmicx, algorithm}
\usepackage{algpseudocode}

%% Отступы между абзацами и в начале абзаца 
\setlength{\parindent}{0pt}
\setlength{\parskip}{\medskipamount}

%% Изменяем размер полей
\usepackage[top=0.5in, bottom=0.75in, left=0.825in, right=0.825in]{geometry}

%% Графика
\usepackage[pdftex]{graphicx}
\graphicspath{{images/}}

%% Различные пакеты для работы с математикой
\usepackage{mathtools}				% Тот же amsmath, только с некоторыми поправками

%\usepackage{amssymb}				% Математические символы
\usepackage{amsthm}					% Пакет для написания теорем
\usepackage{amstext}
\usepackage{array}
\usepackage{amsfonts}
\usepackage{icomma}					% "Умная" запятая: $0,2$ --- число, $0, 2$ --- перечисление
\usepackage{bbm}				    % Для красивого (!) \mathbb с  буквами и цифрами
\usepackage{enumitem}               % Для выравнивания itemise (\begin{itemize}[align=left])

% Номера формул
\mathtoolsset{showonlyrefs=true} % Показывать номера только у тех формул, на которые есть \eqref{} в тексте.

% Ссылки
\usepackage[colorlinks=true, urlcolor=blue]{hyperref}

% Шрифты
\usepackage{euscript}	 % Шрифт Евклид
\usepackage{mathrsfs}	 % Красивый матшрифт

% Свои команды\textbf{}
\DeclareMathOperator{\sgn}{\mathop{sgn}}

% Перенос знаков в формулах (по Львовскому)
\newcommand*{\hm}[1]{#1\nobreak\discretionary{}
	{\hbox{$\mathsurround=0pt #1$}}{}}

% Графики
\usepackage{tikz}
\usepackage{pgfplots}
%\pgfplotsset{compat=1.12}

% Изменим формат \section и \subsection:
%\usepackage{titlesec}
%\titleformat{\section}
%{\vspace{1cm}\centering\LARGE\bfseries}	% Стиль заголовка
%{}										% префикс
%{0pt}									% Расстояние между префиксом и заголовком
%{} 										% Как отображается префикс
%\titleformat{\subsection}				% Аналогично для \subsection
%{\Large\bfseries}
%{}
%{0pt}
%{}

% Информация об авторах
\title{Лекции по предмету \\
	\textbf{Линейная алгебра и геометрия}}

\newtheorem*{Def}{Определение}
\newtheorem*{Lemma}{Лемма}
\newtheorem*{Suggestion}{Предложение}
\newtheorem*{Examples}{Пример}
%\newtheorem*{Comment}{Замечание}
\newtheorem*{Consequence}{Следствие}
\newtheorem*{Theorem}{Теорема}
\newtheorem*{Statement}{Утверждение}
\newtheorem*{Task}{Упражнение}
\newtheorem*{Designation}{Обозначение}
\newtheorem*{Generalization}{Обобщение}
\newtheorem*{Thedream}{Предел мечтаний}
\newtheorem*{Properties}{Свойства}


\renewcommand{\Re}{\mathrm{Re\:}}
\renewcommand{\Im}{\mathrm{Im\:}}
\newcommand{\Arg}{\mathrm{Arg\:}}
\renewcommand{\arg}{\mathrm{arg\:}}
\newcommand{\Mat}{\mathrm{Mat}}
\newcommand{\id}{\mathrm{id}}
\newcommand{\isom}{\xrightarrow{\sim}} 
\newcommand{\leftisom}{\xleftarrow{\sim}}
\newcommand{\Hom}{\mathrm{Hom}}
\newcommand{\Ker}{\mathrm{Ker}\:}
\newcommand{\rk}{\mathrm{rk}\:}
\newcommand{\diag}{\mathrm{diag}}
\newcommand{\ort}{\mathrm{ort}}
\newcommand{\pr}{\mathrm{pr}}
\newcommand{\vol}{\mathrm{vol\:}}
\def\limref#1#2{{#1}\negmedspace\mid_{#2}}
\newcommand{\eps}{\varepsilon}

\renewcommand{\epsilon}{\varepsilon}
\renewcommand{\phi}{\varphi}
\newcommand{\e}{\mathbb{e}}
\renewcommand{\l}{\lambda}
\renewcommand{\C}{\mathbb{C}}
\newcommand{\R}{\mathbb{R}}
\newcommand{\E}{\mathbb{E}}

\newcommand{\vvector}[1]{\begin{pmatrix}{#1}_1 \\\vdots\\{#1}_n\end{pmatrix}}
\renewcommand{\vector}[1]{({#1}_1, \ldots, {#1}_n)}

%Теоремы
%11.01.2016
\newtheorem*{standartbase}{Теорема о стандартном базисе}
\newtheorem*{fulllemma}{Лемма}
\newtheorem*{sl1}{Следствие 1}
\newtheorem*{sl2}{Следствие 2}
\newtheorem*{monotonousbase}{Теорема о монотонном базисе}
\newtheorem*{scheme}{Утверждение 1}
\newtheorem*{n2}{Утверждение 2}
\newtheorem*{usp-rais}{Теорема Успенского-Райса}
\newtheorem*{rec}{Свойство рекурсии}
\newtheorem*{point}{Теорема о неподвижной точке}
\newtheorem*{zhegalkin}{Теорема Жегалкина}
\newtheorem*{poste}{Теорема Поста}
\newtheorem*{algo1}{Первое свойство алгоритмов}

%18.01.2016
\newtheorem*{theorem}{Теорема}

\renewcommand{\qedsymbol}{\textbf{Q.E.D.}}
\newcommand{\definition}{\underline{Определение:} }
\newcommand{\definitions}{\underline{Определения:} }
\newcommand{\definitionone}{\underline{Определение 1:} }
\newcommand{\definitiontwo}{\underline{Определение 2:} }
\newcommand{\statement}{\underline{Утверждение:} }
\newcommand{\note}{\underline{Замечание:} }
\newcommand{\sign}{\underline{Обозначения:} }
\newcommand{\statements}{\underline{Утверждения:} }

\newcommand{\Z}{\mathbb{Z}}
\newcommand{\N}{\mathbb{N}}
\newcommand{\Q}{\mathbb{Q}}
\begin{document}
	\section{Домашнее задание 5} 
	\subsection{Задача 1}
	\% -- деление с остатком.
	а) Рассмотрим сначала исходный граф-цикл на 2n вершинах.Пронумеруем его вершины. Единтственная его возможная раскраска в два цвета -- это когда вершины с четными номерами покрашены в один цвет, а с нечетными -- в другой. Заметим, что противоположная вершина к текущей($t$) будет $(t + 2n/2) \% 2n$.  Осюда если n четное, то и номер противолежащей вершины к текущей вершине будет давать тоот же остаток при делении на 2, что и номер текущей вершины, т.к. (четное + четное) = четное и (нечетное + четное) = нечетное и деление с остатком так же оставляет ту же четность. 
	
	А вот если n -- нечетное, то номер противолежащаей вершины будет четным, если номер текущей вершины нечетен и нечетным иначе. 
	
	Значит при нечетных n граф можно расскрасить в два цвета, а при нечетных -- нет. 
	
	б) Если граф можно раскрасить в два цвета, то его, очевидно, можно раскрасить и в три цвета. То есть при нечетных n граф можно раскрасить в три цвета. Теперь рассмотрим случай четных n.
	
	Сразу отметим, что при n = 2, если провести все ребра соединяющие противоположные вершины, то получим полносвязный граф на четырех вершинах, который нельзя раскрасить в три цвета.
	
	Рассмотрим остальные графы при нечетных n. Пусть у нас будут 3 цвета: I, II, III.
	Покрасим первые n вершин чередуя цвета I и II. n + 1 и 2n покрасим в цвет III. А все, между n + 1 и 2n покрасим, опять же, чередуя I и II. Тогда исходный граф-цикл и вершины n + 1 и 2n и им противолежащие правильно раскрашены по построению. А также остальные тоже, потому что первая половина графа покрашена в чередующиеся I и II и вторая так же, но со "сдвигом" на одну вершину, то есть вершина, противолежащая к вершине из первой половины графа, будет как раз иметь другой цвет.   
	
	\subsection{Задача 2}
	
	а) Простой путь мы моежм строить только переходя из вершин одной доли в вершины другой доли и его длина будет  не больше, чем количество вершин в наименьшей доли +1 (так как мы можем вернуться еще в первую долю). Значит k <= 2 * min(n, m) + 1.
	
	б) Из рассуждений к задаче 5 видно, что при m=n в таком графе будет гамильтонов цикл длиной 2*n. Также мы заметили, что если вершин в одной из долей меньше, то нам потом будет "не хватать" вершин для завершения цикла. Отметим, что при построении простого цикла, также как и при построении Гимильтонова мы можем переходить из вершины одной доли только к вершинам другой доли, а так же нам в конце нужно вернуться в исходную долю. Из этого всего следует, что для цикла k <= 2 * min(m, n).  	
	 
	\subsection{Задача 3}
	Это граф, состоящий из двух компонент связности из 4 и 5 вершин, каждая из которых полносвязна. Тогда в первой будет четыре вершины со степенью 3, а во второй пять вершин со степенью 4, что и требуется по условию задачи. А так как наш граф несвязный, то он не содержит гамильтонов цикл по определению, то есть не является гамильтоновым графом.
	
	\subsection{Задача 5}
	а) Эйлеров цикл существует тогда и только тогда, когда граф связный и в нем отсутсвуют вершины нечетной степени. Так как в полном двудольном графе степень каждой вершины по определению равна количеству вершин в доле, в которой ее нет, то он Эйлеров цикл будет существовать в графе $K_{m,n}$ только при четных m и n.
	
	б) Так как граф двудольный, то при построении гамильтонова цикла мы можем переходить из вершины одной доли только  в вершины другой доли и при этом не можем посещать одну вершину более чем один раз. Отсюда следует, что m должно быть равно n, так как иначе в какой-то момент при построении нам просто не "хватит" вершин в одной доле, чтобы обойти оставшиеся в другой. И при этом оба числа должны быть больше одного, так как в графе из двух вершин, по одной в каждой доле, не будет гамильтонова цикла.
	
	в) Обозначим вершины первой доли как $1, 2, \dots 6$, а второй как $1i, 2i, \dots 6i$. 
	
	Тогда Гамильтонов цикл: $1 - 1i - 2 - 2i - 3 - 3i - 4 - 4i - 5 - 5i - 6 - 6i - 1$.
	
	\subsection{Задача 5}
	Добавим кратные ребра между теми вершинами, между которыми в исходном графе есть ребро. Тогда кол-во ребер инцендентных каждой вершине удвоится, а значит четные степени вершин останутся четными, а нечетные станут четными. А по теореме, доказанной Эйлером, Эйлеров цикл существует тогда и только тогда, когда граф связный и в нем отсутствуют вершины нечетной степени. 
	
	Полученный граф связен, потому что исходный граф был связен, а мы не удаляли ребра, а только добавляли их. То есть в полученном графе с кратными ребрами существует эйлеров цикл, то есть такой обход, в процессе которого мы пройдемся по каждому его ребру единожды. Теперь, если в порядке этого обхода прохождение по добавленным нами кратным ребрам рассматривать как прохождение по соответствующим ребрам исходного графа, мы как раз и получим такой обход в исходном графе, при котором мы пройдемся по каждому его ребру дважды. 
	
	
	\end{document}