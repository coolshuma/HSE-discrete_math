\documentclass[a4paper,12pt]{article}

%% Работа с русским языком
\usepackage{cmap}					% поиск в PDF
\usepackage{mathtext} 				% русские буквы в формулах
\usepackage[T2A]{fontenc}			% кодировка
\usepackage[utf8]{inputenc}			% кодировка исходного текста
\usepackage[english,russian]{babel}	% локализация и переносы
\usepackage{amsmath, amsfonts, amsthm, mathtools, amssymb, icomma, units}
\usepackage{algorithmicx, algorithm}
\usepackage{algpseudocode}

%% Отступы между абзацами и в начале абзаца 
\setlength{\parindent}{0pt}
\setlength{\parskip}{\medskipamount}

%% Изменяем размер полей
\usepackage[top=0.5in, bottom=0.75in, left=0.825in, right=0.825in]{geometry}

%% Графика
\usepackage[pdftex]{graphicx}
\graphicspath{{images/}}

%% Различные пакеты для работы с математикой
\usepackage{mathtools}				% Тот же amsmath, только с некоторыми поправками

%\usepackage{amssymb}				% Математические символы
\usepackage{amsthm}					% Пакет для написания теорем
\usepackage{amstext}
\usepackage{array}
\usepackage{amsfonts}
\usepackage{icomma}					% "Умная" запятая: $0,2$ --- число, $0, 2$ --- перечисление
\usepackage{bbm}				    % Для красивого (!) \mathbb с  буквами и цифрами
\usepackage{enumitem}               % Для выравнивания itemise (\begin{itemize}[align=left])

% Номера формул
\mathtoolsset{showonlyrefs=true} % Показывать номера только у тех формул, на которые есть \eqref{} в тексте.

% Ссылки
\usepackage[colorlinks=true, urlcolor=blue]{hyperref}

% Шрифты
\usepackage{euscript}	 % Шрифт Евклид
\usepackage{mathrsfs}	 % Красивый матшрифт

% Свои команды\textbf{}
\DeclareMathOperator{\sgn}{\mathop{sgn}}

% Перенос знаков в формулах (по Львовскому)
\newcommand*{\hm}[1]{#1\nobreak\discretionary{}
	{\hbox{$\mathsurround=0pt #1$}}{}}

% Графики
\usepackage{tikz}
\usepackage{pgfplots}
%\pgfplotsset{compat=1.12}

% Изменим формат \section и \subsection:
%\usepackage{titlesec}
%\titleformat{\section}
%{\vspace{1cm}\centering\LARGE\bfseries}	% Стиль заголовка
%{}										% префикс
%{0pt}									% Расстояние между префиксом и заголовком
%{} 										% Как отображается префикс
%\titleformat{\subsection}				% Аналогично для \subsection
%{\Large\bfseries}
%{}
%{0pt}
%{}

% Информация об авторах
\title{Лекции по предмету \\
	\textbf{Линейная алгебра и геометрия}}

\newtheorem*{Def}{Определение}
\newtheorem*{Lemma}{Лемма}
\newtheorem*{Suggestion}{Предложение}
\newtheorem*{Examples}{Пример}
%\newtheorem*{Comment}{Замечание}
\newtheorem*{Consequence}{Следствие}
\newtheorem*{Theorem}{Теорема}
\newtheorem*{Statement}{Утверждение}
\newtheorem*{Task}{Упражнение}
\newtheorem*{Designation}{Обозначение}
\newtheorem*{Generalization}{Обобщение}
\newtheorem*{Thedream}{Предел мечтаний}
\newtheorem*{Properties}{Свойства}


\renewcommand{\Re}{\mathrm{Re\:}}
\renewcommand{\Im}{\mathrm{Im\:}}
\newcommand{\Arg}{\mathrm{Arg\:}}
\renewcommand{\arg}{\mathrm{arg\:}}
\newcommand{\Mat}{\mathrm{Mat}}
\newcommand{\id}{\mathrm{id}}
\newcommand{\isom}{\xrightarrow{\sim}} 
\newcommand{\leftisom}{\xleftarrow{\sim}}
\newcommand{\Hom}{\mathrm{Hom}}
\newcommand{\Ker}{\mathrm{Ker}\:}
\newcommand{\rk}{\mathrm{rk}\:}
\newcommand{\diag}{\mathrm{diag}}
\newcommand{\ort}{\mathrm{ort}}
\newcommand{\pr}{\mathrm{pr}}
\newcommand{\vol}{\mathrm{vol\:}}
\def\limref#1#2{{#1}\negmedspace\mid_{#2}}
\newcommand{\eps}{\varepsilon}

\renewcommand{\epsilon}{\varepsilon}
\renewcommand{\phi}{\varphi}
\newcommand{\e}{\mathbb{e}}
\renewcommand{\l}{\lambda}
\renewcommand{\C}{\mathbb{C}}
\newcommand{\R}{\mathbb{R}}
\newcommand{\E}{\mathbb{E}}

\newcommand{\vvector}[1]{\begin{pmatrix}{#1}_1 \\\vdots\\{#1}_n\end{pmatrix}}
\renewcommand{\vector}[1]{({#1}_1, \ldots, {#1}_n)}

%Теоремы
%11.01.2016
\newtheorem*{standartbase}{Теорема о стандартном базисе}
\newtheorem*{fulllemma}{Лемма}
\newtheorem*{sl1}{Следствие 1}
\newtheorem*{sl2}{Следствие 2}
\newtheorem*{monotonousbase}{Теорема о монотонном базисе}
\newtheorem*{scheme}{Утверждение 1}
\newtheorem*{n2}{Утверждение 2}
\newtheorem*{usp-rais}{Теорема Успенского-Райса}
\newtheorem*{rec}{Свойство рекурсии}
\newtheorem*{point}{Теорема о неподвижной точке}
\newtheorem*{zhegalkin}{Теорема Жегалкина}
\newtheorem*{poste}{Теорема Поста}
\newtheorem*{algo1}{Первое свойство алгоритмов}

%18.01.2016
\newtheorem*{theorem}{Теорема}

\renewcommand{\qedsymbol}{\textbf{Q.E.D.}}
\newcommand{\definition}{\underline{Определение:} }
\newcommand{\definitions}{\underline{Определения:} }
\newcommand{\definitionone}{\underline{Определение 1:} }
\newcommand{\definitiontwo}{\underline{Определение 2:} }
\newcommand{\statement}{\underline{Утверждение:} }
\newcommand{\note}{\underline{Замечание:} }
\newcommand{\sign}{\underline{Обозначения:} }
\newcommand{\statements}{\underline{Утверждения:} }

\newcommand{\Z}{\mathbb{Z}}
\newcommand{\N}{\mathbb{N}}
\newcommand{\Q}{\mathbb{Q}}
\begin{document}
	\section{Домашнее задание 18\\ Шумилкин Андрей, группа 163} 
	\subsection{Задача 1}
	Заметим, что данное выражение истинно, когда в нем либо одна переменная равна 1, либо три переменных равны 1. \\
	Мы можем с помощью выражения $(x \land y \land z)$  записать ту часть выражения, которая истинна, когда все три переменных рывны 1. \\
	Тогда нам остается записать часть выражения, которая истинна тогда и только тогда, когда лишь одна переменная равна 1. \\
	Мы можем это сделать с помощью такого выражения: $(\lnot((x \land y) \lor (x \land z) \lor (y \land z)))\land (x \lor y \lor z)$. То есть мы сначала проверяем, что в каэждой паре переменных хотя бы один равен нулю, откуда и получаем, что всего будет не больше одной переменной, равной единице. А потом проверяем, что она именно одна, а не ноль. \\
	В итоге получаем: 
	\[
		(x \land y \land z) \lor (\lnot((x \land y) \lor (x \land z) \lor (y \land z)))\land (x \lor y \lor z).
	\]
	
	\subsection{Задача 2}
	Выпишем 9 наборов переменных, на которых функция истинна: \\
	\{0111, 1000, 1001, 1010, 1011, 1100, 1101, 1110, 1111\} \\
	Видим, что в восьми из них первая цифра истинна, а в одном -- три последних. Заметим, что в оставшихся наборах, на которых функция не истинна такого нет, поэтому мы можем записать ее просто в виде $x_1 \lor (x_2 \land x_3 \land x_4)$. \\
	Тогда схему можем записать как $g_1 = x_1,\ g_2 = x_2,\ g_3 = x_3,\ g_4 = x_4,\ g_ 5 = g_2 \land g_3,\ g_6 = g_4 \land g_5,\ g_7 = g_1 \land g_7$. Выход -- $g_7$.
	
	\subsection{Задача 3}
	Мы можем строить схему по выражению вида: $(x_1 \land \lnot x_2 \land x_3) \lor (x_2 \land \lnot x_3 \land x_4) \lor \ldots \lor (x_{n-2} \land \lnot x_{n-1} \land x_n)$. \\
	Тогда мы можем добавить на первом "уровне" 2n вершин в наш граф-схему: сами значения переменных $x_1, x_2, \ldots x_n$ и отрицания значений переменных $x_2, x_3, \ldots x_{n-1}$. \\
	Тогда на втором уровне будем делать конъюкицию первых двух переменных в каждой из скобок, то еcть $(x_1 \land \lnot x_2), (x_2 \land \lnot x_3), \ldots (x_{n-2} \land \lnot x_{n-1})$ -- на нем у нас будет n вершин. \\
	Тогда на третьем уровне будем делать конъюкцию значений в вершинах, полученных на втором уровне с третьей переменной в скобках. На это так же потребуется n вершин. \\
	После чего у нас останется n значений и нам нужно будет проверить равно ли хотя бы одно из них 1. Тогда мы просто сделаем дизъюнкцию первого значения со вторым, далее полученного значения с третьим и так далее. На выходе у нас как раз будет 1, если в исходном слове было подслово 101 и на эти дизъюнкции у нас уйдет n вершин. \\
	Тогда размер полученной схемы будет (из рассчета 2n + n + n + n) -- $O(6n) = O(n)$.  
	
	\subsection{Задача 4}
	Будем представлять двоичное число из n разрядов как набор переменных $x_{n-1}, x_{n-2}, \ldots x_0$. \\
	Сначала поймем, что простой путь умножить число на три в двоичном представлении -- это сначала удвоить его, а потом прибавить к результату его самого. \\
	Удвоение числа делается достаточно просто -- оно сдвигается влево на 1 бит. \\
	Теперь разберемся со сложением. Заметим, что при сложении двух данных чисел у нас получиться число с $n+1$ разрядом. Обозначим их как $y_{n}, x_{n-1}, \ldots y_0$. \\
	Начнем складывать. $y_0 = x_0$, поскольку второе число сдвинуто влево в этом разряде у него 0. \\
	Чтобы продолжить складывать нам нужно ввести дополнительную переменную, назовем их $z$, в которой мы будем хранить то число, которое будет переходить в следующий разряд. Т.е. $y_1 = x_1 \lor x_0$ и $z_1 = x_0 \land x_1$ и тогда $y_2 = x_2 \lor x_1 \lor z_1$. \\
	Тогда далее имеем: $y_i = x_i \lor x_{i-1} \lor z_{i-1}$,\\ $z_i = (x_{i-1} \land x_{i}) \lor (x_{i-1} \land z_{i-1}) \lor (x_{i} \land z_{i-1})$. \\
	И такое упрощение, а точнее то, что мы не повторяем вычисление пременных, а уже используем вычисленные раннее позволяет нам построить схему полиномиального размера. 
		
\end{document}