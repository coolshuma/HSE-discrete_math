\documentclass[a4paper,12pt]{article}
\usepackage{amsmath}
\usepackage{cmap}					% поиск в PDF
\usepackage{mathtext} 				% русские буквы в формулах
\usepackage[T2A]{fontenc}			% кодировка
\usepackage[utf8]{inputenc}			% кодировка исходного текста
\usepackage[english,russian]{babel}	% локализация и переносы

% Изменим формат \section и \subsection:
\usepackage{titlesec}
\titleformat{\section}
{\vspace{1cm}\centering\LARGE\bfseries}	% Стиль заголовка
{}										% префикс
{0pt}									% Расстояние между префиксом и заголовком
{} 										% Как отображается префикс
\titleformat{\subsection}				% Аналогично для \subsection
{\Large\bfseries}
{}
{0pt}
{}

%% Отступы между абзацами и в начале абзаца 
\setlength{\parindent}{0pt}
\setlength{\parskip}{\medskipamount}

% Перенос знаков в формулах (по Львовскому)
\newcommand*{\hm}[1]{#1\nobreak\discretionary{}
	{\hbox{$\mathsurround=0pt #1$}}{}}

%% Изменяем размер полей
\usepackage[top=1in, bottom=1in, left=1in, right=1in]{geometry}
\begin{document}
	\section{Домашнее задание 19\\ Шумилкин Андрей, группа 163} 
	\subsection{Задача 1}
	Если любая цифра числа $\pi$ вычислима, то и любая пятерка подряд идущих цифр в числе $\pi$ вычислима, т.е. мы моежм вычислять по числу из числа $\pi$  и для всем встречающимся пятеркам сопоставлять единицу в некоторой функции, а тогда и их множество разрешимо по определению, т.к. множество называется разрешимым, если его характеристическая функция вычислима. \\
	Хоть число $\pi$  и бесконечно, а значит мы точно не знаем все пятерки цифр, которые могут в нем встречаться, но всего таких пятерок может быть лищь $10^5$, а значит кол-во элементов в подмножестве, которое должен перечислить алгоритм также будет конечно. Любое конечное множество разрешимо и если множество разрешимо, то оно и перечислимо. \\
	
	\subsection{Задача 2}
	Раз мы можем перечислить элементы всего множества $X$, то мы точно так же можем перечислить элементы и необходимого подмножества $X$ -- чисел, сумма цифр которых равна 10. \\
	Т.е. мы можем <<идти>> по $X$ как при его перечислении, но включать в то множество, которое мы перечисляем только его элементы, соответсвующие условию. 
	
	\subsection{Задача 3}
	Мы можем описать алгоритм перечисления такого множества: берем декартово произведние множеств $A\times B$ и каждый его элемент выводим. Тогда получается, что это вычилсимая функиция из множества $A\times B$ в некоторое множество его значений, а множество значений вычислимой функции -- перечислимо. \\
	
	\subsection{Задача 4}
	По сути мы можем описать такой алгоритм, проверяющий некоторое свойство натуральных чисел: он на вход получает число и выводит и дает ответ 1(да), если данное число не принадлежит конечному подмножеству элементы которого не являются значениями рассматриваемой функции. \\
	Тогда полученное множество будет по определению разрешимым, так как его характиристическая функция вычилсима. А разрешимое множество также является перечислимым. \\
	А для непустого множество это равносильно тому, что оно является множеством значений некойвсюду определенной вычислимой функции, а значит рассматриваемая функция вычислима.
\end{document}