\documentclass[a4paper,12pt]{article}
\usepackage{amsmath}
\usepackage{cmap}					% поиск в PDF
\usepackage{mathtext} 				% русские буквы в формулах
\usepackage[T2A]{fontenc}			% кодировка
\usepackage[utf8]{inputenc}			% кодировка исходного текста
\usepackage[english,russian]{babel}	% локализация и переносы

% Изменим формат \section и \subsection:
\usepackage{titlesec}
\titleformat{\section}
{\vspace{1cm}\centering\LARGE\bfseries}	% Стиль заголовка
{}										% префикс
{0pt}									% Расстояние между префиксом и заголовком
{} 										% Как отображается префикс
\titleformat{\subsection}				% Аналогично для \subsection
{\Large\bfseries}
{}
{0pt}
{}

%% Отступы между абзацами и в начале абзаца 
\setlength{\parindent}{0pt}
\setlength{\parskip}{\medskipamount}

% Перенос знаков в формулах (по Львовскому)
\newcommand*{\hm}[1]{#1\nobreak\discretionary{}
	{\hbox{$\mathsurround=0pt #1$}}{}}

%% Изменяем размер полей
\usepackage[top=1in, bottom=1in, left=1in, right=1in]{geometry}
\begin{document}
	\section{Домашнее задание 7} 
	\subsection{Задача 1} 
	\textit{a)} Верно. $a = k \cdot c, b = l \cdot c + r$. Тогда $a + b = (k+l) \cdot c + r.$, а значит $a + b$ не делится на $с$. \\
	\textit{b)} Неверно. Контрпример: $a = 1, b = 5, c = 3$.\\
	1 не делится на 3 и 5 не делится на 3, но $1 + 5 = 6$ делится на 3.\\
	\textit{c)} Неверно. $a = k \cdot c + r, b = l \cdot c + t$. Тогда $a \cdot b = c \cdot(klc + kt + lr) + rt$. Значит если произведение остатков будет делиться на $c$, то и $ab$ будет делиться на $c$. Как контрпример можно привести: $a = 12, b = 15, c = 10$. 180 делится на 10.\\
	\textit{d)} Верно. $a = b \cdot l, b = c \cdot k$, тогда $a = c \cdot k \cdot l$ и $a \cdot b = c^2 \cdot k^2 \cdot l$, что, очевидно, делится на $c^2$.
	
	\subsection{Задача 2} 
	\textit{a)}При разложении каждого множителя факториала на простые двойка присутствует в 2(1), 4(2), 6(1), 8(3), 10(1), 12(2), 14(1), 16(4), 18(1), 20(2). В скобках указано в какой степени она в него входит. Всего получается $2^{18}$. Значи $20!$ при делении на $2^{15}$ даст остаток 0.\\
	\textit{b)} Остаток будет равен $2^{18}$. \\ 
	Как видно из предыдущего пункта 20! можно представить в виде $2^{18} \cdot (2k+1) = 2^{19} \cdot k + 2^{18}$.
	Видно, что $2^{19} \cdot k + 2^{18} \equiv 2^{18}  (mod\ 2^{19})$.
	
	\subsection{Задача 4}
	Воспользуемся алгоритмом Евклида.\\
	$(2^{2016} - 1, 2^{450} - 1).$\\
	$(2^{2016} - 1 - (2^{450} - 1), 2^{450} - 1).$\\
	$(2^{2016} - 2^{450}, 2^{450} - 1).$\\
	$(2^{2016} - 2 \cdot 2^{450} + 1, 2^{450} - 1).$\\
	$(2^{2016} - 2^{1566} \cdot 2^{450} + 2^{1566} - 1, 2^{450} - 1).$\\
	$(2^{1566} - 1, 2^{450} - 1).$\\
	Тогда далее:\\
	$(2^{1566} - 1, 2^{450} - 1).$\\
	$(2^{216} - 1, 2^{450} - 1).$\\
	$(2^{216} - 1, 2^{18} - 1).$\\
	$(2^{18} - 1, 2^{18} - 1).$\\
	$gcd(2^{2016} - 1, 2^{450} - 1) = 2^{18} - 1$.
	
	
	\subsection{Задача 5}
	$74 \cdot t \equiv 1\  (mod\  47)$.
	Решим диофантово уравнение вида 
	$74x + 47y = 1$\\
	$47x + 27y = 1$\\
	$27x + 20y = 1$\\
	$20x + 7y = 1$\\
	$7x + 6y = 1$\\
	$6x + y = 1$\\
	$x = 1$\\
	$y = -7$\\
	Значит искомый обратный элемент равен семи.
	
	\subsection{Задача 6}
	Решим диофантово уравнение вида 
	$39x + 221y = 104$\\
	$221x + 39y = 104$\\
	$39x + 26y = 104$\\
	$26x + 13y = 104$\\
	$13x = 104$\\
	$x = 8$\\
	$gcd(221, 39) = 13$.
	
	\subsection{Задача 7}
	Чтобы данное число делилось на 22 оно должно делиться на 2 и 11.
	Очевидно, что для делимости $n^{10} - 1$ на 2  $n$ должно быть нечетным.
	По малой теореме Ферма для любого $n$, не делящегося на 11, $n^{10}$ при делении на 11 даст остаток 1. Тогда $n^{10} - 1$ кратно 11. \\
	Значит $n^{10} - 1$ кратно 22 для всех нечетных $n$, не кратных 11.
\end{document}