\documentclass[a4paper,12pt]{article}

%% Работа с русским языком
\usepackage{cmap}					% поиск в PDF
\usepackage{mathtext} 				% русские буквы в формулах
\usepackage[T2A]{fontenc}			% кодировка
\usepackage[utf8]{inputenc}			% кодировка исходного текста
\usepackage[english,russian]{babel}	% локализация и переносы
\usepackage{amsmath, amsfonts, amsthm, mathtools, amssymb, icomma, units}
\usepackage{algorithmicx, algorithm}
\usepackage{algpseudocode}

%% Отступы между абзацами и в начале абзаца 
\setlength{\parindent}{0pt}
\setlength{\parskip}{\medskipamount}

%% Изменяем размер полей
\usepackage[top=0.5in, bottom=0.75in, left=0.825in, right=0.825in]{geometry}

%% Графика
\usepackage[pdftex]{graphicx}
\graphicspath{{images/}}

%% Различные пакеты для работы с математикой
\usepackage{mathtools}				% Тот же amsmath, только с некоторыми поправками

%\usepackage{amssymb}				% Математические символы
\usepackage{amsthm}					% Пакет для написания теорем
\usepackage{amstext}
\usepackage{array}
\usepackage{amsfonts}
\usepackage{icomma}					% "Умная" запятая: $0,2$ --- число, $0, 2$ --- перечисление
\usepackage{bbm}				    % Для красивого (!) \mathbb с  буквами и цифрами
\usepackage{enumitem}               % Для выравнивания itemise (\begin{itemize}[align=left])

% Номера формул
\mathtoolsset{showonlyrefs=true} % Показывать номера только у тех формул, на которые есть \eqref{} в тексте.

% Ссылки
\usepackage[colorlinks=true, urlcolor=blue]{hyperref}

% Шрифты
\usepackage{euscript}	 % Шрифт Евклид
\usepackage{mathrsfs}	 % Красивый матшрифт

% Свои команды\textbf{}
\DeclareMathOperator{\sgn}{\mathop{sgn}}

% Перенос знаков в формулах (по Львовскому)
\newcommand*{\hm}[1]{#1\nobreak\discretionary{}
	{\hbox{$\mathsurround=0pt #1$}}{}}

% Графики
\usepackage{tikz}
\usepackage{pgfplots}
%\pgfplotsset{compat=1.12}

% Изменим формат \section и \subsection:
%\usepackage{titlesec}
%\titleformat{\section}
%{\vspace{1cm}\centering\LARGE\bfseries}	% Стиль заголовка
%{}										% префикс
%{0pt}									% Расстояние между префиксом и заголовком
%{} 										% Как отображается префикс
%\titleformat{\subsection}				% Аналогично для \subsection
%{\Large\bfseries}
%{}
%{0pt}
%{}

% Информация об авторах
\title{Лекции по предмету \\
	\textbf{Линейная алгебра и геометрия}}

\newtheorem*{Def}{Определение}
\newtheorem*{Lemma}{Лемма}
\newtheorem*{Suggestion}{Предложение}
\newtheorem*{Examples}{Пример}
%\newtheorem*{Comment}{Замечание}
\newtheorem*{Consequence}{Следствие}
\newtheorem*{Theorem}{Теорема}
\newtheorem*{Statement}{Утверждение}
\newtheorem*{Task}{Упражнение}
\newtheorem*{Designation}{Обозначение}
\newtheorem*{Generalization}{Обобщение}
\newtheorem*{Thedream}{Предел мечтаний}
\newtheorem*{Properties}{Свойства}


\renewcommand{\Re}{\mathrm{Re\:}}
\renewcommand{\Im}{\mathrm{Im\:}}
\newcommand{\Arg}{\mathrm{Arg\:}}
\renewcommand{\arg}{\mathrm{arg\:}}
\newcommand{\Mat}{\mathrm{Mat}}
\newcommand{\id}{\mathrm{id}}
\newcommand{\isom}{\xrightarrow{\sim}} 
\newcommand{\leftisom}{\xleftarrow{\sim}}
\newcommand{\Hom}{\mathrm{Hom}}
\newcommand{\Ker}{\mathrm{Ker}\:}
\newcommand{\rk}{\mathrm{rk}\:}
\newcommand{\diag}{\mathrm{diag}}
\newcommand{\ort}{\mathrm{ort}}
\newcommand{\pr}{\mathrm{pr}}
\newcommand{\vol}{\mathrm{vol\:}}
\def\limref#1#2{{#1}\negmedspace\mid_{#2}}
\newcommand{\eps}{\varepsilon}

\renewcommand{\epsilon}{\varepsilon}
\renewcommand{\phi}{\varphi}
\newcommand{\e}{\mathbb{e}}
\renewcommand{\l}{\lambda}
\renewcommand{\C}{\mathbb{C}}
\newcommand{\R}{\mathbb{R}}
\newcommand{\E}{\mathbb{E}}

\newcommand{\vvector}[1]{\begin{pmatrix}{#1}_1 \\\vdots\\{#1}_n\end{pmatrix}}
\renewcommand{\vector}[1]{({#1}_1, \ldots, {#1}_n)}

%Теоремы
%11.01.2016
\newtheorem*{standartbase}{Теорема о стандартном базисе}
\newtheorem*{fulllemma}{Лемма}
\newtheorem*{sl1}{Следствие 1}
\newtheorem*{sl2}{Следствие 2}
\newtheorem*{monotonousbase}{Теорема о монотонном базисе}
\newtheorem*{scheme}{Утверждение 1}
\newtheorem*{n2}{Утверждение 2}
\newtheorem*{usp-rais}{Теорема Успенского-Райса}
\newtheorem*{rec}{Свойство рекурсии}
\newtheorem*{point}{Теорема о неподвижной точке}
\newtheorem*{zhegalkin}{Теорема Жегалкина}
\newtheorem*{poste}{Теорема Поста}
\newtheorem*{algo1}{Первое свойство алгоритмов}

%18.01.2016
\newtheorem*{theorem}{Теорема}

\renewcommand{\qedsymbol}{\textbf{Q.E.D.}}
\newcommand{\definition}{\underline{Определение:} }
\newcommand{\definitions}{\underline{Определения:} }
\newcommand{\definitionone}{\underline{Определение 1:} }
\newcommand{\definitiontwo}{\underline{Определение 2:} }
\newcommand{\statement}{\underline{Утверждение:} }
\newcommand{\note}{\underline{Замечание:} }
\newcommand{\sign}{\underline{Обозначения:} }
\newcommand{\statements}{\underline{Утверждения:} }

\newcommand{\Z}{\mathbb{Z}}
\newcommand{\N}{\mathbb{N}}
\newcommand{\Q}{\mathbb{Q}}
\begin{document}
	\section{Домашнее задание 11\\ Шумилкин Андрей, группа 163} 
	\subsection{Задача 1}
	Назовем функции $f$ и $g$. По определению функции она сопоставляет каждому элементу из области определения ровно один элемент из области значения. Пусть $A$ -- область опеределения для обеих функция, так как они обе определены всюду. $F$ -- область значений $f$, а $G$ -- область значений $g$.\\
	\textit{а)} Если объединение функций тоже функция, то она доложны всем элементам(так как обе функции всюду определены) из множества $A$ сопоставлять только лишь один элемент из множества $(A \cup B)$. Тогда если $\forall\ x\ in\ A \Rightarrow  f(x) = g(x)$, то объединение функций так же будет функцией. \\
	\textit{б)} Если пересечение функций тоже функция, то она доложны всем элементам(так как обе функции всюду определены) из множества $A$ сопоставлять только лишь один элемент из множества $(A \cap B)$. А так как в пересечении функций и так будут только лишь те элементы, для которых $f(x) = g(x)$, то оно всегда будет функцией. \\
	
	\subsection{Задача 2}
	Сразу приведем контрпример для $=\ и\ \subseteq$. Пусть $X = \{1, 2, 3, 4\}, Y=\{1, 2\}, A=\{1\}\ и\ f(1) = 1, f(2) = 1, f(3) = 1, f(4) = 2$. Тогда $f^{-1}(f(A)) = \{1, 2, 3\}$ и оно явно не равно $A$ а так же не является его подмножеством. \\
	А вот $\supseteq$ подходит по определению, поскольку $f(A) = \{y : y=f(x), x \in A\}\ и\ f^{-1}(f(A)) = \{x : f(x) \in f(A)\}$, а значит $\forall a \in A$ точно окажутся в прообразе, так как $f(a) \in f(A)$.
	
	\subsection{Задача 3}
	Сразу приведем контрпример для $=\ и\ \subseteq$. Пусть $A = \{1, 2\}, B=\{3, 4\} и\ f(1) = 1, f(2) = 1, f(3) = 1, f(4) = 2$. Тогда $f(A \setminus B) = \{1\}$, а $f(A) \setminus f(B) = \varnothing$. \\
	А вот $\supseteq$ подходит, поскольку в $f(A \setminus B)$ будут значения от всех элементов из $A$, которых нет в $B$. А вот в $f(A)$ могут быть значения элементов, которые есть в $A$ и при этом есть в $B$, но они пропадут, когда мы применим разность множеств, поскольку их значения так же будут и в $f(B)$. С ними так же могут пропасть и некоторые значения от элементов только из $A$, если $f(a) = f(b)$, но точно ничего не добавиться, поэтому это множество и будет подмножеством $f(A \setminus B).$ 
	
	\subsection{Задача 4}
	Сразу приведем контрпример для $=\ и\ \subseteq$. Пусть $A = \{1\}, B=\{2\} и\ f(1) = 1, f(2) = 1, f(3) = 1, f(4) = 2$. Тогда $f^{-1}(A \setminus B) = \{1\}$, а $f^{-1}(A) \setminus f^{-1}(B) = \varnothing$. \\
	
	\subsection{Задача 5}
	\textit{а)} Нам нужно каждому элементу из $A$ поставить в соответствие некий элемент из $B$ или не сопоставлять, считая, что для него функция не определена. Выбрать один элемент у нас $b+1$ способов(так как мы еще можем его вообще не брать) и это надо сделать $a$ раз. Значит количество способов это сделать $b^a$.\\
	\textit{б)} Нам нужно элементам из $A$ поставить в соответствие некий элемент из $B$, притом так как это инъекция все выбранные элементы из $B$ должны быть различны, но так же мы можем не сопоставлять некоторым элементам из $A$ никакие элементы из $B$ вообще. \\
	Будем определять для какого кол-ва элементов(обозначим $n$) функция опеределена. Тогда кол-во таких функций будет: $C_{a}^{n} \cdot C_{b}^n$ -- кол-во способов выбрать сами эти элементы, умноженное на кол-во способов выбрать сопоставленные им. Кол-во элементов, для которых определена функция может быть от нуля до $a$ и тогда итоговая формула:
	\[
		\sum_{n=0}^{a}C_{a}^{n} \cdot C_{b}^n.
	\]
	
	\subsection{Задача 6}
\end{document}
