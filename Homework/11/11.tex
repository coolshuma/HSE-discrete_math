\documentclass[a4paper,12pt]{article}
\usepackage{amsmath}
\usepackage{cmap}					% поиск в PDF
\usepackage{mathtext} 				% русские буквы в формулах
\usepackage[T2A]{fontenc}			% кодировка
\usepackage[utf8]{inputenc}			% кодировка исходного текста
\usepackage[english,russian]{babel}	% локализация и переносы

% Изменим формат \section и \subsection:
\usepackage{titlesec}
\titleformat{\section}
{\vspace{1cm}\centering\LARGE\bfseries}	% Стиль заголовка
{}										% префикс
{0pt}									% Расстояние между префиксом и заголовком
{} 										% Как отображается префикс
\titleformat{\subsection}				% Аналогично для \subsection
{\Large\bfseries}
{}
{0pt}
{}

%% Отступы между абзацами и в начале абзаца 
\setlength{\parindent}{0pt}
\setlength{\parskip}{\medskipamount}

% Перенос знаков в формулах (по Львовскому)
\newcommand*{\hm}[1]{#1\nobreak\discretionary{}
	{\hbox{$\mathsurround=0pt #1$}}{}}

%% Изменяем размер полей
\usepackage[top=1in, bottom=1in, left=1in, right=1in]{geometry}
\begin{document}
	\section{Домашнее задание 11\\ Шумилкин Андрей, группа 163} 
	\subsection{Задача 1}
	Назовем функции $f$ и $g$. По определению функции она сопоставляет каждому элементу из области определения ровно один элемент из области значения. Пусть $A$ -- область опеределения для обеих функция, так как они обе определены всюду. $F$ -- область значений $f$, а $G$ -- область значений $g$.\\
	\textit{а)} Если объединение функций тоже функция, то она доложны всем элементам(так как обе функции всюду определены) из множества $A$ сопоставлять только лишь один элемент из множества $(A \cup B)$. Тогда если $\forall\ x\ in\ A \Rightarrow  f(x) = g(x)$, то объединение функций так же будет функцией. \\
	\textit{б)} Если пересечение функций тоже функция, то она доложны всем элементам(так как обе функции всюду определены) из множества $A$ сопоставлять только лишь один элемент из множества $(A \cap B)$. А так как в пересечении функций и так будут только лишь те элементы, для которых $f(x) = g(x)$, то оно всегда будет функцией. \\
	
	\subsection{Задача 2}
	Сразу приведем контрпример для $=\ и\ \subseteq$. Пусть $X = \{1, 2, 3, 4\}, Y=\{1, 2\}, A=\{1\}\ и\ f(1) = 1, f(2) = 1, f(3) = 1, f(4) = 2$. Тогда $f^{-1}(f(A)) = \{1, 2, 3\}$ и оно явно не равно $A$ а так же не является его подмножеством. \\
	А вот $\supseteq$ подходит по определению, поскольку $f(A) = \{y : y=f(x), x \in A\}\ и\ f^{-1}(f(A)) = \{x : f(x) \in f(A)\}$, а значит $\forall a \in A$ точно окажутся в прообразе, так как $f(a) \in f(A)$.
	
	\subsection{Задача 3}
	Сразу приведем контрпример для $=\ и\ \subseteq$. Пусть $A = \{1, 2\}, B=\{3, 4\} и\ f(1) = 1, f(2) = 1, f(3) = 1, f(4) = 2$. Тогда $f(A \setminus B) = \{1\}$, а $f(A) \setminus f(B) = \varnothing$. \\
	А вот $\supseteq$ подходит, поскольку в $f(A \setminus B)$ будут значения от всех элементов из $A$, которых нет в $B$. А вот в $f(A)$ могут быть значения элементов, которые есть в $A$ и при этом есть в $B$, но они пропадут, когда мы применим разность множеств, поскольку их значения так же будут и в $f(B)$. С ними так же могут пропасть и некоторые значения от элементов только из $A$, если $f(a) = f(b)$, но точно ничего не добавиться, поэтому это множество и будет подмножеством $f(A \setminus B).$ 
	
	\subsection{Задача 4}
	Сразу приведем контрпример для $=\ и\ \subseteq$. Пусть $A = \{1\}, B=\{2\} и\ f(1) = 1, f(2) = 1, f(3) = 1, f(4) = 2$. Тогда $f^{-1}(A \setminus B) = \{1\}$, а $f^{-1}(A) \setminus f^{-1}(B) = \varnothing$. \\
	
	\subsection{Задача 5}
	\textit{а)} Нам нужно каждому элементу из $A$ поставить в соответствие некий элемент из $B$ или не сопоставлять, считая, что для него функция не определена. Выбрать один элемент у нас $b+1$ способов(так как мы еще можем его вообще не брать) и это надо сделать $a$ раз. Значит количество способов это сделать $b^a$.\\
	\textit{б)} Нам нужно элементам из $A$ поставить в соответствие некий элемент из $B$, притом так как это инъекция все выбранные элементы из $B$ должны быть различны, но так же мы можем не сопоставлять некоторым элементам из $A$ никакие элементы из $B$ вообще. \\
	Будем определять для какого кол-ва элементов(обозначим $n$) функция опеределена. Тогда кол-во таких функций будет: $C_{a}^{n} \cdot C_{b}^n$ -- кол-во способов выбрать сами эти элементы, умноженное на кол-во способов выбрать сопоставленные им. Кол-во элементов, для которых определена функция может быть от нуля до $a$ и тогда итоговая формула:
	\[
		\sum_{n=0}^{a}C_{a}^{n} \cdot C_{b}^n.
	\]
	
	\subsection{Задача 6}
\end{document}
