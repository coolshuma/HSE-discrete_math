\documentclass[a4paper,12pt]{article}
\usepackage{amsmath}
\usepackage{cmap}					% поиск в PDF
\usepackage{mathtext} 				% русские буквы в формулах
\usepackage[T2A]{fontenc}			% кодировка
\usepackage[utf8]{inputenc}			% кодировка исходного текста
\usepackage[english,russian]{babel}	% локализация и переносы

% Изменим формат \section и \subsection:
\usepackage{titlesec}
\titleformat{\section}
{\vspace{1cm}\centering\LARGE\bfseries}	% Стиль заголовка
{}										% префикс
{0pt}									% Расстояние между префиксом и заголовком
{} 										% Как отображается префикс
\titleformat{\subsection}				% Аналогично для \subsection
{\Large\bfseries}
{}
{0pt}
{}

%% Отступы между абзацами и в начале абзаца 
\setlength{\parindent}{0pt}
\setlength{\parskip}{\medskipamount}

% Перенос знаков в формулах (по Львовскому)
\newcommand*{\hm}[1]{#1\nobreak\discretionary{}
	{\hbox{$\mathsurround=0pt #1$}}{}}

%% Изменяем размер полей
\usepackage[top=1in, bottom=1in, left=1in, right=1in]{geometry}
\begin{document}
	\section{Домашнее задание 23\\ Шумилкин Андрей, группа 163} 
	\subsection{Задача 1}
	Как нам известно из тезиса Чёрна-Тьюринга всякая вычислимая функция вычислима машиной Тьюринга, а нигде не определенная функция является вычислимой, поскольку достаточно просто "зациклить" программу, то есть чтобы она уклонилась от выдачи какого-либо ответа  -- это и будет алгоритм ее вычисления. \\
	Значит существует МТ, которая вычисляет нигде не определенную функцию. \\
	Пусть $A$ -- входной алфавит и $q$ -- некотрый символ из $A \cup \{\Lambda\}$
	Тогда нам достаточно просто зациклить МТ и мы можем сделать это следующей таблице переходов:
	
	\[\delta: \mapsto
	\begin{cases}
	(q, 0) \mapsto (q(тот\ же\ самый), 0, +1)
	\end{cases}
	\]
	Где 0 -- начальное состояние. \\
	Действительно, согласно таблице, машина сначала просто "пройдет" по входным данным в виде числа, а потом продолжит двигаться вправо по пустым символам бесконечно, так как лента машины бесконечна. 
	
	\subsection{Задача 2}
	
	Приведем таблицу переходов:
	
	\[\delta: \mapsto
	\begin{cases}
	(0, 0) \mapsto (1, 0, +1) \\
	(1, 0) \mapsto (0, 0, +1) \\
	(\Lambda, 0) \mapsto (\Lambda, 0, 0) \\
	\end{cases}
	\]
	
	Где 0 -- начальное состояние.
	
	Заметим, что если головка находится над символом из алфавита $w$, то есть над 0 или 1, то она сдвинется вправо, предварительно записав в предыдущую ячейку "отрицание" текущего символа, то есть инвертирует его. \\
	Когда же головка дойдет до пробельного символа она просто остановится, при этом к этому моменту все символы входа из $w$ будут инвертированы, что по определению, данному в задаче, и будет представлять инвертированное входное слово. 
	
	\subsection{Задача 3}
	
	Пусть $q$ -- любая буква входного алфавита, а $l$ -- любой символ из символов входного алфавита и $\Lambda$.
	
	Приведем таблицу переходов:
	
	\[\delta: \mapsto
	\begin{cases}
	(a, 0) \mapsto (a, 1, +1) \\
	(b, 1) \mapsto (b, 2, +1) \\
	(a, 2) \mapsto (a, 3, +1) \\
	(q, 3) \mapsto (q, 3, +1) \\
	(\Lambda, 3) \mapsto (\Lambda, 4, -1) \\
	(q, 4) \mapsto (\Lambda, 4, -1) \\
	(\Lambda, 4) \mapsto (1, 5, +1) \\\\

	(a,1) \mapsto (a, 0, +1) \\
	(b,d \in \{0, 2\}) \mapsto (b, 0, +1) \\
	(c,d \in \{0, 1, 2\}) \mapsto (c, 0, +1) \\

	(\Lambda, d \in \{0, 1, 2\}) \mapsto (\Lambda, 6, -1) \\
	(q, 6) \mapsto (\Lambda, 6, -1) \\
	(\Lambda, 6) \mapsto (0, 5, +1) \\\\
	
	(\Lambda, 5) \mapsto (\Lambda, 5, 0) \\
	\end{cases}
	\]
	
	Где 0 -- начальное состояние. \\
	Действительно, если головка наткнется на последовательность $aba$, то машина перейдет в состояние 3, после чего просто сдвинется до ближайшего справа $\Lambda$, перейдет в состояние 4 и пойдет влево, до ближайшего $\Lambda$, попутно <<затирая>> все входные данные. Когда она дойдет до $\Lambda$, то у нас на ленте будут только пробельные символы, а значит достаточно лишь вывести ответ -- 1, поскольку мы в  состоянии 4, а в него можно попасть, только если мы нашли $aba$. Поэтому мы ставим один и переходим в состояние 5, а также сдвигаемся на одну клетку вправо, где стоит $\Lambda$ и, согласно таблице переходов, останавливаемся. Таким образом в случае нахождения последовательности символов $aba$ машина работает правильно. \\\\
	
	Если же машина не находит $aba$, то она так и идет до ближайшего справа $\Lambda$, не доходя до состояния 3 и когда она приходит в него она переходит в состояние 6 и идет до ближайшего слева $\Lambda$, попутно <<затирая>> входные данные, подобному случаю, когда мы все-таки нашли $aba$, только в данном случае у нее другое состояние --6. Придя в ближайшее слева $\Lambda$ она выводит ответ -- 0, и переходит в состояние 5, сдвигается вправо и останавливается. Таким образом в случае ненахождения последовательности символов $aba$ машина также работает правильно. \\\\
	
	\subsection{Задача 4}
	Опишем общую идею алгоритма, который будем реализовывать с помощью МТ. \\
	У нас сначала должны идти нули, а потом единицы, то есть по итогу у нас не должно быть подслов вида $10$, но кол-во 0 и 1 не должно измениться. Значит нам нужно преобразовывать $10$ в $01$. \\
	Тогда мы можем просто ходить по слову и искать самое левое $10$, преобразовывать его в $01$ и возвращаться в начало слова, и так до тех пор, пока $10$ не останется, то есть пока мы не дойдем до ближайшего справа $\Lambda$, не найдя ни одного. \\
	Мы можем реализовать следующим образом. Пусть $s_0$ -- начальное состояние. Начинаем двигаться влево, сохраняя его и символы, если встречаем нули. 
	Если же мы встречаем 1, то переходим в $s_1$ и сдвигаемся вправо далее, если далее встречаем 0, находясь в состоянии $s_1$, то переходим в $s_2$, пишем на его месте 1, сдвигаемся влево. Далее пишем 0, снова сдвигаемся влево и переходим в состояние $s_3$. \\
	В состоянии $s_3$ мы должны просто двигаться влево до ближайшего $Lambda$ и, дойдя до него, не изменяя его, сдвинуться вправо и перейти в состояние $s_0$ для которого уже описаны действия выше. \\
	Если же мы встречаем $\Lambda$ в состоянии $s_1$ или $s_0$, то мы просто останавливаемся, потому что это значит, что мы не встретили по пути $10$ и слово уже отсортированно в требующемся нам порядке.\\
	Поскольку мы смогли описать алгоритм в виде для МТ мы и можем построить такую МТ, то есть она существует. 
	

\end{document}