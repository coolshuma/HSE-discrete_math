\documentclass[a4paper,12pt]{article}
\usepackage{amsmath}
\usepackage{cmap}					% поиск в PDF
\usepackage{mathtext} 				% русские буквы в формулах
\usepackage[T2A]{fontenc}			% кодировка
\usepackage[utf8]{inputenc}			% кодировка исходного текста
\usepackage[english,russian]{babel}	% локализация и переносы

% Изменим формат \section и \subsection:
\usepackage{titlesec}
\titleformat{\section}
{\vspace{1cm}\centering\LARGE\bfseries}	% Стиль заголовка
{}										% префикс
{0pt}									% Расстояние между префиксом и заголовком
{} 										% Как отображается префикс
\titleformat{\subsection}				% Аналогично для \subsection
{\Large\bfseries}
{}
{0pt}
{}

%% Отступы между абзацами и в начале абзаца 
\setlength{\parindent}{0pt}
\setlength{\parskip}{\medskipamount}

% Перенос знаков в формулах (по Львовскому)
\newcommand*{\hm}[1]{#1\nobreak\discretionary{}
	{\hbox{$\mathsurround=0pt #1$}}{}}

%% Изменяем размер полей
\usepackage[top=1in, bottom=1in, left=1in, right=1in]{geometry}
\begin{document}
	\section{Домашнее задание 6} 
	\subsection{Задача 1} 
	Как известно, в дереве есть 2-раскраска. Так как в дереве 2n вершин, то кол-во вершин одного из цветов точно больше или равно n. Тогда мы можем просто выбрать n вершин этого цвета, а они по определению раскраски графа не будут соединены ребрами.
	
	\subsection{Задача 2} 
	Предположим, что число листьев меньше или равно половине вершин. Для определенности предположим, что равно. В нашем дереве n-1 ребро. Тогда по лемме о рукопожатиях сумма степеней всех вершин должна быть равна 2n - 2.
	
	Посчитаем теперь сумму степеней, получающуюся у нас. Для определенности будем считать, что, раз у не листовых вершин степень не равна 0(так как дерево связно), 1(так как листья мы уже отобрали) и 2(по условию задачи) оно равно трем. $n/2(cумма степеней листьев) + n/2*3 = 2n$, что больше оценки, полученной с помощью леммы у рукопожатиях. Получили противоречие. При этом, если мы решим сделать все-таки кол-во листьев меньше, чем n/2, то сумма степеней вершин будет увеличиваться, потому что "переделывая" лист во внутреннюю вершину мы увеличиваем его степень как минимум на два(потому что миимальная степень три, как было показано выше).
	
	Значит в дереве обязательно должно быть больше половины листьев.
	
	\subsection{Задача 4}
	Сразу заметим, что кубики, которые находятся на границе, можно "отбросить", только учтя, что  нам необходимо сделать к ним "выход"\ из внутренних кубиков. 
	
	Тогда внутренних кубиков останется $(n-2)^3$. Значит всего компонент связности графа $(n-2)^3 + 1$, учитывая необходимость связать все внутренние кубики хотя бы с одним внешним.
	
	Значит, чтобы сделать одну компоненту связности, нам нужно добавить как минимум $(n-2)^3$ ребер, то есть, говоря в терминах задачи, удалить $(n-2)^3$ перегородок.
	
	\subsection{Задача 5}
	Сделаем ориентированный граф из всех цифр и соединим ребрами те, которые образуют двузначное число, делящееся на 7. Направлено ребро из той вершины, которое обозначает десятки в числе в ту, которая обозначает единицы.

	Тогда для поиска необходимого числа мы можем просто начать с как можно большего числа и переходить в максимально возможное, при этом помня, что можно заходить в каждую вершину только один раз. 
	
	На построенном графе(рисунок прикрепил к письму) такой обход будет таким: начнем с 9, как с самого большого, перейдем в 8, оттуда в 4 из 4 в 2 и оттуда в 1. Получится 98421.
	
	Это число удовлетворяет условию о том, что соседние числа должны образовывать двузначное, делящееся на семь и оно наибольшее, так как в графе мы начали в максимально возможной цифре и переходили в максимально возможную. 
	
	\subsection{Задача 6}
	Начнем обход в ширину из той вершины, которую мы желаем сделать достижимой из всех остальных. То есть сначала перейдем во все доступные из нее вершины, потом из этих вершин во все доступные и них. При этом каждый раз переходя от одной вершины к другой, будем добавлять ребро, направление которого противоположно направлению нашего перемещения между этими двумя вершинами. 
	Тогда в итоге мы получим граф из каждой вершины которого мы можем попасть в нашу начальную вершину, поскольку в исходном неориентированном графе существовал путь от нашей начальной вершины до некоторой другой и мы, проходя по этому пути добавляли ребра, ведущие по направлению к нашей начальной вершине. 
	
	\subsection{Задача 7}
	Докажем по индукции. 
	
	\textit{База.}\\
	n=3. Как бы не направили ребра в таком графе"треугольнике"\ мы можем просто начать с той вершины из которой будет путь длиной два, а такой точно будет и который и будет являться ггамильтоновым. \\
	\textit{Шаг.} \\
	Пусть существует гамильтонов обход в любом турнире на n вершинах. Докажем, что он существует так же и в графе на n+1 вершине.
	
	Рассмотрим граф на n+1 вершине. Выберем в нем любую вершину $v$. Тогда в графе с удаленной вершиной $v$ и всеми инцидентными ей ребрами, существует гамильтонов путь по предположению индукции. Обозначим данный путь как $g_1 \to g_2 \to \dots \to g_n$.
	
	В нашем графе на n+1 вершине по определению должно быть либо ребро $v \to g_1$, либо ребро $g_1 \to v$. Если у нас есть ребро $v \to g_1$, то все -- мы нашли гамильтонов путь в данном графе, то есть мы прееходим из нашей вершины по ребру в гамильтонов путь на графе из n вершин и получаем как раз гамильтонов путь на нашем графе.
	
	Если же у нас ребро $g_1 \to v$, то посмотрим есть ли такое ребро, которое ведет из нашей вершины в  гамильтонов путь на графе из n вершин, т.е. $v \to g_t$. Если есть, то возьмем вершину с минимальным t, а значит все ребра до этого будут вести в нашу вершину $v$ и тогда мы можем просто перейти из предыдущей вершины в гамильтоновом пути -- $g_{t-1}$ -- в нашу вершину $v$ и это и получится гамильтонов путь в графе на n+1 вершине.
	
	Если же у нас нет такой вершины, в которую ведет ребро из нашей вершины, то тогда обязательно есть ребро $g_n \to v$, то есть пройдя по гамильтонову пути для графа на n вершинах мы перейдем в нашу вершину $v$ по данному ребру, а это и есть гамильтонов путь для графа на n+1 вершине. 
	
	Получается, что в любом турнире мы можем построить гамильтонов путь. 
	
	
	\end{document}