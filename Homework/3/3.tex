\documentclass[a4paper,12pt]{article}

%% Работа с русским языком
\usepackage{cmap}					% поиск в PDF
\usepackage{mathtext} 				% русские буквы в формулах
\usepackage[T2A]{fontenc}			% кодировка
\usepackage[utf8]{inputenc}			% кодировка исходного текста
\usepackage[english,russian]{babel}	% локализация и переносы
\usepackage{amsmath, amsfonts, amsthm, mathtools, amssymb, icomma, units}
\usepackage{algorithmicx, algorithm}
\usepackage{algpseudocode}

%% Отступы между абзацами и в начале абзаца 
\setlength{\parindent}{0pt}
\setlength{\parskip}{\medskipamount}

%% Изменяем размер полей
\usepackage[top=0.5in, bottom=0.75in, left=0.825in, right=0.825in]{geometry}

%% Графика
\usepackage[pdftex]{graphicx}
\graphicspath{{images/}}

%% Различные пакеты для работы с математикой
\usepackage{mathtools}				% Тот же amsmath, только с некоторыми поправками

%\usepackage{amssymb}				% Математические символы
\usepackage{amsthm}					% Пакет для написания теорем
\usepackage{amstext}
\usepackage{array}
\usepackage{amsfonts}
\usepackage{icomma}					% "Умная" запятая: $0,2$ --- число, $0, 2$ --- перечисление
\usepackage{bbm}				    % Для красивого (!) \mathbb с  буквами и цифрами
\usepackage{enumitem}               % Для выравнивания itemise (\begin{itemize}[align=left])

% Номера формул
\mathtoolsset{showonlyrefs=true} % Показывать номера только у тех формул, на которые есть \eqref{} в тексте.

% Ссылки
\usepackage[colorlinks=true, urlcolor=blue]{hyperref}

% Шрифты
\usepackage{euscript}	 % Шрифт Евклид
\usepackage{mathrsfs}	 % Красивый матшрифт

% Свои команды\textbf{}
\DeclareMathOperator{\sgn}{\mathop{sgn}}

% Перенос знаков в формулах (по Львовскому)
\newcommand*{\hm}[1]{#1\nobreak\discretionary{}
	{\hbox{$\mathsurround=0pt #1$}}{}}

% Графики
\usepackage{tikz}
\usepackage{pgfplots}
%\pgfplotsset{compat=1.12}

% Изменим формат \section и \subsection:
%\usepackage{titlesec}
%\titleformat{\section}
%{\vspace{1cm}\centering\LARGE\bfseries}	% Стиль заголовка
%{}										% префикс
%{0pt}									% Расстояние между префиксом и заголовком
%{} 										% Как отображается префикс
%\titleformat{\subsection}				% Аналогично для \subsection
%{\Large\bfseries}
%{}
%{0pt}
%{}

% Информация об авторах
\title{Лекции по предмету \\
	\textbf{Линейная алгебра и геометрия}}

\newtheorem*{Def}{Определение}
\newtheorem*{Lemma}{Лемма}
\newtheorem*{Suggestion}{Предложение}
\newtheorem*{Examples}{Пример}
%\newtheorem*{Comment}{Замечание}
\newtheorem*{Consequence}{Следствие}
\newtheorem*{Theorem}{Теорема}
\newtheorem*{Statement}{Утверждение}
\newtheorem*{Task}{Упражнение}
\newtheorem*{Designation}{Обозначение}
\newtheorem*{Generalization}{Обобщение}
\newtheorem*{Thedream}{Предел мечтаний}
\newtheorem*{Properties}{Свойства}


\renewcommand{\Re}{\mathrm{Re\:}}
\renewcommand{\Im}{\mathrm{Im\:}}
\newcommand{\Arg}{\mathrm{Arg\:}}
\renewcommand{\arg}{\mathrm{arg\:}}
\newcommand{\Mat}{\mathrm{Mat}}
\newcommand{\id}{\mathrm{id}}
\newcommand{\isom}{\xrightarrow{\sim}} 
\newcommand{\leftisom}{\xleftarrow{\sim}}
\newcommand{\Hom}{\mathrm{Hom}}
\newcommand{\Ker}{\mathrm{Ker}\:}
\newcommand{\rk}{\mathrm{rk}\:}
\newcommand{\diag}{\mathrm{diag}}
\newcommand{\ort}{\mathrm{ort}}
\newcommand{\pr}{\mathrm{pr}}
\newcommand{\vol}{\mathrm{vol\:}}
\def\limref#1#2{{#1}\negmedspace\mid_{#2}}
\newcommand{\eps}{\varepsilon}

\renewcommand{\epsilon}{\varepsilon}
\renewcommand{\phi}{\varphi}
\newcommand{\e}{\mathbb{e}}
\renewcommand{\l}{\lambda}
\renewcommand{\C}{\mathbb{C}}
\newcommand{\R}{\mathbb{R}}
\newcommand{\E}{\mathbb{E}}

\newcommand{\vvector}[1]{\begin{pmatrix}{#1}_1 \\\vdots\\{#1}_n\end{pmatrix}}
\renewcommand{\vector}[1]{({#1}_1, \ldots, {#1}_n)}

%Теоремы
%11.01.2016
\newtheorem*{standartbase}{Теорема о стандартном базисе}
\newtheorem*{fulllemma}{Лемма}
\newtheorem*{sl1}{Следствие 1}
\newtheorem*{sl2}{Следствие 2}
\newtheorem*{monotonousbase}{Теорема о монотонном базисе}
\newtheorem*{scheme}{Утверждение 1}
\newtheorem*{n2}{Утверждение 2}
\newtheorem*{usp-rais}{Теорема Успенского-Райса}
\newtheorem*{rec}{Свойство рекурсии}
\newtheorem*{point}{Теорема о неподвижной точке}
\newtheorem*{zhegalkin}{Теорема Жегалкина}
\newtheorem*{poste}{Теорема Поста}
\newtheorem*{algo1}{Первое свойство алгоритмов}

%18.01.2016
\newtheorem*{theorem}{Теорема}

\renewcommand{\qedsymbol}{\textbf{Q.E.D.}}
\newcommand{\definition}{\underline{Определение:} }
\newcommand{\definitions}{\underline{Определения:} }
\newcommand{\definitionone}{\underline{Определение 1:} }
\newcommand{\definitiontwo}{\underline{Определение 2:} }
\newcommand{\statement}{\underline{Утверждение:} }
\newcommand{\note}{\underline{Замечание:} }
\newcommand{\sign}{\underline{Обозначения:} }
\newcommand{\statements}{\underline{Утверждения:} }

\newcommand{\Z}{\mathbb{Z}}
\newcommand{\N}{\mathbb{N}}
\newcommand{\Q}{\mathbb{Q}}
\begin{document}
	\section{Домашнее задание 2} 
	Примечание: я запутался с обозначением $C_{n}^{k}$ и $\left(_{k}^{n}\right)$, подумав, что в формуле с буквой С большее число так же пишется сверху. Это оказалось не так, но, к сожалению, я узнал об этом слишком поздно, а этой формулы в работе достаточно и я просто не успел ее поправить во всех местах, поэтому, во избежание путаницы, везде оставил ее в виде $C_{k}^{n}$,  т.е. с написанием большего числа сверху.
	
	\subsection{Задача 2}
	Начнем рассматривать комнаты по пордяку. Способов поселить в первую двухместную комнату у нас $C_{2}^{18}$. Далее способов поселить во вторую уже $C_{2}^{16}$. Человек осталось 16, так как двоих мы уже "поселили".
	Далее так же продолжим рассуждать для трехместных и четырехместных комнат. Тогда итоговая формула примет вид:
	$C_{2}^{18} \cdot C_{2}^{16} \cdot C_{3}^{14} \cdot C_{3}^{11} \cdot C_{4}^{8} \cdot C_{4}^{4} = 77189112000$.
	
	\subsection{Задача 3}
	a)
	
	\[
		\left(_{m}^{n}\right) \cdot \left(_{k}^{m}\right) = \left(_{k}^{n}\right) \cdot \left(_{m - k}^{n - k}\right).
	 \]
	 \[
		 \frac{n!}{m!*(n-m)!} \cdot  \frac{m!}{k!*(m-k)!} = \frac{n!}{k!*(n-k)!} \cdot  \frac{(n-k)!}{(m-k)!*(n-k-(m-k))!}
	 \]
	 \[
		 \frac{n!}{(n-m)!} \cdot  \frac{1}{k!*(m-k)!} = \frac{n!}{k!} \cdot  \frac{1}{(m-k)!*(n-k-(m-k))!}
	 \]
	 
	 Разделим обе части уравнения на $n!$ и домножим на $(m-k)!$.
	 \[
		 \frac{1}{(n-m)!} \cdot  \frac{1}{k!} = \frac{1}{k!} \cdot  \frac{1}{(n-k-(m-k))!}
	 \]
	 
	 Домножим обе части уравнения на $k!*(n-m)!$.
	 \[
		 1 = 1
	 \]
	 
	
	\subsection{Задача 4}
	Отнимем от общего числа акций пять, так как людей пять и каждому должно достаться хотя бы по одной акции, эти мы как бы изначально "раздадим"\  им по одной. 
	Теперь мы можем просто воспользоваться методом перегородок, при котором кому-то может и не достаться акций вообще, но ответ будет верный, так как мы уже раздали по одной акции.
	У нас будет 95 акций и четыре перегородки: 
	$C_{95}^{99} = \frac{99!}{95! * 4!} = \frac{96*97*98*99}{24} = 3764376$.
	
	\subsection{Задача 5}
	Воспользуемся методом перегородок, при этом будем помнить, что в пределах дня порядок пациентов важен. Перегородки будет четыре. Тогда количество всех перестановок элементов и перегородок равно $10!$.
	Но нас не интересуют перестановки самих перегородок, нас интересуют только перестановки пациентов. Тогда, чтобы не учитывать перестановки четырех перегородок, нам нужно количество перестановок всех разделить на количество перестановок, которое равно $4!$.
	Итоговая формула примет вид $10!/4!$.
	
	\subsection{Задача 6}
	
	а) Будем рассматривать количество способов проголосовать для одного человека. Так как он может голосовать за любого, в том числе и за себя, способов $n$. У следующего так же $n$. Тогда выходит, что есть $n$ ччеловек, каждый из которых может проголосовать $n$ способaми, тогда итоговая формула будет $n^n$.

	б) Слегка прееформулируем задачу. У нас есть n голосов, которые нужно распреедлить между n кандидатами. Видно, что можем воспользоваться методом перегородок, при этом перегородок $n-1$.
	Тогда итоговая формула $C_{n-1}^{2n-1}$.
	
	\subsection{Задача 7}
	Рассмотрим сначала пересечение двух событий: играющей музыки и идущего дождя, которые продолжались, соответственно, $90\%$ и $50\%$. Очевидно, что только 10\% не заполнено первым событием, а значит их пересечение не может быть меньше 40\%.
	Теперь будем рассматривать пересечение объединения первых двух событий с третьим событием, которое продолжалось 80\% времени -- выключенным светом. Очевидно, что только пересечение третьего события с подсчитанным уже пересечением первых двух и будет составлять ответ.
	Опять же видно, что третее событие не заполняет лишь 20\%, а значит его пересечение с пересечением первых двух, составляющим 40\%, будет никак не меньше 20\%, что и будет ответом. 
	
	Ответ: 20\%. 	
	
	\subsection{Задача 9}
	
	Из материала учебника мы знаем, что найти кол-во решений уравнения в целых неотрицательных числах $x_1 + x_2 + \dots + x_k = n$ мы можем методом перегородок, формулой $C_{k - 1}^{n + k - 1}$.
	В задаче везде под "количество решений уравнения" \  подразумевается кол-во решений в целых неотрицательных числах.
	
	а) Посчитаем количество решений данного уравнения, когда $x_1 > 3$. Воспользуемся методом преегородок, перегородок три, так как переменных 4. При этом сначала отнимем от десяти четыре для того, чтобы изначально сделать x1 = 4, чтобы когда мы будем считать перегородками и в x1 не "уйдет" ничего, все равно выполнялось условие. Тогда количество решений при $x_1 > 3$ равно $C_{3}^{9} = 84$.
	
	Теперь мы можем просто посчитать вышеописанным способом общее количество решений данного уравнения, отнять от него кол-во неподходящих решений и получить ответ. Общее кол-во решений равно $C_{3}^{13} = 286$.
	
	А ответ будет $286 - 84 = 202$.
	
	б) Теперь будем рассматривать четыре случай, когда $x_1=0, x_1=1, x_1=2, x_1=3$. Для каждого случая мы можем посчитать количество решений уравнения еще с условием $x_2 \le 3$ аналогично тому, как мы делали в предыдущем пункте. 
	К примеру, рассмотрим первый вариант, когда $x_1=0$. 
	
	Тогда общее кол-во решений уравнения можно посчитать, только теперь переменных уже осталось три, так как одну мы определили, а значит перегородок будет две.  Общее кол-во решений равно $C_{2}^{12}$. Теперь нам остается только посчитать кол-во "лишних"\  решений, т.е. когда $x_2 > 3$. Это мы можем сделать аналогично прошлому пункту, то есть отняв от десяти четыре, как бы изначально приравняв $x_2$ к четырем и посчитав кол-во решений для уравнения. Таких решений будет $C_{2}^{8}$. А подходящих нам решений будет $ C_{2}^{12} - C_{2}^{8} = 38$.
	
	Аналогично мы можем посчитать для $x_1=1, x_1=2, x_1=3$, только сначала нужно не забыть отнять текущее значение $x_1$ от десяти.
	Ответы для этих случаев будут соответсвенно 34, 30 и 26.
	Тогда общее кол-во решений это кол-во решений уравнения по всем случаям и оно равно 128.
	
	
	\end{document}