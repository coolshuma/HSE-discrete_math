\documentclass[a4paper,12pt]{article}

%% Работа с русским языком
\usepackage{cmap}					% поиск в PDF
\usepackage{mathtext} 				% русские буквы в формулах
\usepackage[T2A]{fontenc}			% кодировка
\usepackage[utf8]{inputenc}			% кодировка исходного текста
\usepackage[english,russian]{babel}	% локализация и переносы
\usepackage{amsmath, amsfonts, amsthm, mathtools, amssymb, icomma, units}
\usepackage{algorithmicx, algorithm}
\usepackage{algpseudocode}

%% Отступы между абзацами и в начале абзаца 
\setlength{\parindent}{0pt}
\setlength{\parskip}{\medskipamount}

%% Изменяем размер полей
\usepackage[top=0.5in, bottom=0.75in, left=0.825in, right=0.825in]{geometry}

%% Графика
\usepackage[pdftex]{graphicx}
\graphicspath{{images/}}

%% Различные пакеты для работы с математикой
\usepackage{mathtools}				% Тот же amsmath, только с некоторыми поправками

%\usepackage{amssymb}				% Математические символы
\usepackage{amsthm}					% Пакет для написания теорем
\usepackage{amstext}
\usepackage{array}
\usepackage{amsfonts}
\usepackage{icomma}					% "Умная" запятая: $0,2$ --- число, $0, 2$ --- перечисление
\usepackage{bbm}				    % Для красивого (!) \mathbb с  буквами и цифрами
\usepackage{enumitem}               % Для выравнивания itemise (\begin{itemize}[align=left])

% Номера формул
\mathtoolsset{showonlyrefs=true} % Показывать номера только у тех формул, на которые есть \eqref{} в тексте.

% Ссылки
\usepackage[colorlinks=true, urlcolor=blue]{hyperref}

% Шрифты
\usepackage{euscript}	 % Шрифт Евклид
\usepackage{mathrsfs}	 % Красивый матшрифт

% Свои команды\textbf{}
\DeclareMathOperator{\sgn}{\mathop{sgn}}

% Перенос знаков в формулах (по Львовскому)
\newcommand*{\hm}[1]{#1\nobreak\discretionary{}
	{\hbox{$\mathsurround=0pt #1$}}{}}

% Графики
\usepackage{tikz}
\usepackage{pgfplots}
%\pgfplotsset{compat=1.12}

% Изменим формат \section и \subsection:
%\usepackage{titlesec}
%\titleformat{\section}
%{\vspace{1cm}\centering\LARGE\bfseries}	% Стиль заголовка
%{}										% префикс
%{0pt}									% Расстояние между префиксом и заголовком
%{} 										% Как отображается префикс
%\titleformat{\subsection}				% Аналогично для \subsection
%{\Large\bfseries}
%{}
%{0pt}
%{}

% Информация об авторах
\title{Лекции по предмету \\
	\textbf{Линейная алгебра и геометрия}}

\newtheorem*{Def}{Определение}
\newtheorem*{Lemma}{Лемма}
\newtheorem*{Suggestion}{Предложение}
\newtheorem*{Examples}{Пример}
%\newtheorem*{Comment}{Замечание}
\newtheorem*{Consequence}{Следствие}
\newtheorem*{Theorem}{Теорема}
\newtheorem*{Statement}{Утверждение}
\newtheorem*{Task}{Упражнение}
\newtheorem*{Designation}{Обозначение}
\newtheorem*{Generalization}{Обобщение}
\newtheorem*{Thedream}{Предел мечтаний}
\newtheorem*{Properties}{Свойства}


\renewcommand{\Re}{\mathrm{Re\:}}
\renewcommand{\Im}{\mathrm{Im\:}}
\newcommand{\Arg}{\mathrm{Arg\:}}
\renewcommand{\arg}{\mathrm{arg\:}}
\newcommand{\Mat}{\mathrm{Mat}}
\newcommand{\id}{\mathrm{id}}
\newcommand{\isom}{\xrightarrow{\sim}} 
\newcommand{\leftisom}{\xleftarrow{\sim}}
\newcommand{\Hom}{\mathrm{Hom}}
\newcommand{\Ker}{\mathrm{Ker}\:}
\newcommand{\rk}{\mathrm{rk}\:}
\newcommand{\diag}{\mathrm{diag}}
\newcommand{\ort}{\mathrm{ort}}
\newcommand{\pr}{\mathrm{pr}}
\newcommand{\vol}{\mathrm{vol\:}}
\def\limref#1#2{{#1}\negmedspace\mid_{#2}}
\newcommand{\eps}{\varepsilon}

\renewcommand{\epsilon}{\varepsilon}
\renewcommand{\phi}{\varphi}
\newcommand{\e}{\mathbb{e}}
\renewcommand{\l}{\lambda}
\renewcommand{\C}{\mathbb{C}}
\newcommand{\R}{\mathbb{R}}
\newcommand{\E}{\mathbb{E}}

\newcommand{\vvector}[1]{\begin{pmatrix}{#1}_1 \\\vdots\\{#1}_n\end{pmatrix}}
\renewcommand{\vector}[1]{({#1}_1, \ldots, {#1}_n)}

%Теоремы
%11.01.2016
\newtheorem*{standartbase}{Теорема о стандартном базисе}
\newtheorem*{fulllemma}{Лемма}
\newtheorem*{sl1}{Следствие 1}
\newtheorem*{sl2}{Следствие 2}
\newtheorem*{monotonousbase}{Теорема о монотонном базисе}
\newtheorem*{scheme}{Утверждение 1}
\newtheorem*{n2}{Утверждение 2}
\newtheorem*{usp-rais}{Теорема Успенского-Райса}
\newtheorem*{rec}{Свойство рекурсии}
\newtheorem*{point}{Теорема о неподвижной точке}
\newtheorem*{zhegalkin}{Теорема Жегалкина}
\newtheorem*{poste}{Теорема Поста}
\newtheorem*{algo1}{Первое свойство алгоритмов}

%18.01.2016
\newtheorem*{theorem}{Теорема}

\renewcommand{\qedsymbol}{\textbf{Q.E.D.}}
\newcommand{\definition}{\underline{Определение:} }
\newcommand{\definitions}{\underline{Определения:} }
\newcommand{\definitionone}{\underline{Определение 1:} }
\newcommand{\definitiontwo}{\underline{Определение 2:} }
\newcommand{\statement}{\underline{Утверждение:} }
\newcommand{\note}{\underline{Замечание:} }
\newcommand{\sign}{\underline{Обозначения:} }
\newcommand{\statements}{\underline{Утверждения:} }

\newcommand{\Z}{\mathbb{Z}}
\newcommand{\N}{\mathbb{N}}
\newcommand{\Q}{\mathbb{Q}}
\begin{document}
	\section{Домашнее задание 8\\ Шумилкин Андрей, группа 163} 
	\subsection{Задача 1} 
	Переформулируя задачу, нам нужно получить такую строчку, состоящую только из чисел 2, 3, 4 и 5, чтобы можно было между некоторыми из них поставить знаки умножения и полученное выражение равнялось 2007.\\
	По ОТА мы можем разложить 2007 на простые множители, притом единственным образом. Это разложение будет иметь вид $3^2 \cdot 223$.\\
	Тогда искомая строчка будет иметь вид $33223$ и Вовочка расставил знаки следующим образом: $3 \cdot 3 \cdot 223 =2007$. Значит у него две двойки и три тройки, то есть итоговая в четверти у него выйдет "3".\\
	Можно задуматься, а не существует ли другой подходящей строки. Нет, поскольку по-другому  представить число 2007 множителями(не только простыми), состоящими только из 2, 3, 4 и 5 нельзя, так как вообще единственное другое разложение(не считая самого 2007) имеет вид $669 \cdot 3$.
	\subsection{Задача 2} 
	Нужно доказать, что произведение $(m + 1) \cdot (m + 2) \cdot \ldots \cdot (m + n)$ делится на:\\
	\textit{a)} n. \\
	Пусть $(m + 1) \equiv t\ (mod\ n).$ Тогда $(m + 2) \equiv t + 1\ (mod\ n).$
	И в какой-то момент мы найдем такое $(m + k)$, когда $(t + k - 1)$ будет равно $n$( поскольку $k$ принимает значения от 1 до n и значит $t$ в нашем сравнении принимает значения от 0 до n-1 по определению сравнений), а значит $(m + k) \equiv 0\ (mod\ n)$, т.е. $(m + k)$ делящееся на $n$, а значит и произведение в которое оно входит в качестве множителя тоже будет делиться на $n$.\\
	\textit{б)} n!. \\
	Надо доказать, что $\frac{(m+1) \cdot (m+2) \cdot \ldots \cdot(m+n)}{n!}$. Можно заметить, что это равно $С_{m + n}^{n}.$, потому что $С_{m + n}^{n} = \frac{(m+n)!}{n!*(m + n - n)!} = \frac{(m+1) \cdot (m+2) \cdot \ldots \cdot(m+n)!}{n!}$, а, как известно, биноминальные коэффициенты -- это целые числа(можно заметить с помощью треугольника Паскаля, т.е. определения в виде $C_{n}^{k} = C_{n-1}^{k} + C_{n}^{k - 1}$ и того, что в первых двух строках треугольника все числа -- это единицы, т.е. они целочисленны), т.е. $n$ подряд идущих чисел делятся на $n!$.  
	
	
	\subsection{Задача 4}
	\textit{a)} \\
	Пусть число из 69 единиц будет $x$. Тогда заметим, что $9x + 1 = 10^{69}$. \\
	71 -- простое число, тогда по малой теореме Ферма $10^{70} \equiv 1\ (mod\ 71)$. \\
	Значит пусть $t = 10^{69}$. Тогда $10t \equiv 1\ (mod\ 71)$. Легко заметить, что решением сравнения будет $t_0 = 64$ и тогда $10^{69} \equiv 64\ (mod\ 71)$.\\
	Из представления $x$ получаем: $9x \equiv 63\ (mod\ 71)$. $(9, 71) = 1 = 9 \cdot(8) - 71$.
	$8 \cdot 63 \equiv 4 \cdot 126 \equiv 4 \cdot 55 \equiv 2 \cdot 110 \equiv 220 \equiv 7\ (mod\ 71) $.\\
	Ответ: 7. \\
	\textit{б)} \\
	
	\subsection{Задача 5}
	Переформулируя: $4^{n!} \equiv 1\ (mod\ n)$.\\
	Любое нечетное положительное n будет взаимно просто с 4. Тогда по теореме Эйлера $4^{\varphi(n)} \equiv 1\ (mod\ N)$. И тогда $4^{n!} \equiv 1\ (mod\ N)$, поскольку n! делится на $\phi(n)$(т.к. $phi(n) < n$),то есть $4^{n!} -1$ делится на $n$.
	
	\subsection{Задача 6}
	Если число делится на 4 и 5, то оно делится на 20. И если число делится на 20 и 6, то оно делится на 60.\\
	Значит нам нужно найти минимальное число, которое на 60 делится с остатком 1, а на 7 с остатком ноль. Мы можем сделать это просто подбором(61, 121, 181, $\ldots$) и таким числом будет 301. 
	
	\subsection{Задача 9}
	Переформулируя задачу, нужно найти подходящий $x$, при том, что $7^x \equiv 1\ (mod\ 10000)$.Заметим, что 7 взаимно просто с $N$.\\
	По теореме Эйлера $7^{\varphi(10000)} \equiv 1\ (mod\ 10000)$ и значит такая тепень существует.\\
	Теперь осталось вычислить функцию Эйлера от $10000$. Разложим 10000 - 1 = 9999 на протые множители. $9999 = 3^2 \cdot 11 \cdot 101$. Тогда функция Эйлера от 10000 равна: $(3^2 \cdot \frac{2}{3}) \cdot (11 \cdot \frac{10}{11}) \cdot (101 \cdot \frac{100}{101}) = 6000$.
	Имеем $7^{6000} \equiv 1\ (mod\ 10000)$.
\end{document}
