\documentclass[a4paper,12pt]{article}
\usepackage{amsmath}
\usepackage{cmap}					% поиск в PDF
\usepackage{mathtext} 				% русские буквы в формулах
\usepackage[T2A]{fontenc}			% кодировка
\usepackage[utf8]{inputenc}			% кодировка исходного текста
\usepackage[english,russian]{babel}	% локализация и переносы

% Изменим формат \section и \subsection:
\usepackage{titlesec}
\titleformat{\section}
{\vspace{1cm}\centering\LARGE\bfseries}	% Стиль заголовка
{}										% префикс
{0pt}									% Расстояние между префиксом и заголовком
{} 										% Как отображается префикс
\titleformat{\subsection}				% Аналогично для \subsection
{\Large\bfseries}
{}
{0pt}
{}

%% Отступы между абзацами и в начале абзаца 
\setlength{\parindent}{0pt}
\setlength{\parskip}{\medskipamount}

% Перенос знаков в формулах (по Львовскому)
\newcommand*{\hm}[1]{#1\nobreak\discretionary{}
	{\hbox{$\mathsurround=0pt #1$}}{}}

%% Изменяем размер полей
\usepackage[top=1in, bottom=1in, left=1in, right=1in]{geometry}
\begin{document}
	\section{Домашнее задание 22\\ Шумилкин Андрей, группа 163} 
	\subsection{Задача 1}
	Мы можем написать бесконечное количество различных программ, результатом работы которых при входных даных, равных некоторому $x$, будет 2017, потому что мы можем делать в самой программе хоть что(к примеру, комментарии, которые содержат произвольный текст приводят к тому, что возникает бесконечное множество таких программ), поэтому все они будут различны, но выводить константное число. Тогда, если мы занумеруем эти программы, как раз получится, что найдется бесконечно много подходящих $p$.
	
	\subsection{Задача 2} 
	Введем функцию от двух аргументов -- $V(n, x) = nx$. Видно, что она вычислима -- достаточно просто домножить входные данные на константу $n$ и также она всюду определена. \\
	По свойству главных нумераций есть $V(n, x) = U(q(n), x) = nx$, то есть так же есть и необходимая всюду определенная вычислимая функция $q(n)$. \\
	
	И тогда по теореме о неподвижной точке существует $U(q(n), x) = U(n, x) = nx$, поскольку функция $q(n)$ вычислима, что мы заметили из свойства главных нумераций. 
	
	\subsection{Задача 3} 
	По свойству главных нумераций есть $V(n, x) = U(q(n), x)$, то есть так же есть и необходимая всюду определенная вычислимая функция $q(n)$. \\
	
	И тогда по теореме о неподвижной точке существует $U(q(p), x) = U(p, x)$, поскольку функция $q(n)$ вычислима, что мы заметили из свойства главных нумераций. Отсюда и выходит, что найдется такое p, что $V(p, x) = U(p, x)$
	
\end{document}