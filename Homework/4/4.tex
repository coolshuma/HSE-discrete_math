\documentclass[a4paper,12pt]{article}
\usepackage{amsmath}
\usepackage{cmap}					% поиск в PDF
\usepackage{mathtext} 				% русские буквы в формулах
\usepackage[T2A]{fontenc}			% кодировка
\usepackage[utf8]{inputenc}			% кодировка исходного текста
\usepackage[english,russian]{babel}	% локализация и переносы

% Изменим формат \section и \subsection:
\usepackage{titlesec}
\titleformat{\section}
{\vspace{1cm}\centering\LARGE\bfseries}	% Стиль заголовка
{}										% префикс
{0pt}									% Расстояние между префиксом и заголовком
{} 										% Как отображается префикс
\titleformat{\subsection}				% Аналогично для \subsection
{\Large\bfseries}
{}
{0pt}
{}

%% Отступы между абзацами и в начале абзаца 
\setlength{\parindent}{0pt}
\setlength{\parskip}{\medskipamount}

% Перенос знаков в формулах (по Львовскому)
\newcommand*{\hm}[1]{#1\nobreak\discretionary{}
	{\hbox{$\mathsurround=0pt #1$}}{}}

%% Изменяем размер полей
\usepackage[top=1in, bottom=1in, left=1in, right=1in]{geometry}
\begin{document}
	\section{Домашнее задание 4} 
	\subsection{Задача 1}
	Представим задачу в виде графа. Тогда вершины это люди, а ребра -- их знакомство. Тогда количество пар знакомых людей -- это количество ребер в графе. И из того, что у каждых двух человек ровно пять общих знакомых мы можем заключить, что каждое ребро входит в ровно пять циклов длины три(цикл: ребро между двумя рассматриваемыми людьми -- ребро от первого человека до общего знакомого -- ребро от второго человека до общго знаокомого). 
	Тогда 5 * (кол-во ребер) = 3 * (кол-во циклов длины 3). А так как 5 не делится на три, то кол-во ребер(кол-во пар знакомых) должно делиться на 3.
	
	\subsection{Задача 2}
	Обозначим кол-во ребер за $num(E)$.
	
	а) Чтобы граф из n вершин был связным нужно как минимум n-1 ребро (это будет дерево). А полносвязный граф имеет $\frac{n(n-1)}{2}$ ребер. Тогда $n - 1 \le num(E) \le \frac{n(n-1)}{2}$.
	
	б) Граф с минимальным кол-вом ребер, а значит с максимальным кол-вом вершин при каком-то кол-ве ребер -- дерево и в нем ($num(E)$) + 1 вершина. 
	А наименьшее кол-во вершин при наибольшем количестве ребер достигается в полносвязном графе. Значит минимальное кол-во вершин в графе с $num(E)$ ребер будет равно количеству вершин $n_{num(E)}$ в полносвязном графе, для которого $num(E) \le \frac{n_{num(E)}(n_{num(E)}-1)}{2}$, потому что мы рассматриваем только простые графы, а значит в них не может быть кратных ребер и петель. 
	Значит $n_{num(E)} \le num(E) \le (num(E)) + 1$.
	
	\subsection{Задача 4}
	Количество ребер в графе равно половине суммы степеней его вершин. Количество вершин у нас $n$ и степень каждой вершины равна $k$ по определению регулярного графа. Тогда $nk$ должно быть четным, иначе при делении его на два получится нецелое число, а кол-во ребер не может быть нецелым.
	
	Степень вершины в графе из n вершин без кратных ребер и петель(а мы такие и рассматриваем) не может быть по определению больше, чем $n-1$.
	
	\subsection{Задача 5}
	Мы можем взять подграф нашего графа, который будет являться связным и ацикличным. Такой подграф можно найти с помощью алгоритма: начнем с любой вершины. Отметим ее как посещенную. После перейдем во все вершины, доступные из нее и также пометим их как посещенные, а ребра по которым перешли добавим в подграф. Далее перейдем во все вершины, которые доступны из этих вершин и опять добавим ребра по которым преешли в подграф. И так до тех пор, пока останутся непосещенные вершины. Так как граф неориентированный и связный мы точно посетим все вершины. После мы можем просто удалить любой лист этого подграфа и удалить в наешм основном графе вершину ему соответсвующую и основной граф все равно останется связен, поскольку наша вершина -- лист, она имеет только одно ребро в подграфе, притом с вершиной до которой мы смогли дойти из предыдущих вершин. А с другими вершинами она не имеет ребер ,что означает, что мы смогли добраться до тех вершин каким-то другим путем и удаление этой вершины никак не повлияет на его существование.  
	
	\subsection{Задача 6}
	а) Пример прикрепил во вложении к письму.
	
.	
	\end{document}