\documentclass[a4paper,12pt]{article}

%% Работа с русским языком
\usepackage{cmap}					% поиск в PDF
\usepackage{mathtext} 				% русские буквы в формулах
\usepackage[T2A]{fontenc}			% кодировка
\usepackage[utf8]{inputenc}			% кодировка исходного текста
\usepackage[english,russian]{babel}	% локализация и переносы
\usepackage{amsmath, amsfonts, amsthm, mathtools, amssymb, icomma, units}
\usepackage{algorithmicx, algorithm}
\usepackage{algpseudocode}

%% Отступы между абзацами и в начале абзаца 
\setlength{\parindent}{0pt}
\setlength{\parskip}{\medskipamount}

%% Изменяем размер полей
\usepackage[top=0.5in, bottom=0.75in, left=0.825in, right=0.825in]{geometry}

%% Графика
\usepackage[pdftex]{graphicx}
\graphicspath{{images/}}

%% Различные пакеты для работы с математикой
\usepackage{mathtools}				% Тот же amsmath, только с некоторыми поправками

%\usepackage{amssymb}				% Математические символы
\usepackage{amsthm}					% Пакет для написания теорем
\usepackage{amstext}
\usepackage{array}
\usepackage{amsfonts}
\usepackage{icomma}					% "Умная" запятая: $0,2$ --- число, $0, 2$ --- перечисление
\usepackage{bbm}				    % Для красивого (!) \mathbb с  буквами и цифрами
\usepackage{enumitem}               % Для выравнивания itemise (\begin{itemize}[align=left])

% Номера формул
\mathtoolsset{showonlyrefs=true} % Показывать номера только у тех формул, на которые есть \eqref{} в тексте.

% Ссылки
\usepackage[colorlinks=true, urlcolor=blue]{hyperref}

% Шрифты
\usepackage{euscript}	 % Шрифт Евклид
\usepackage{mathrsfs}	 % Красивый матшрифт

% Свои команды\textbf{}
\DeclareMathOperator{\sgn}{\mathop{sgn}}

% Перенос знаков в формулах (по Львовскому)
\newcommand*{\hm}[1]{#1\nobreak\discretionary{}
	{\hbox{$\mathsurround=0pt #1$}}{}}

% Графики
\usepackage{tikz}
\usepackage{pgfplots}
%\pgfplotsset{compat=1.12}

% Изменим формат \section и \subsection:
%\usepackage{titlesec}
%\titleformat{\section}
%{\vspace{1cm}\centering\LARGE\bfseries}	% Стиль заголовка
%{}										% префикс
%{0pt}									% Расстояние между префиксом и заголовком
%{} 										% Как отображается префикс
%\titleformat{\subsection}				% Аналогично для \subsection
%{\Large\bfseries}
%{}
%{0pt}
%{}

% Информация об авторах
\title{Лекции по предмету \\
	\textbf{Линейная алгебра и геометрия}}

\newtheorem*{Def}{Определение}
\newtheorem*{Lemma}{Лемма}
\newtheorem*{Suggestion}{Предложение}
\newtheorem*{Examples}{Пример}
%\newtheorem*{Comment}{Замечание}
\newtheorem*{Consequence}{Следствие}
\newtheorem*{Theorem}{Теорема}
\newtheorem*{Statement}{Утверждение}
\newtheorem*{Task}{Упражнение}
\newtheorem*{Designation}{Обозначение}
\newtheorem*{Generalization}{Обобщение}
\newtheorem*{Thedream}{Предел мечтаний}
\newtheorem*{Properties}{Свойства}


\renewcommand{\Re}{\mathrm{Re\:}}
\renewcommand{\Im}{\mathrm{Im\:}}
\newcommand{\Arg}{\mathrm{Arg\:}}
\renewcommand{\arg}{\mathrm{arg\:}}
\newcommand{\Mat}{\mathrm{Mat}}
\newcommand{\id}{\mathrm{id}}
\newcommand{\isom}{\xrightarrow{\sim}} 
\newcommand{\leftisom}{\xleftarrow{\sim}}
\newcommand{\Hom}{\mathrm{Hom}}
\newcommand{\Ker}{\mathrm{Ker}\:}
\newcommand{\rk}{\mathrm{rk}\:}
\newcommand{\diag}{\mathrm{diag}}
\newcommand{\ort}{\mathrm{ort}}
\newcommand{\pr}{\mathrm{pr}}
\newcommand{\vol}{\mathrm{vol\:}}
\def\limref#1#2{{#1}\negmedspace\mid_{#2}}
\newcommand{\eps}{\varepsilon}

\renewcommand{\epsilon}{\varepsilon}
\renewcommand{\phi}{\varphi}
\newcommand{\e}{\mathbb{e}}
\renewcommand{\l}{\lambda}
\renewcommand{\C}{\mathbb{C}}
\newcommand{\R}{\mathbb{R}}
\newcommand{\E}{\mathbb{E}}

\newcommand{\vvector}[1]{\begin{pmatrix}{#1}_1 \\\vdots\\{#1}_n\end{pmatrix}}
\renewcommand{\vector}[1]{({#1}_1, \ldots, {#1}_n)}

%Теоремы
%11.01.2016
\newtheorem*{standartbase}{Теорема о стандартном базисе}
\newtheorem*{fulllemma}{Лемма}
\newtheorem*{sl1}{Следствие 1}
\newtheorem*{sl2}{Следствие 2}
\newtheorem*{monotonousbase}{Теорема о монотонном базисе}
\newtheorem*{scheme}{Утверждение 1}
\newtheorem*{n2}{Утверждение 2}
\newtheorem*{usp-rais}{Теорема Успенского-Райса}
\newtheorem*{rec}{Свойство рекурсии}
\newtheorem*{point}{Теорема о неподвижной точке}
\newtheorem*{zhegalkin}{Теорема Жегалкина}
\newtheorem*{poste}{Теорема Поста}
\newtheorem*{algo1}{Первое свойство алгоритмов}

%18.01.2016
\newtheorem*{theorem}{Теорема}

\renewcommand{\qedsymbol}{\textbf{Q.E.D.}}
\newcommand{\definition}{\underline{Определение:} }
\newcommand{\definitions}{\underline{Определения:} }
\newcommand{\definitionone}{\underline{Определение 1:} }
\newcommand{\definitiontwo}{\underline{Определение 2:} }
\newcommand{\statement}{\underline{Утверждение:} }
\newcommand{\note}{\underline{Замечание:} }
\newcommand{\sign}{\underline{Обозначения:} }
\newcommand{\statements}{\underline{Утверждения:} }

\newcommand{\Z}{\mathbb{Z}}
\newcommand{\N}{\mathbb{N}}
\newcommand{\Q}{\mathbb{Q}}
\begin{document}
	\section{Домашнее задание 4} 
	\subsection{Задача 1}
	Представим задачу в виде графа. Тогда вершины это люди, а ребра -- их знакомство. Тогда количество пар знакомых людей -- это количество ребер в графе. И из того, что у каждых двух человек ровно пять общих знакомых мы можем заключить, что каждое ребро входит в ровно пять циклов длины три(цикл: ребро между двумя рассматриваемыми людьми -- ребро от первого человека до общего знакомого -- ребро от второго человека до общго знаокомого). 
	Тогда 5 * (кол-во ребер) = 3 * (кол-во циклов длины 3). А так как 5 не делится на три, то кол-во ребер(кол-во пар знакомых) должно делиться на 3.
	
	\subsection{Задача 2}
	Обозначим кол-во ребер за $num(E)$.
	
	а) Чтобы граф из n вершин был связным нужно как минимум n-1 ребро (это будет дерево). А полносвязный граф имеет $\frac{n(n-1)}{2}$ ребер. Тогда $n - 1 \le num(E) \le \frac{n(n-1)}{2}$.
	
	б) Граф с минимальным кол-вом ребер, а значит с максимальным кол-вом вершин при каком-то кол-ве ребер -- дерево и в нем ($num(E)$) + 1 вершина. 
	А наименьшее кол-во вершин при наибольшем количестве ребер достигается в полносвязном графе. Значит минимальное кол-во вершин в графе с $num(E)$ ребер будет равно количеству вершин $n_{num(E)}$ в полносвязном графе, для которого $num(E) \le \frac{n_{num(E)}(n_{num(E)}-1)}{2}$, потому что мы рассматриваем только простые графы, а значит в них не может быть кратных ребер и петель. 
	Значит $n_{num(E)} \le num(E) \le (num(E)) + 1$.
	
	\subsection{Задача 4}
	Количество ребер в графе равно половине суммы степеней его вершин. Количество вершин у нас $n$ и степень каждой вершины равна $k$ по определению регулярного графа. Тогда $nk$ должно быть четным, иначе при делении его на два получится нецелое число, а кол-во ребер не может быть нецелым.
	
	Степень вершины в графе из n вершин без кратных ребер и петель(а мы такие и рассматриваем) не может быть по определению больше, чем $n-1$.
	
	\subsection{Задача 5}
	Мы можем взять подграф нашего графа, который будет являться связным и ацикличным. Такой подграф можно найти с помощью алгоритма: начнем с любой вершины. Отметим ее как посещенную. После перейдем во все вершины, доступные из нее и также пометим их как посещенные, а ребра по которым перешли добавим в подграф. Далее перейдем во все вершины, которые доступны из этих вершин и опять добавим ребра по которым преешли в подграф. И так до тех пор, пока останутся непосещенные вершины. Так как граф неориентированный и связный мы точно посетим все вершины. После мы можем просто удалить любой лист этого подграфа и удалить в наешм основном графе вершину ему соответсвующую и основной граф все равно останется связен, поскольку наша вершина -- лист, она имеет только одно ребро в подграфе, притом с вершиной до которой мы смогли дойти из предыдущих вершин. А с другими вершинами она не имеет ребер ,что означает, что мы смогли добраться до тех вершин каким-то другим путем и удаление этой вершины никак не повлияет на его существование.  
	
	\subsection{Задача 6}
	а) Пример прикрепил во вложении к письму.
	
.	
	\end{document}