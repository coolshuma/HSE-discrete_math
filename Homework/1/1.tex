\documentclass[a4paper,12pt]{article}

%% Работа с русским языком
\usepackage{cmap}					% поиск в PDF
\usepackage{mathtext} 				% русские буквы в формулах
\usepackage[T2A]{fontenc}			% кодировка
\usepackage[utf8]{inputenc}			% кодировка исходного текста
\usepackage[english,russian]{babel}	% локализация и переносы
\usepackage{amsmath, amsfonts, amsthm, mathtools, amssymb, icomma, units}
\usepackage{algorithmicx, algorithm}
\usepackage{algpseudocode}

%% Отступы между абзацами и в начале абзаца 
\setlength{\parindent}{0pt}
\setlength{\parskip}{\medskipamount}

%% Изменяем размер полей
\usepackage[top=0.5in, bottom=0.75in, left=0.825in, right=0.825in]{geometry}

%% Графика
\usepackage[pdftex]{graphicx}
\graphicspath{{images/}}

%% Различные пакеты для работы с математикой
\usepackage{mathtools}				% Тот же amsmath, только с некоторыми поправками

%\usepackage{amssymb}				% Математические символы
\usepackage{amsthm}					% Пакет для написания теорем
\usepackage{amstext}
\usepackage{array}
\usepackage{amsfonts}
\usepackage{icomma}					% "Умная" запятая: $0,2$ --- число, $0, 2$ --- перечисление
\usepackage{bbm}				    % Для красивого (!) \mathbb с  буквами и цифрами
\usepackage{enumitem}               % Для выравнивания itemise (\begin{itemize}[align=left])

% Номера формул
\mathtoolsset{showonlyrefs=true} % Показывать номера только у тех формул, на которые есть \eqref{} в тексте.

% Ссылки
\usepackage[colorlinks=true, urlcolor=blue]{hyperref}

% Шрифты
\usepackage{euscript}	 % Шрифт Евклид
\usepackage{mathrsfs}	 % Красивый матшрифт

% Свои команды\textbf{}
\DeclareMathOperator{\sgn}{\mathop{sgn}}

% Перенос знаков в формулах (по Львовскому)
\newcommand*{\hm}[1]{#1\nobreak\discretionary{}
	{\hbox{$\mathsurround=0pt #1$}}{}}

% Графики
\usepackage{tikz}
\usepackage{pgfplots}
%\pgfplotsset{compat=1.12}

% Изменим формат \section и \subsection:
%\usepackage{titlesec}
%\titleformat{\section}
%{\vspace{1cm}\centering\LARGE\bfseries}	% Стиль заголовка
%{}										% префикс
%{0pt}									% Расстояние между префиксом и заголовком
%{} 										% Как отображается префикс
%\titleformat{\subsection}				% Аналогично для \subsection
%{\Large\bfseries}
%{}
%{0pt}
%{}

% Информация об авторах
\title{Лекции по предмету \\
	\textbf{Линейная алгебра и геометрия}}

\newtheorem*{Def}{Определение}
\newtheorem*{Lemma}{Лемма}
\newtheorem*{Suggestion}{Предложение}
\newtheorem*{Examples}{Пример}
%\newtheorem*{Comment}{Замечание}
\newtheorem*{Consequence}{Следствие}
\newtheorem*{Theorem}{Теорема}
\newtheorem*{Statement}{Утверждение}
\newtheorem*{Task}{Упражнение}
\newtheorem*{Designation}{Обозначение}
\newtheorem*{Generalization}{Обобщение}
\newtheorem*{Thedream}{Предел мечтаний}
\newtheorem*{Properties}{Свойства}


\renewcommand{\Re}{\mathrm{Re\:}}
\renewcommand{\Im}{\mathrm{Im\:}}
\newcommand{\Arg}{\mathrm{Arg\:}}
\renewcommand{\arg}{\mathrm{arg\:}}
\newcommand{\Mat}{\mathrm{Mat}}
\newcommand{\id}{\mathrm{id}}
\newcommand{\isom}{\xrightarrow{\sim}} 
\newcommand{\leftisom}{\xleftarrow{\sim}}
\newcommand{\Hom}{\mathrm{Hom}}
\newcommand{\Ker}{\mathrm{Ker}\:}
\newcommand{\rk}{\mathrm{rk}\:}
\newcommand{\diag}{\mathrm{diag}}
\newcommand{\ort}{\mathrm{ort}}
\newcommand{\pr}{\mathrm{pr}}
\newcommand{\vol}{\mathrm{vol\:}}
\def\limref#1#2{{#1}\negmedspace\mid_{#2}}
\newcommand{\eps}{\varepsilon}

\renewcommand{\epsilon}{\varepsilon}
\renewcommand{\phi}{\varphi}
\newcommand{\e}{\mathbb{e}}
\renewcommand{\l}{\lambda}
\renewcommand{\C}{\mathbb{C}}
\newcommand{\R}{\mathbb{R}}
\newcommand{\E}{\mathbb{E}}

\newcommand{\vvector}[1]{\begin{pmatrix}{#1}_1 \\\vdots\\{#1}_n\end{pmatrix}}
\renewcommand{\vector}[1]{({#1}_1, \ldots, {#1}_n)}

%Теоремы
%11.01.2016
\newtheorem*{standartbase}{Теорема о стандартном базисе}
\newtheorem*{fulllemma}{Лемма}
\newtheorem*{sl1}{Следствие 1}
\newtheorem*{sl2}{Следствие 2}
\newtheorem*{monotonousbase}{Теорема о монотонном базисе}
\newtheorem*{scheme}{Утверждение 1}
\newtheorem*{n2}{Утверждение 2}
\newtheorem*{usp-rais}{Теорема Успенского-Райса}
\newtheorem*{rec}{Свойство рекурсии}
\newtheorem*{point}{Теорема о неподвижной точке}
\newtheorem*{zhegalkin}{Теорема Жегалкина}
\newtheorem*{poste}{Теорема Поста}
\newtheorem*{algo1}{Первое свойство алгоритмов}

%18.01.2016
\newtheorem*{theorem}{Теорема}

\renewcommand{\qedsymbol}{\textbf{Q.E.D.}}
\newcommand{\definition}{\underline{Определение:} }
\newcommand{\definitions}{\underline{Определения:} }
\newcommand{\definitionone}{\underline{Определение 1:} }
\newcommand{\definitiontwo}{\underline{Определение 2:} }
\newcommand{\statement}{\underline{Утверждение:} }
\newcommand{\note}{\underline{Замечание:} }
\newcommand{\sign}{\underline{Обозначения:} }
\newcommand{\statements}{\underline{Утверждения:} }

\newcommand{\Z}{\mathbb{Z}}
\newcommand{\N}{\mathbb{N}}
\newcommand{\Q}{\mathbb{Q}}
\begin{document}
	\section{Домашнее задание 1}
	\subsection{Задача 1}
	$$
	1 \cdot (n - 1) + 2 \cdot (n -2) + \dots + (n - 1) \cdot 1 =  \frac{(n - 1)  \cdot  n \cdot (n + 1)}{6}.
	$$
	
	{\bfРешение.}
	
	\textit{База}.
	
	$n = 1.$
	
	$ 1 \cdot 0 = \frac{(1 - 1)  \cdot  1 \cdot (1 + 1)}{6}. $
	
	$0 = 0$
	
	\textit{Шаг}.
	
	Мы можем представить левую часть уравнения как
	 \[
		\sum_{i = 1}^{n - 1} i \cdot (n - i)= \sum_{i = 1}^{n - 1}(i \cdot n - i^2) = \sum_{i = 1}^{n - 1} (i \cdot n) - \sum_{i = 1}^{n - 1} i^2.
	\]
	Также заметим, используя формулу суммы арифмитической прогрессии, чтo
	\[
		\sum_{i = 1}^{n - 1} (i \cdot n) = \left(\frac{1 + (n-1)}{2} \cdot (n - 1)\right) \cdot n = \frac{n^3 - n^2}{2}.
	\]
	
	и 
	
	\[
		\sum_{i = 1}^{n} (i \cdot (n + 1)) = \sum_{i = 1}^{n} (i \cdot n) + \sum_{i = 1}^{n} (i) = \frac{n^3 + n^2}{2} + \frac{n^2 + n}{2} = \frac{n^3 + 2n^2 + n}{2}.
	\]
	Тогда докажем, что 
	\[     
		\left(\sum_{i = 1}^{n} (i \cdot n) - \sum_{i = 1}^{n} i^2\right) - \left(\sum_{i = 1}^{n - 1} (i \cdot n) - \sum_{i = 1}^{n - 1} i^2\right) = \frac{n \cdot  (n + 1) \cdot (n + 2)}{6} - \frac{(n - 1)  \cdot  n \cdot (n + 1)}{6}.
	\]
	\[     
		\frac{n^3 + 2n^2 + n}{2} - \frac{n^3 - n^2}{2} - n^2 = \frac{(n^3 + 2n^2 + n^2 + 2n) - (n^3 + n^2 - n^2 - n)}{6}.
	\]
	\[     
		\frac{3n^2 + n}{2} - n^2 = \frac{3n^2 + 3n}{6}.
	\]
	\[     
		\frac{n^2 + n}{2} = \frac{n^2 + n}{2}.
	\]
	\\*
	
	\subsection{Задача 2}
	\[
		\cos(x) + \cos(2x) + \dots + \cos(nx) = \frac{\sin(n+\frac{1}{2})x}{2sin\frac{x}{2}} - \frac{1}{2}.
	\]
	
	\textit{База}.
	
	$ n = 1$
	\[
		\cos(x) = \frac{\sin\frac{3}{2}x}{2sin\frac{x}{2}} - \frac{1}{2}.
	\]
	\[
		\cos(x) = \frac{\sin\frac{3}{2}x}{2sin\frac{x}{2}} - \frac{\sin\frac{1}{2}x}{2sin\frac{x}{2}}.
	\]
	\[
		\cos(x) = \frac{2\sin\frac{x}{2} \cdot \cos(x)} {2\sin\frac{x}{2}}.
	\]
	\[
		\cos(x) = \cos(x).
	\]
	
	\textit{Шаг}.
	Докажем, что
	\[
		\cos(x) + \cos(2x) + \dots + \cos(nx) + \cos x(n+1) = \frac{\sin(n+1+\frac{1}{2})x}{2sin\frac{x}{2}} - \frac{1}{2}.
	\]
	\[
		\frac{\sin(n+1+\frac{1}{2})x}{2sin\frac{x}{2}} - \frac{1}{2} = \frac{\sin(n+\frac{1}{2})x}{2sin\frac{x}{2}} - \frac{1}{2} + cos x(n + 1).
	\]
	\[
		\frac{\sin(n+1+\frac{1}{2})x}{2sin\frac{x}{2}} - \frac{1}{2} = \frac{\sin(n+\frac{1}{2})x - \sin \frac{x}{2} + 2\sin \frac{x}{2}\cos x(n+1)}{2\sin \frac{x}{2}}.
	\]
	\[
		\frac{\sin(n+1+\frac{1}{2})x}{2sin\frac{x}{2}} - \frac{1}{2} = \frac{\sin(n+\frac{1}{2})x - \sin \frac{x}{2} + \sin(\frac{x}{2} + nx + x) + \sin(\frac{x}{2} - nx - x)}{2\sin \frac{x}{2}}.
	\]
	\[
		\frac{\sin(\frac{x}{2} + nx + x) - \sin \frac{x}{2}}{2sin\frac{x}{2}} = \frac{\sin(n+\frac{1}{2})x - \sin \frac{x}{2} + \sin(\frac{x}{2} + nx + x) + \sin(\frac{x}{2} - nx - x)}{2\sin \frac{x}{2}}.
	\]
	\[
		\frac{\sin(n+\frac{1}{2})x + \sin(\frac{x}{2} - nx - x)}{2\sin \frac{x}{2}} = 0.
	\]
	\[
		\frac{\sin(n+\frac{1}{2})x - \sin(n+\frac{1}{2})x}{2\sin \frac{x}{2}} = 0.
	\]
	\[
		\frac{0}{2\sin \frac{x}{2}} = 0.
	\]
	\[
		0 = 0.
	\]
	
\subsection{Задача 4}
	\[
		\frac{1}{n + 1} + \frac{1}{n + 2} + \dots + \frac{1}{2n} > 13/24,\  для\  всех\  n > 1.
	\]

	\textit{База}.
	
	$ n = 2$
	\[
		\frac{1}{3} + \frac{1}{4} = 14/24 > 13/24.
	\]
	\textit{Шаг}.
	
	Докажем, что послдеовательность будет возрастающей, а значит, что $f(n + 1) > f(n)$ и тогда из базы => $f(n) \ge 14/24$ при $n \ge 2$ и значит $f(n) > 13/24$ при $n > 1$.
	
	$\left(\frac{1}{n + 2} + \frac{1}{n + 3} + \dots + \frac{1}{2(n+1)}\right) - \left(\frac{1}{n + 1} + \frac{1}{n + 2} + \dots + \frac{1}{2n}\right) = \frac{1}{2n+1} + \frac{1}{2(n+1)} - \frac{1}{n+1} =$
	
	$\frac{2(n+1)+2n+1-2(2n+1)}{2(n+1)(2n+1)} = \frac{1}{2(n+1)(2n+1)} > 0$.
	
	Последовательность возрастающая, так как разность между ее текущим и предыдущим членом положительна.
	
	\subsection{Задача 7}
	\[
	\frac{1}{n + 1} + \frac{1}{n + 2} + \dots + \frac{1}{2n} > 13/24,\  для\  всех\  n > 1.
	\]
	
	\textit{База}.

	$ n = 1$.
	
	$ 4/2^1 = 2$.
	
	\textit{Шаг}.
	
	Заметим, что текущее число на шаге n будет делиться на $2^{n+1}$ либо нацело, либо с остатком равным единице. 
	
	Обозначим число на шаге n за t.
	Будем просто приписывать слева 3 или 4. Тогда мы можем представить следующее число как $3*10^n + t = 3*(2^n*5^n) + t$ в случае, если мы приписали тройку и $4*10^n + t = 2^{n+2}*5^n + t$. 
	Легко заметить, что $2^{n+2}*5^n$ делится на $2^{n+1}$ нацело, а $3*(2^n*5^n)$ с остатком один.
	Тогда мы можем в случае, если t делится на $2^{n+1}$ нацело добавлять в начало числа четверку, а если же с остатком 1 -- тройку.
	
\end{document}