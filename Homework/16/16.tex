\documentclass[a4paper,12pt]{article}
\usepackage{amsmath}
\usepackage{cmap}					% поиск в PDF
\usepackage{mathtext} 				% русские буквы в формулах
\usepackage[T2A]{fontenc}			% кодировка
\usepackage[utf8]{inputenc}			% кодировка исходного текста
\usepackage[english,russian]{babel}	% локализация и переносы

% Изменим формат \section и \subsection:
\usepackage{titlesec}
\titleformat{\section}
{\vspace{1cm}\centering\LARGE\bfseries}	% Стиль заголовка
{}										% префикс
{0pt}									% Расстояние между префиксом и заголовком
{} 										% Как отображается префикс
\titleformat{\subsection}				% Аналогично для \subsection
{\Large\bfseries}
{}
{0pt}
{}

%% Отступы между абзацами и в начале абзаца 
\setlength{\parindent}{0pt}
\setlength{\parskip}{\medskipamount}

% Перенос знаков в формулах (по Львовскому)
\newcommand*{\hm}[1]{#1\nobreak\discretionary{}
	{\hbox{$\mathsurround=0pt #1$}}{}}

%% Изменяем размер полей
\usepackage[top=1in, bottom=1in, left=1in, right=1in]{geometry}
\begin{document}
	\section{Домашнее задание 16\\ Шумилкин Андрей, группа 163} 
	\subsection{Задача 1}
	Заметим, что множество вещественных положительных чисел и ноль континуально , поскольку его подмножеством является интервал [0,1], который имеет мощность континуум. \\
	Мы можем каждый круг охарактеризовать тройкой чисел $(x, y, r)$, то есть его координатами центра и радиусом, при этом видно, что для разных кругов эта характеристика будет разной.\\
	Мы можем строить круг в любой точке плоскости и с любым радиусом, значит все три числа примут всевозможные значения из множества положительных вещественных чисел и нуля.\\
	И, как нам известно, $\mathbb{R}^3$ равномощно $\mathbb{R}$, откуда и следует, что множество всех кругов на плоскости континуально. 
	
	\subsection{Задача 2}
	Нет, неверно, поскольку мы можем выбрать какую-либо точку и построить континуум окружностей с центром в ней и которые имеют радиусы, к примеру, которые равны всем точкам из отрезка [0,1]. Множество таких окружностей будет континуально, поскольку множество всех точек отрезка [0,1] континуально, но множество их центров будет иметь мощность один, так как мы по построению сделали все центры в одной точке.
	
	\subsection{Задача 3}
	Да, существует. \\
	Мы знаем, что $\mathbb{R}^2$ равномощно $\mathbb{R}$, а в $\mathbb{R}^2$ мы можем найти такое семейство -- это множество параллельных оси x прямых, которые характеризуются $y = c$. Каждая прямая континуальна и их множество тоже континуально, так как $c$ может быть любым из $\mathbb{R}$. И, так как $\mathbb{R}^2$ равномощно $\mathbb{R}$, существует инъекция из $\mathbb{R}^2$ в $\mathbb{R}$ мы каждой прямой можем сопоставить некоторое число и они не будут пересекаться, так как и сами прямые не пересекаются. 
	
	\subsection{Задача 4}
	Оно будет иметь мощность точно не больше мощности континуума, потому что мы можем любую последовательность перевести из двоичной системы счисления в десятичную и получить некоторое число, которое точно принадлежит $R$, а $R$ имеет мощность континуума. \\
	Воспользуемся теоремой Кантора-Берштейна. \\
	Инъекция из множества двоичных последовательностей без трех подряд идущих единиц в обычные двоичные последовательности понятна  -- мы можем просто переводить в те же числа. \\
	Теперь построим инъекцию  из множества обычных двоичных послдеовательностей в последовательности без трех единиц подряд. 
	Заметим, что единиц подряд тогда  может быть одна или две. Тогда будем "переводить" наше число следующим образом: если на текущей позиции "0", то пишем одну единицу и за ней ноль, а если "1", то две единицы и за ними ноль. \\
	Тогда в итоге получим последовательность без трех единиц подряд, соответствующую обычной двоичной последовательности, при том для разных они будут разные. \\
	По теореме Кантора-Берштейна получаем, что наше множество равномощно множеству обычных двоичных последовательностей, а оно континуально, откуда следует, что наше множество так же имеет мощность континуум. 
	
\end{document}