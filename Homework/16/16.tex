\documentclass[a4paper,12pt]{article}

%% Работа с русским языком
\usepackage{cmap}					% поиск в PDF
\usepackage{mathtext} 				% русские буквы в формулах
\usepackage[T2A]{fontenc}			% кодировка
\usepackage[utf8]{inputenc}			% кодировка исходного текста
\usepackage[english,russian]{babel}	% локализация и переносы
\usepackage{amsmath, amsfonts, amsthm, mathtools, amssymb, icomma, units}
\usepackage{algorithmicx, algorithm}
\usepackage{algpseudocode}

%% Отступы между абзацами и в начале абзаца 
\setlength{\parindent}{0pt}
\setlength{\parskip}{\medskipamount}

%% Изменяем размер полей
\usepackage[top=0.5in, bottom=0.75in, left=0.825in, right=0.825in]{geometry}

%% Графика
\usepackage[pdftex]{graphicx}
\graphicspath{{images/}}

%% Различные пакеты для работы с математикой
\usepackage{mathtools}				% Тот же amsmath, только с некоторыми поправками

%\usepackage{amssymb}				% Математические символы
\usepackage{amsthm}					% Пакет для написания теорем
\usepackage{amstext}
\usepackage{array}
\usepackage{amsfonts}
\usepackage{icomma}					% "Умная" запятая: $0,2$ --- число, $0, 2$ --- перечисление
\usepackage{bbm}				    % Для красивого (!) \mathbb с  буквами и цифрами
\usepackage{enumitem}               % Для выравнивания itemise (\begin{itemize}[align=left])

% Номера формул
\mathtoolsset{showonlyrefs=true} % Показывать номера только у тех формул, на которые есть \eqref{} в тексте.

% Ссылки
\usepackage[colorlinks=true, urlcolor=blue]{hyperref}

% Шрифты
\usepackage{euscript}	 % Шрифт Евклид
\usepackage{mathrsfs}	 % Красивый матшрифт

% Свои команды\textbf{}
\DeclareMathOperator{\sgn}{\mathop{sgn}}

% Перенос знаков в формулах (по Львовскому)
\newcommand*{\hm}[1]{#1\nobreak\discretionary{}
	{\hbox{$\mathsurround=0pt #1$}}{}}

% Графики
\usepackage{tikz}
\usepackage{pgfplots}
%\pgfplotsset{compat=1.12}

% Изменим формат \section и \subsection:
%\usepackage{titlesec}
%\titleformat{\section}
%{\vspace{1cm}\centering\LARGE\bfseries}	% Стиль заголовка
%{}										% префикс
%{0pt}									% Расстояние между префиксом и заголовком
%{} 										% Как отображается префикс
%\titleformat{\subsection}				% Аналогично для \subsection
%{\Large\bfseries}
%{}
%{0pt}
%{}

% Информация об авторах
\title{Лекции по предмету \\
	\textbf{Линейная алгебра и геометрия}}

\newtheorem*{Def}{Определение}
\newtheorem*{Lemma}{Лемма}
\newtheorem*{Suggestion}{Предложение}
\newtheorem*{Examples}{Пример}
%\newtheorem*{Comment}{Замечание}
\newtheorem*{Consequence}{Следствие}
\newtheorem*{Theorem}{Теорема}
\newtheorem*{Statement}{Утверждение}
\newtheorem*{Task}{Упражнение}
\newtheorem*{Designation}{Обозначение}
\newtheorem*{Generalization}{Обобщение}
\newtheorem*{Thedream}{Предел мечтаний}
\newtheorem*{Properties}{Свойства}


\renewcommand{\Re}{\mathrm{Re\:}}
\renewcommand{\Im}{\mathrm{Im\:}}
\newcommand{\Arg}{\mathrm{Arg\:}}
\renewcommand{\arg}{\mathrm{arg\:}}
\newcommand{\Mat}{\mathrm{Mat}}
\newcommand{\id}{\mathrm{id}}
\newcommand{\isom}{\xrightarrow{\sim}} 
\newcommand{\leftisom}{\xleftarrow{\sim}}
\newcommand{\Hom}{\mathrm{Hom}}
\newcommand{\Ker}{\mathrm{Ker}\:}
\newcommand{\rk}{\mathrm{rk}\:}
\newcommand{\diag}{\mathrm{diag}}
\newcommand{\ort}{\mathrm{ort}}
\newcommand{\pr}{\mathrm{pr}}
\newcommand{\vol}{\mathrm{vol\:}}
\def\limref#1#2{{#1}\negmedspace\mid_{#2}}
\newcommand{\eps}{\varepsilon}

\renewcommand{\epsilon}{\varepsilon}
\renewcommand{\phi}{\varphi}
\newcommand{\e}{\mathbb{e}}
\renewcommand{\l}{\lambda}
\renewcommand{\C}{\mathbb{C}}
\newcommand{\R}{\mathbb{R}}
\newcommand{\E}{\mathbb{E}}

\newcommand{\vvector}[1]{\begin{pmatrix}{#1}_1 \\\vdots\\{#1}_n\end{pmatrix}}
\renewcommand{\vector}[1]{({#1}_1, \ldots, {#1}_n)}

%Теоремы
%11.01.2016
\newtheorem*{standartbase}{Теорема о стандартном базисе}
\newtheorem*{fulllemma}{Лемма}
\newtheorem*{sl1}{Следствие 1}
\newtheorem*{sl2}{Следствие 2}
\newtheorem*{monotonousbase}{Теорема о монотонном базисе}
\newtheorem*{scheme}{Утверждение 1}
\newtheorem*{n2}{Утверждение 2}
\newtheorem*{usp-rais}{Теорема Успенского-Райса}
\newtheorem*{rec}{Свойство рекурсии}
\newtheorem*{point}{Теорема о неподвижной точке}
\newtheorem*{zhegalkin}{Теорема Жегалкина}
\newtheorem*{poste}{Теорема Поста}
\newtheorem*{algo1}{Первое свойство алгоритмов}

%18.01.2016
\newtheorem*{theorem}{Теорема}

\renewcommand{\qedsymbol}{\textbf{Q.E.D.}}
\newcommand{\definition}{\underline{Определение:} }
\newcommand{\definitions}{\underline{Определения:} }
\newcommand{\definitionone}{\underline{Определение 1:} }
\newcommand{\definitiontwo}{\underline{Определение 2:} }
\newcommand{\statement}{\underline{Утверждение:} }
\newcommand{\note}{\underline{Замечание:} }
\newcommand{\sign}{\underline{Обозначения:} }
\newcommand{\statements}{\underline{Утверждения:} }

\newcommand{\Z}{\mathbb{Z}}
\newcommand{\N}{\mathbb{N}}
\newcommand{\Q}{\mathbb{Q}}
\begin{document}
	\section{Домашнее задание 16\\ Шумилкин Андрей, группа 163} 
	\subsection{Задача 1}
	Заметим, что множество вещественных положительных чисел и ноль континуально , поскольку его подмножеством является интервал [0,1], который имеет мощность континуум. \\
	Мы можем каждый круг охарактеризовать тройкой чисел $(x, y, r)$, то есть его координатами центра и радиусом, при этом видно, что для разных кругов эта характеристика будет разной.\\
	Мы можем строить круг в любой точке плоскости и с любым радиусом, значит все три числа примут всевозможные значения из множества положительных вещественных чисел и нуля.\\
	И, как нам известно, $\mathbb{R}^3$ равномощно $\mathbb{R}$, откуда и следует, что множество всех кругов на плоскости континуально. 
	
	\subsection{Задача 2}
	Нет, неверно, поскольку мы можем выбрать какую-либо точку и построить континуум окружностей с центром в ней и которые имеют радиусы, к примеру, которые равны всем точкам из отрезка [0,1]. Множество таких окружностей будет континуально, поскольку множество всех точек отрезка [0,1] континуально, но множество их центров будет иметь мощность один, так как мы по построению сделали все центры в одной точке.
	
	\subsection{Задача 3}
	Да, существует. \\
	Мы знаем, что $\mathbb{R}^2$ равномощно $\mathbb{R}$, а в $\mathbb{R}^2$ мы можем найти такое семейство -- это множество параллельных оси x прямых, которые характеризуются $y = c$. Каждая прямая континуальна и их множество тоже континуально, так как $c$ может быть любым из $\mathbb{R}$. И, так как $\mathbb{R}^2$ равномощно $\mathbb{R}$, существует инъекция из $\mathbb{R}^2$ в $\mathbb{R}$ мы каждой прямой можем сопоставить некоторое число и они не будут пересекаться, так как и сами прямые не пересекаются. 
	
	\subsection{Задача 4}
	Оно будет иметь мощность точно не больше мощности континуума, потому что мы можем любую последовательность перевести из двоичной системы счисления в десятичную и получить некоторое число, которое точно принадлежит $R$, а $R$ имеет мощность континуума. \\
	Воспользуемся теоремой Кантора-Берштейна. \\
	Инъекция из множества двоичных последовательностей без трех подряд идущих единиц в обычные двоичные последовательности понятна  -- мы можем просто переводить в те же числа. \\
	Теперь построим инъекцию  из множества обычных двоичных послдеовательностей в последовательности без трех единиц подряд. 
	Заметим, что единиц подряд тогда  может быть одна или две. Тогда будем "переводить" наше число следующим образом: если на текущей позиции "0", то пишем одну единицу и за ней ноль, а если "1", то две единицы и за ними ноль. \\
	Тогда в итоге получим последовательность без трех единиц подряд, соответствующую обычной двоичной последовательности, при том для разных они будут разные. \\
	По теореме Кантора-Берштейна получаем, что наше множество равномощно множеству обычных двоичных последовательностей, а оно континуально, откуда следует, что наше множество так же имеет мощность континуум. 
	
\end{document}