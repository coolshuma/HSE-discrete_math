\documentclass[a4paper,12pt]{article}
\usepackage{amsmath}
\usepackage{cmap}					% поиск в PDF
\usepackage{mathtext} 				% русские буквы в формулах
\usepackage[T2A]{fontenc}			% кодировка
\usepackage[utf8]{inputenc}			% кодировка исходного текста
\usepackage[english,russian]{babel}	% локализация и переносы

% Изменим формат \section и \subsection:
\usepackage{titlesec}
\titleformat{\section}
{\vspace{1cm}\centering\LARGE\bfseries}	% Стиль заголовка
{}										% префикс
{0pt}									% Расстояние между префиксом и заголовком
{} 										% Как отображается префикс
\titleformat{\subsection}				% Аналогично для \subsection
{\Large\bfseries}
{}
{0pt}
{}

%% Отступы между абзацами и в начале абзаца 
\setlength{\parindent}{0pt}
\setlength{\parskip}{\medskipamount}

% Перенос знаков в формулах (по Львовскому)
\newcommand*{\hm}[1]{#1\nobreak\discretionary{}
	{\hbox{$\mathsurround=0pt #1$}}{}}

%% Изменяем размер полей
\usepackage[top=1in, bottom=1in, left=1in, right=1in]{geometry}
\begin{document}
	\section{Домашнее задание 14\\ Шумилкин Андрей, группа 163} 
	\subsection{Задача 1}
	Раз на выирыши уходит $40\%$, то есть это <<то, что мы будем получать в среднем, если будем повторять эксперимент много раз>> => $E[x] = 40 $. Тогда по неравенству Маркова $Pr[x \ge 5000] \le \frac{40}{5000} \le 0,008$, то есть вероятность выиграть 5000 и больше, меньше или равна 0,008 что меньше 0,01.
	
	\subsection{Задача 2}
	В среднем люди жили 26 лет, то есть можем говорить, что математическое ожидание равно 26.
	Рассмотрим два крайних случая, когда люди, которые жили мало: когда они проживали 0 лет и когда проживали восемь. По условию прожить меньше 9 лет равна 1/2. Тогда прожить больше так же будет 1/2. \\
	Отсюда для 0 получаем: $1/2 \cdot 0 + 1/2 \cdot t_1 = 26$, где $t_1$ -- средний возраст проживших более 8 лет, когда жившие мало жили 0 лет в среднем. Тогда $t_1 = 52$. \\
	И для 8 получаем: $1/2 \cdot 8 + 1/2 \cdot t_2 = 26$, где $t_2$ -- средний возраст проживших более 8 лет, когда жившие мало жили 8 лет в среднем. Тогда $t_2 = 44$.\\
	И раз мы рассмотрели крайние значения, то мы получили границы искомого интервала: 44 и 52. 
	
	\subsection{Задача 3}
	Как известно, мат. ожидание честной кости равно 3,5, тогда у первого игрока мат. ожидание, то есть и средний выигрыш будет равен 12,25. А у второго средний выигрыш равен $\frac{1^2 + 2^2 + 3^2 + 4^2 + 5^2 + 6^2}{6} = \frac{91}{6}$, что примерно равно 15,1 и больше, чем 12,25.
	Значит средний выигрыш второго игрока больше, чем у первого.
	
	\subsection{Задача 4}
	Пусть у нас будет некоторая индикаторная величина, которая обозначает начинается ли с данного элемента подстрока ab. Заметим, что ее мат. ожидание равно 1/4, потому что возможных строк всего 4 -- $\{1, 2, 3, 4\}$. \\
	Тогда мы можем представить нашу функцию как сумму индикаторных величин, при этом заметим, что позиций с которых может начинаться строка длины 2 всего 19. \\
	И мы знаем, что математическое ожидание мы тогда можем представить как сумму математических ожиданий индикаторных величин, которых всего 19 и значит искомое мат. ожидание равно 19/4.
	
	\subsection{Задача 5}
	Пусть у нас есть некоторая индикаторная велиина, которая обозначает, что завтрак попробован. Вероятность того, что завтрак попробован будет $1 - \frac{9^{15}}{10^{15}}$ и ее математическое ожидание тому же. \\
	Тогда мы можем представить среднее кол-во попробованных завтраков как сумму математических ожиданий индикаторных величин. Их всего 10 и тогда искомое математическое ожидание равно $\frac{9^{15}}{10^{14}}$.
		
\end{document}