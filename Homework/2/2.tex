\documentclass[a4paper,12pt]{article}
\usepackage{amsmath}
\usepackage{cmap}					% поиск в PDF
\usepackage{mathtext} 				% русские буквы в формулах
\usepackage[T2A]{fontenc}			% кодировка
\usepackage[utf8]{inputenc}			% кодировка исходного текста
\usepackage[english,russian]{babel}	% локализация и переносы

% Изменим формат \section и \subsection:
\usepackage{titlesec}
\titleformat{\section}
{\vspace{1cm}\centering\LARGE\bfseries}	% Стиль заголовка
{}										% префикс
{0pt}									% Расстояние между префиксом и заголовком
{} 										% Как отображается префикс
\titleformat{\subsection}				% Аналогично для \subsection
{\Large\bfseries}
{}
{0pt}
{}

%% Отступы между абзацами и в начале абзаца 
\setlength{\parindent}{0pt}
\setlength{\parskip}{\medskipamount}

% Перенос знаков в формулах (по Львовскому)
\newcommand*{\hm}[1]{#1\nobreak\discretionary{}
	{\hbox{$\mathsurround=0pt #1$}}{}}

%% Изменяем размер полей
\usepackage[top=1in, bottom=1in, left=1in, right=1in]{geometry}
\begin{document}
	\section{Домашнее задание 2}
	Во всей работе $\left(_{k}^{n}\right)$ -- это количество сочетаний по k из n элементов. 
	\subsection{Задача 1}
	Найти сумму всех пятизначных чисел, составленных из нечетных цифр.
	
	{\bfРешение.}
	
	Заметим, что всего у нас пять нечетных цифр: $1, 3, 5, 7, 9$.	
	Их сумма равна 25.
	Будем рассматривать поразрядно пятизначные числа. Каждое число в каждом разряде может войти $ 5*5*5*5$ раз, потому что у когда мы "установили" одно число у нас остается четыре разряда в которх может быть любое другое нечетное число. Отсюда мы можем заметить, что и сумма в одном разряде может войти $5^4$ раз. 
	Также заметим, что сумма в первом разряде будет $25 * 10^1$, $25 * 10^2$ во втором $\dots$ и $25 * 10^5$ в пятом.
	
	Тогда, если нумеровать разряды справа налево и обозначить текущий номер разряда за $i$ для каждого разряда формула будет: $25*10^i*5^4$.
	То есть итоговая формула: 
	\[
		\sum_{i = 1}^{5} 25*5^4*10^i = 173609375.
	\]
	\subsection{Задача 2}
	Для определенности будем говорить о \textit{натуральных} числах.
	
	\textit{Пункт а)}	
	
	Посчитаем все числа до миллиона в которых нет единицы. Мы можем сделать это так:
	Будем считать, что у нас есть шесть позиций на которых могут стоять числа от нуля до девяти, исключая единицу, то есть их девять штук. Тогда, к примеру, когда на первых пяти позициях будут стоять нули мы учтем все однозначные числа, потом когда на первых четырех позициях будут стоять нули мы учтем двухзначные и так далее до того, когда нулей не будет и мы учтем все шестизначные числа без единиц.
	$9^6 = 531441$.
	Теперь отнимем от $10^6$ вычисленное кол-во чисел без единиц, очевидно, получим кол-во чисел с единицами.
	$10^6 - 9^6 = 468559$.
	Надо отметить, что здесь учтен ноль, но не учтено $10^6$. То есть итоговые значения будут: 531440 чисел без единиц и 468560 чисел с единицами. Но на общий ответ пункта а это, как видим, не влияет. 
	Ответ: в первом миллионе больше чисел без единиц.
	
	\textit{Пункт б)}	
	
	Аналогично. Чисел без единиц $9^7 - 1$(не учитываем ноль). $ = 4782968$.
	С единицами $10^7 - 9^7 + 1$(учитываем $10^7$) $= 5217032$. 
	Ответ: в певых десяти миллионах больше чисел с единицами. 
	
	\subsection{Задача 3}
	Найдем количество десятизначных чисел в которых все цифры различны. Будем рассматривать число с левого края. На первое место мы можем поставить любую из 10 цифр, кроме нуля(так как число не может начинаться с нуля), то есть 9. На второе место, чтобы сохранялось свойство различности всех цифр, мы можем поставить любую цифру, включая теперь уже ноль, но исключая то, которое было на первом месте = 9. На третье место мы можем поставить любую цифру за исключением двух, что стояли на предыдущих местах = 8. И так же рассуждаем далее, до 10-го места, на которое мы можем поставить  уже лишь одну цифру. В итоге имееи $9*9*8*7*6*5*4*3*2*1 = 3265920 = 9 * 9!$  десятизначных чисел в которых все цифры различны.
	Тогда мы можем получить кол-во десятизначных чисел у которых хотя бы две цифры совпадают просто отняв полученное значение от общего кол-ва десятизначных цифр. $(10^10 - 10^9) - 3265920 = 8996734080 = (9 * 10^9) - 9 * 9!$.
	А теперь просто посчитаем вероятность по известной формуле (кол-во благоприятных исходов)/(кол-во всех исходов).
	$8996734080 / (10^10 - 10^9) = \frac{9*(10^9 - 9!)}{9 * 10^9} = \frac{10^9 - 9!}{10^9}$.
	
	Ответ: $\frac{10^9 - 9!}{10^9}$.
	
	\subsection{Задача 4}
	а)Возьмем некого человека. Способов выбрать ему пару у нас будет $2n - 1$. Первая пара готова. После чего возьмем другого человека. Способов выбрать ему пару уже будет $2n - 3$. И т.д. вплоть до предпоследнего человека, способов выбрать которому пару будет один, так как останется лишь один человек.
	Можем записать это как $\prod_{i = 0}^{n - 1} (2n - 2 * i + 1)$.
	б) Из предыдущего пункта: 2n и $2 * i$ очевидно четные и четное-четное=четное. Тогда как четное+1 = нечетное. И нечетное*нечетное = нечетное, а у нас произведения ряда нечетных чисел -- значит и все произведение будет нечетное.
	
	
	\subsection{Задача 5}
	Пусть первая масть у нас уже стоит в порядке возрастания. Теперь нам нужно вставить в колоду вторую масть в порядке возрастания. Вставлять каждую карту новой масти мы можем как до всех карт первой масти, так и между ними, так и после них, а значит мы точно можем воспользоваться методом перегородок, разбивая их на 14 групп: $\left(_{13}^{13 + 14 - 1}\right)$. 
	У нас станет уже 26 карт в колоде. Будем втсавлять карты третьей масти аналогично. Получим 39 карт и вставим точно таким же способом карты четвертой масти. Тогда итоговая формула будет:  $\left(_{13}^{26}\right) * \left(_{13}^{39}\right) * \left(_{13}^{52}\right)$.
	
	Количество всех способов расположений карт -- $52!$.
	Тогда итоговая вероятность $\frac{52!}{\left(_{13}^{26}\right) * \left(_{13}^{39}\right) * \left(_{13}^{52}\right)!} = \frac{1}{3*13!}$.
	
	\subsection{Задача 6}
	Всего способов выбрать шесть карт из колоды в 36 существует $\left(_{6}^{36}\right) = \frac{31 * 32 * 33 * 34 * 35 * 36}{6!} = 1947792$.
	
	a) Туза всего четыре, тогда в колоде без них 32 карты, а значит выбрать шесть карт без тузов совсем будет $\left(_{6}^{32}\right) = \frac{27 * 28 * 29* 30 * 31 * 32}{6!}$ способа. Значит количество способов выбрать шесть карт с хотя бы одним тузом будет равняться разности количества всех способов выбрать шесть карт и количества способов выбрать шесть карт без тузов.
	
	Тогда вероятность будет равна: $\frac{\frac{31 * 32 * 33 * 34 * 35 * 36}{6!} - \frac{27 * 28 * 29* 30 * 31 * 32}{6!}}{\frac{31 * 32 * 33 * 34 * 35 * 36}{6!}} \approx 0,534$.
	
	б) Рассмотрим как вообще мы можем соотнести четыре масти к шести картам. У нас будет всего два способа такого разбиения: $2+2+1+1$ и $3+1+1+1$. 
	Теперь посчитаем кол-во способов подобрать такие разбиения. 
	Для $2+2+1+1$: Выбрать две карты из одной масти у нас $\left(_{2}^{9}\right)$ и это войдет в квадрате, так как мы выбираем две карты два раза для двух разных мастей. Еще, так как масти четыре, а нам нужно выбрать две из которых мы и будем брать по две карты, то это еще домножим на $\left(_{2}^{4}\right)$ -- кол-во способов выбрать две масти из четырех. И еще нам остается выбрать две любых карты из двух оставшихся мастей -- это $9^2$. В общем получаем $\left(_{2}^{4}\right) * (\left(_{2}^{9}\right))^2*9^2$. 
	Для $3+1+1+1$: Выбрать три карты из одной масти у нас $\left(_{3}^{9}\right)$ и, так как масти четыре нам еще нужно домножить это на 4 -- кол-во способов выбрать одну масть из четырех. Теперь нам нужно выбратть три оставшихся карты других мастей. В других мастях так же по 9 карт, а значит способов сделать это $9^3$. В общем получаем $4 * \left(_{3}^{9}\right) * 9^3$.
	
	Раз у нас два способа различных разбиений на масти, то ответом будет сумма вариантов каждого из способов. 
	Конечная формула: $\left(_{2}^{4}\right) * (\left(_{2}^{9}\right))^2*9^2 + 4 * \left(_{3}^{9}\right) * 9^3$.
	
	А вероятность равна: $\frac{\left(_{2}^{4}\right) * (\left(_{2}^{9}\right))^2*9^2 + 4 * \left(_{3}^{9}\right) * 9^3}{1947792} \approx 0,45$.
	
	
	\subsection{Задача 7}
	Начнем рассматривать комнаты по пордяку. Логично, что так как студентов 7, то у нас существует 7 способов поселить их в одноместную комнату. Далее будем заселять двухместную комнату, но так как мы перед этим уже заселили одноместную, то расчет будем делать уже из шести студентов. Тогда кол-во таких способов: $\left(_{2}^{6}\right) = /frac{6!}{2!*4!} = 15$. В конце будем заселять четырехместную комнату, но, так как у нас к этому моменту осталось всего четыре студента, то и способ будет один. 
	Итоговая формула примет вид $7 * 15 * 1 = 105$.\underline{}

	\subsection{Задача 8}
	Будем выбирать команды, которые играют на домашнем поле. Из 16 нам нужно выбрать 8 команд. Это $\left(_{8}^{16}\right)$. Мы выбрали некую последовательность  команд, играющих на домашнем поле, теперь нужно сопоставить каждому члену этой последовательности противника из оставшихся команд, которых восемь, а значит кол-во способов сопоставить противников $8!$.
	Итоговая формула $\left(_{8}^{16}\right) * 8!$.
	
	\subsection{Задача 9}
	Представим книги числами от одного до 20. После заметим, что нам нужно разделить их четырьмя перегородками, т.е. до первой перегородки -- на первую полку, между первой и второй -- на вторую, ... после четвертой -- на пятую. Тогда добавим эти перегородки в наш ряд  книг(к примеру представим их цифрой "0") и всего цифр получится 24. 
	Тогда чтобы каждая книга побывала на всех возможных местах нам нужно рассмотреть все перестановки получившегося "ряда" чисел, а их будет $24!$.  Но так как мы считали перегородки  тоже за книги у нас появились лишние перестановки. Так как перегородок  было 4, то итоговая формула примет вид $24! / 4! = 10626$.

	\end{document}