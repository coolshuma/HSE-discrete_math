\documentclass[a4paper,12pt]{article}
\usepackage{amsmath}
\usepackage{cmap}					% поиск в PDF
\usepackage{mathtext} 				% русские буквы в формулах
\usepackage[T2A]{fontenc}			% кодировка
\usepackage[utf8]{inputenc}			% кодировка исходного текста
\usepackage[english,russian]{babel}	% локализация и переносы

% Изменим формат \section и \subsection:
\usepackage{titlesec}
\titleformat{\section}
{\vspace{1cm}\centering\LARGE\bfseries}	% Стиль заголовка
{}										% префикс
{0pt}									% Расстояние между префиксом и заголовком
{} 										% Как отображается префикс
\titleformat{\subsection}				% Аналогично для \subsection
{\Large\bfseries}
{}
{0pt}
{}

%% Отступы между абзацами и в начале абзаца 
\setlength{\parindent}{0pt}
\setlength{\parskip}{\medskipamount}

% Перенос знаков в формулах (по Львовскому)
\newcommand*{\hm}[1]{#1\nobreak\discretionary{}
	{\hbox{$\mathsurround=0pt #1$}}{}}

%% Изменяем размер полей
\usepackage[top=1in, bottom=1in, left=1in, right=1in]{geometry}
\begin{document}
	\section{Домашнее задание 9\\ Шумилкин Андрей, группа 163} 
	\subsection{Задача 1}
	\textit{а)} Нарисуем ориентированный граф, согласно танным отношениям и, смотря на него, будет легко составить линейный порядок(рисунок прикрепил во вложениях к письму). Один из возможных порядков таков: Очки < Носки < Брюки < Туфли < Ремень < Рубашка < Галстук < Пиджак < Часы.
	
	\subsection{Задача 2}
	Из определения ацикличности видим, что отношение антирефлексивно. Достаточно взять $k=1$ и заметим, что $\forall a \in A aPa$ не выполняется. \\
	Также из определения ацикличности заметим, что отношение для каждой пары либо выполняется, либо нет, поскольку $\forall a, b \in A aPbPa$ не выполняется, т.е. отношение антисимметрично, а значит и транзитивно. Поскольку если $aPb$ и $bPc$ из-за связности отношения должно быть либо $aPc$, либо $cPa$, но если $aPc$, то это противоречит ацикличности. \\
    А по условию оно так же связно. И, как известно, отношение, обладающее всеми этими свойствами как раз является строгим линейным порядком.
     
     \subsection{Задача 4} 
     По условию отношение антирефлексивно, антисимметрично и связно. Если оно к тому же транзитивно, то это линейный порядок по определению, а если же нет, то тогда и возникают альтернативы a,b,c для которых $aPb, bPc$ и $cPa$, откуда следует(если учесть и антисимметричность), что $aPc$ -- ложно , т.е. отношение нетранзитивно 
     
     \subsection{Задача 7}
     \textit{а)} $I_p = \bar{P} ^{-1} \cap \bar{P}$, поскольку если ни $(x, y)$, ни $(y, x)$ не входят в $P$, то они будут входить в его дополнение, но там же будут и "лишние" пары $(a, b)$, $(b, a)$ которых входит в $P$. Но если мы построим обратное отношение к дополнению $P$, то $(x,y)$ преобразуется в $(y, x)$ и наоборот и войдут в пересечение, а вот $(a, b)$ преобразуется в $(b,a)$ и не войдет в пересечение, так как $(b,a)$ не присутствует в $\bar{P}$.
\end{document}
