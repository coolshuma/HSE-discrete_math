\documentclass[a4paper,12pt]{article}
\usepackage{amsmath}
\usepackage{cmap}					% поиск в PDF
\usepackage{mathtext} 				% русские буквы в формулах
\usepackage[T2A]{fontenc}			% кодировка
\usepackage[utf8]{inputenc}			% кодировка исходного текста
\usepackage[english,russian]{babel}	% локализация и переносы

% Изменим формат \section и \subsection:
\usepackage{titlesec}
\titleformat{\section}
{\vspace{1cm}\centering\LARGE\bfseries}	% Стиль заголовка
{}										% префикс
{0pt}									% Расстояние между префиксом и заголовком
{} 										% Как отображается префикс
\titleformat{\subsection}				% Аналогично для \subsection
{\Large\bfseries}
{}
{0pt}
{}

%% Отступы между абзацами и в начале абзаца 
\setlength{\parindent}{0pt}
\setlength{\parskip}{\medskipamount}

% Перенос знаков в формулах (по Львовскому)
\newcommand*{\hm}[1]{#1\nobreak\discretionary{}
	{\hbox{$\mathsurround=0pt #1$}}{}}

%% Изменяем размер полей
\usepackage[top=1in, bottom=1in, left=1in, right=1in]{geometry}
\begin{document}
	\section{Домашнее задание 17\\ Шумилкин Андрей, группа 163} 
	\subsection{Задача 1}
	Воспользуемся теоремой Кантора-Берштейна. \\
	Составим инъективные отображения из нашей последовательности в последовательность двоичных слов и обратно. \\
	В последовательности двоичных слов мы можем просто между каждыми двумя символами вставить цифру два и тогда мы получим последовательность, состоящию из 0, 1 и 2 в которой по построению никакие два символа не идут подряд, а так же различным двоичным последовательностям соответствуют различные последовательности из 0, 1 и 2. \\
	Нашу же последовательность мы можем перекодировать следующим образом: 0 будем сопоставлять 10, 1 -- 100 и 2 -- 1000. Тогда на выходе мы получим обычную двоичную последовательность, при этом она однозначно будет раскодироваться в последовательность из 0, 1 и 2, что и значит, что мы построили инъекцию. \\
	Тогда по теореме Кантора-Берштейна выходит, что мощность множества наших последовательностей равна мощности множества двоичных последовательностей $\Rightarrow$ оно континуально.
	
	\subsection{Задача 2}
	Заметим, что отношений эквивалентности не меньше континуума, поскольку если взять даже те отношения, которые разбивают множества на два клааса эквивалентности мы можем закодировать их некоторыми бесконечными двоичными последовательностями таким образом: возьмем первый элемент и все элементы эквивалентные ему, включая его самого обозначим 0, а остальные -- 1.\\
	Далее будем идти по элементам и записывать цифру, которую мы сопоставили этому элементу и таким образом будет получаться некая двоичная последовательность, множество которых, как нам известно, континуально. \\
	Теперь заметим, что не больше, поскольку мы каждому отношению можем постаавить в соответствие последовательность натуральных чисел, а множество таких последовательностей так же континуально. \\
	Строить такое соответвствие будем следующим образом: первому элементу сопоставляем класс 1 и все эквивалентные ему элементы так же обозначаем 1. Далее находим первый необозначенный элемент и ставим ему в соответствие 2, а так же всем эквивалентным ему элементам. И т.д. n-ому необозначенному элементу ставим число n. В итоге получаем последовательность натуральных чисел. \\
	Таким образом выходит, что наше множество континуально. 
	
	\subsection{Задача 4}
	Соответствующая ДНФ будет иметь следующий вид:
	\[
		(\lnot x_1 \land \lnot x_2 \land x_3 \land \lnot x_4 \land x_5 \land \lnot x_6 \land x_7 \land \lnot x_8 \land x_9) \lor(\lnot x_1 \land x_2 \land \lnot x_3 \land x_4 \land \lnot x_5 \land x_6 \land \lnot x_7 \land x_8 \land \lnot x_9) \lor
	\]
	\[
		(\lnot x_1 \land \lnot x_2 \land x_3 \land \lnot x_4 \land x_5 \land \lnot x_6 \land x_7 \land \lnot x_8 \land x_9) \lor(\lnot x_1 \land x_2 \land \lnot x_3 \land x_4 \land \lnot x_5 \land x_6 \land \lnot x_7 \land x_8 \land x_9) \lor
	\]
	\[
		(\lnot x_1 \land \lnot x_2 \land x_3 \land \lnot x_4 \land x_5 \land \lnot x_6 \land x_7 \land x_8 \land x_9) \lor(\lnot x_1 \land x_2 \land \lnot x_3 \land x_4 \land \lnot x_5 \land x_6 \land x_7 \land x_8 \land x_9) \lor
	\]
	\[
		(\lnot x_1 \land \lnot x_2 \land x_3 \land \lnot x_4 \land x_5 \land x_6 \land x_7 \land x_8 \land x_9) \lor(\lnot x_1 \land x_2 \land \lnot x_3 \land x_4 \land x_5 \land x_6 \land x_7 \land x_8 \land x_9) \lor
	\]
	\[
		(\lnot x_1 \land \lnot x_2 \land x_3 \land x_4 \land x_5 \land x_6 \land x_7 \land x_8 \land x_9) \lor(\lnot x_1 \land x_2 \land x_3 \land x_4 \land x_5 \land x_6 \land x_7 \land x_8 \land x_9)
	\]
	
	\subsection{Задача 5}
	Мы можем выразить: 
	\[
		\lnot X = X | X
	\]
	\[
		X \lor Y = (X|X)|(Y|Y)
	\]
	\[
		X \land Y = (X|Y)|(X|Y)
	\]
	А мы знаем, что любое выражение можно выразить в виде ДНФ в котором как раз и применяются только эти три операции $\Rightarrow$ мы можем выразить любое выражение через штрих Шеффера $\Rightarrow$ система связок, состоящая только из штриха Шеффера обладает полнотой. 
	
	\subsection{Задача 6}
	В качестве доказательства можно привести алгоритм приведения к КНФ, практически аналогичный алгоритму приведения к ДНФ, только эквивалентность представим по другому. \\
	1. На первом шаге избавляемся от импликаций и эквивалентностей в выражении, представляя их в виде: 
	\[
		A \rightarrow B = \lnot A \lor \lnot B
	\]
	\[
		A \leftrightarrow B = (\lnot A  \lor B) \land (A \lor \lnot B)
	\]
	2. Далее все знаки отрицания, относящиеся к выражениям заменим так, чтобы они относились к конкретным переменным: 
	\[
		\lnot(A \lor B) = \lnot A \land \lnot B
	\]
	\[
		\lnot(A \land B) = \lnot A \lor \lnot B
	\]
	3. Далее избавимся от всех знаков двойного отрицания. \\
	4. Применяем там, где нужно, свойство дистрибутивности  конъюкции и дизъюнкции. \\
	
	В итоге получаем конъюкцию дизъюнкций переменных или их отрицаний. 
\end{document}