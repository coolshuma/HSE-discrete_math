\documentclass[a4paper,12pt]{article}
\usepackage{amsmath}
\usepackage{cmap}					% поиск в PDF
\usepackage{mathtext} 				% русские буквы в формулах
\usepackage[T2A]{fontenc}			% кодировка
\usepackage[utf8]{inputenc}			% кодировка исходного текста
\usepackage[english,russian]{babel}	% локализация и переносы

% Изменим формат \section и \subsection:
\usepackage{titlesec}
\titleformat{\section}
{\vspace{1cm}\centering\LARGE\bfseries}	% Стиль заголовка
{}										% префикс
{0pt}									% Расстояние между префиксом и заголовком
{} 										% Как отображается префикс
\titleformat{\subsection}				% Аналогично для \subsection
{\Large\bfseries}
{}
{0pt}
{}

%% Отступы между абзацами и в начале абзаца 
\setlength{\parindent}{0pt}
\setlength{\parskip}{\medskipamount}

% Перенос знаков в формулах (по Львовскому)
\newcommand*{\hm}[1]{#1\nobreak\discretionary{}
	{\hbox{$\mathsurround=0pt #1$}}{}}

%% Изменяем размер полей
\usepackage[top=1in, bottom=1in, left=1in, right=1in]{geometry}
\begin{document}
	\section{Домашнее задание 12\\ Шумилкин Андрей, группа 163} 
	\subsection{Задача 1}
	Любое подмножество декартова произведения данных множеств будет бинарным отношением, т.е. всего исходов $2^4 = 16$. \\
	Теперь посчитаем кол-во неблагоприятных исходов. Обозначим элементы множества $A$ цифрами, т.е. $a_1$ будет 1, $a_2$ -- 2. Заметим, что это только три подмножества: $\{(1,1),(1,2), (2,1) \}, \{(2,2),(1,2), (2,1) \}  и \{(1,2), (2,1) \}$.\\
	Тогда благоприятных исходов будет 13 и вероятность будет равна кол-ву благоприятных исходов поделеному на кол-во всех исходов.\\
	\textbf{Ответ:} $\frac{13}{16}$.
	
	\subsection{Задача 2}
	Если $a \not = b$, то вероятность будет равна нулю, поскольку мы не сможем построить биекцию между неравномощными множествами. \\
	Будем рассматривать случай. когда $a = b$. Найдем количество благоприятных исходов. Будем сопоставлять множеству $A$ какую-либо перестановку элементов множества $B$ так, что первый элемент множества $A$ будет отображен в первый элемент этой перестановки, второй -- во второй и т.д. Тогда количество благоприятных исходов будет равно $a!$, т.к. $a=b$.  \\
	А количество всех исходов будет равно $a^a$, так как мы каждый элементу из $A$ выбираем отображение в элемент из $B$, которых всего $b$ и $b = a$.  \\
	Тогда вероятность благоприятного исхода будет равна кол-ву благоприятных исходов поделенному на кол-во всех исходов.\\
	\textbf{Ответ:} $\frac{a!}{a^a}$.
	
	\subsection{Задача 3}
	 19-ый элемент в перестановке может быть любым и никак не повлияет на ответ, поскольку не будет в ходить ни в первые 18, ни в последние 18. \\
	 Так как это перестановка, то наибольшее среди первых восемнадцати либо строго больше наибольшего среди последних восемнадцати, либо строго меньше него. Тогда у нас количество благоприятных исходов будет равно количеству неблагоприятных, поскольку мы можем кажой перестановке с благоприятным исходом сопоставить перестановку с неблагоприятным исходом, просто перевернув ее и при этом этом это будет биекция между множеством благоприятных исходов и множеством неблагоприятных. То есть количество благоприятных исходов будет равна кол-ву неблагоприятных и вероятность благоприятного исхода будет равна 0.5 \\
	 \textbf{Ответ:} $\frac{1}{2}$.
	
	\subsection{Задача 4}
	Всего исходов будет $C_{36}^{5}$, то есть мы выбираем из 36 чисел 5 и просто располагаем их в порядке убывания. \\
	Тогда количество благоприятных исходов будет $C_{35}^{4}$, так как единицу мы как бы выбрали заранее и она по построению последовательности будет стоять последней, так как меньше нее чисел нет.\\
	Тогда вероятность благоприятного исхода будет равна кол-ву благоприятных исходов поделенному на кол-во всех исходов.\\
	\textbf{Ответ:} $\frac{5}{36}$.
	  
	\subsection{Задача 5}
	Всего исходов будет $C_{36 + 5 - 1}^{5}$ -- кол-во сочетаний с повторениями.То есть мы просто выбираем 5 элементов(возможно с повторениями) из 36. \\ 
	Тогда количество благоприятных исходов будет $C_{36 + 4 -1}^{4}$, так как единицу мы как бы выбрали заранее и она по построению последовательности будет стоять последней(даже если мы), так как меньше нее чисел нет.\\
	Тогда вероятность благоприятного исхода будет равна кол-ву благоприятных исходов поделенному на кол-во всех исходов.\\
	\textbf{Ответ:} $\frac{39!}{36! \cdot 4!} \cdot \frac{36! \cdot 5!}{40!} = \frac{5}{40} = \frac{1}{8}$.
	
	\subsection{Задача 6}
	Будем рассматривать первые 10 позиций и последние 10. Обозначим их множества как $A$ и $B$.\\
	У нас может быть три случая:\\
	$1.|A| > |B| \\
	2.|A| < |B| \\
	3.|A| = |B|$.\\
	
	Все двоичные слова подходящие под первый случай будут неблагоприятными исходами, поскольку что бы ни стояло на 11 позиции в последних 11-ти единиц будет уже явно меньше или равно, чем в первых 10, т.е. в первых десяти никак не может быть меньше единиц, чем в последних 11-ти. \\
	Все двоичные слова подходящие под второй случай будут благоприятными исходами, поскольку что бы ни стояло на 11 позиции в последних 11-ти единиц будет уже явно больше, чем в первых 10, потому что их и так больше, а мы либо добавляем еще единицу, либо оставляем все как есть. \\
	Мы можем построить биекцию между множеством двоичных слов, подходящих к первому случаю и множеством двоичных слов, подходящих ко второму. Тогда после рассмотрения двух этих случаев у нас будет поровну благоприятных случаев и неблагоприятных. \\
	В третьем же случае все будет зависить от того стоит ли на 11-ой позиции единица или ноль, а значит половина будет благоприятными исходами, а половина -- нет, так как у нас множество \textit{всех} двоичных слов. \\
	То есть у нас по итогу равное количество благоприятныхх и неблагоприятных исходов, а значит вероятность благоприятного исхода будет равна 0.5. \\
	\textbf{Ответ:} $\frac{1}{2}$.
	
	\subsection{Задача 7}
	Вероятностное пространство -- 
	Посчитаем вероятность неблагоприятного исхода, тогда когда оно будет меньше или равно $0.5$  и будет достигаться то, что вероятность благоприятного исхода будет больше или равно $0.5$. \\
	Обозначим кол-во карт, которые мы вытягиваем $x$. Тогда количество всех исходов будет равно $C_{36}^{x}$, а количество неблагоприятных $C_{32}^{x}$. Значит вероятность неблагоприятного исхода равноа $\frac{(36-x)!}{(32-x)! \cdot 33 \cdot 34 \cdot 35 \cdot 36} = \frac{(33 - x) \cdot (34 - x) \cdot (35 - x) \cdot (36 -x )}{33 \cdot 34 \cdot 35 \cdot 36}$. \\
	Далее посмотрим значение этого выражения при x и когда оно достигнет $0.5$ мы и найдем ответ. \\
	При $x = 1$ оно будет равно $1256640 / 1413720$. \\
	При $x = 2$ оно будет равно $1113024 / 1413720$. \\
	При $x = 3$ оно будет равно $982080 / 1413720$. \\
	При $x = 4$ оно будет равно $863040 / 1413720$. \\
	При $x = 5$ оно будет равно $755160 / 1413720$. \\
	При $x = 6$ оно будет равно $657720 / 1413720$. \\\\
	Видим, что при $x=5$ оно еще больше 0.5, а при $x=6$ меньше. Значит достаточно вытащить шесть карт из колоды.\\
	\textbf{Ответ:} достаточно вытащить 6 карт.
	
	\subsection{Задача 8}
	Вероятностное пространство -- группы студентов по 30 человек\\
	Посчитаем количество неблагоприятных исходов, т.е. такую ситуацию, когда ни у окого из группы не будут совпадать дни рождения(остальные будут благоприятными, поскольку если день рождения совпадают у троих человек, то они, конечно же, совпадают и у двоих). \\
	Берем день рождения первого человека и вероятность того, что оно не совпадет с день рождением кого-то из ранее выбранных будет равна 1, так как до этого мы никого еще не выбрали. Затем будем рассматривать день рождения второго. Вероятность того, что они различны будет $\frac{364}{365}$, поскольку благоприятными исходами будут 364 дня, исключая день рождения первого человека из группы. Для третьего выбранного соответственно вероятность будет $\frac{363}{365}$. И так далее, а дял последнего $\frac{365 - 30}{365} = \frac{335}{365}$. \\
	Тогда итоговая вероятность благоприятного исхода будет $\frac{365 \cdot 364 \cdot 363 \cdot \ldots \cdot 335}{365^n}$ и она будет примерно равна $0.3$. \\
	А количество благоприятных исходов будет равно 1 - (кол-во неблагоприятных), т.е. будет больше или равно $0.6$, а значит больше $0.5$.
	
	\subsection{Линал}
	При фиксированном $m=10$, то есть когда $n$ заметно больше чем $m$ эффективнее оказывается метод Гаусса. 
	
	Иначе же, если фиксированное $n = 100$, то есть $m$ заметно больше, то эффективнее оказывается метод, использующий обратную матрицу. 
	
	Посчитаем их сложность. При этом, поскольку лабораторная по линейной алгебре, будем руководствоваться методами, которые мы изучили в курсе линейной алгебры, не беря во внимание, что во внутренних функциях python они каким-то образом могут быть реализованы несколько быстрее. 
	
	Метод обратной матрицы: 
	1. Мы можем найти обратную матрицу за $O(n^3)$, методом Гаусса-Жордана, к примеру. 
	2. Потом нам надо будет обратную матрицу умножить на матрицу $B$ и это можно сделать за $O(n^2 \cdot m)$ операций.
	
	Итог: $O(n^2 \cdot m + n^3).$ 
	
	Метод Гаусса: 
	1. Сложность самого метода Гаусса составляет $O(n^3)$, но нам нужно еще повторять каждую операцию на присоединенной нами матрице $B$, то есть, по сути, нам нужно $n^2$ раз сложить ее строки и тогда итоговая сложность будет $O(n^3 + n^2 \cdot m)$.
\end{document}