\documentclass[a4paper,12pt]{article}

%% Работа с русским языком
\usepackage{cmap}					% поиск в PDF
\usepackage{mathtext} 				% русские буквы в формулах
\usepackage[T2A]{fontenc}			% кодировка
\usepackage[utf8]{inputenc}			% кодировка исходного текста
\usepackage[english,russian]{babel}	% локализация и переносы
\usepackage{amsmath, amsfonts, amsthm, mathtools, amssymb, icomma, units}
\usepackage{algorithmicx, algorithm}
\usepackage{algpseudocode}

%% Отступы между абзацами и в начале абзаца 
\setlength{\parindent}{0pt}
\setlength{\parskip}{\medskipamount}

%% Изменяем размер полей
\usepackage[top=0.5in, bottom=0.75in, left=0.825in, right=0.825in]{geometry}

%% Графика
\usepackage[pdftex]{graphicx}
\graphicspath{{images/}}

%% Различные пакеты для работы с математикой
\usepackage{mathtools}				% Тот же amsmath, только с некоторыми поправками

%\usepackage{amssymb}				% Математические символы
\usepackage{amsthm}					% Пакет для написания теорем
\usepackage{amstext}
\usepackage{array}
\usepackage{amsfonts}
\usepackage{icomma}					% "Умная" запятая: $0,2$ --- число, $0, 2$ --- перечисление
\usepackage{bbm}				    % Для красивого (!) \mathbb с  буквами и цифрами
\usepackage{enumitem}               % Для выравнивания itemise (\begin{itemize}[align=left])

% Номера формул
\mathtoolsset{showonlyrefs=true} % Показывать номера только у тех формул, на которые есть \eqref{} в тексте.

% Ссылки
\usepackage[colorlinks=true, urlcolor=blue]{hyperref}

% Шрифты
\usepackage{euscript}	 % Шрифт Евклид
\usepackage{mathrsfs}	 % Красивый матшрифт

% Свои команды\textbf{}
\DeclareMathOperator{\sgn}{\mathop{sgn}}

% Перенос знаков в формулах (по Львовскому)
\newcommand*{\hm}[1]{#1\nobreak\discretionary{}
	{\hbox{$\mathsurround=0pt #1$}}{}}

% Графики
\usepackage{tikz}
\usepackage{pgfplots}
%\pgfplotsset{compat=1.12}

% Изменим формат \section и \subsection:
%\usepackage{titlesec}
%\titleformat{\section}
%{\vspace{1cm}\centering\LARGE\bfseries}	% Стиль заголовка
%{}										% префикс
%{0pt}									% Расстояние между префиксом и заголовком
%{} 										% Как отображается префикс
%\titleformat{\subsection}				% Аналогично для \subsection
%{\Large\bfseries}
%{}
%{0pt}
%{}

% Информация об авторах
\title{Лекции по предмету \\
	\textbf{Линейная алгебра и геометрия}}

\newtheorem*{Def}{Определение}
\newtheorem*{Lemma}{Лемма}
\newtheorem*{Suggestion}{Предложение}
\newtheorem*{Examples}{Пример}
%\newtheorem*{Comment}{Замечание}
\newtheorem*{Consequence}{Следствие}
\newtheorem*{Theorem}{Теорема}
\newtheorem*{Statement}{Утверждение}
\newtheorem*{Task}{Упражнение}
\newtheorem*{Designation}{Обозначение}
\newtheorem*{Generalization}{Обобщение}
\newtheorem*{Thedream}{Предел мечтаний}
\newtheorem*{Properties}{Свойства}


\renewcommand{\Re}{\mathrm{Re\:}}
\renewcommand{\Im}{\mathrm{Im\:}}
\newcommand{\Arg}{\mathrm{Arg\:}}
\renewcommand{\arg}{\mathrm{arg\:}}
\newcommand{\Mat}{\mathrm{Mat}}
\newcommand{\id}{\mathrm{id}}
\newcommand{\isom}{\xrightarrow{\sim}} 
\newcommand{\leftisom}{\xleftarrow{\sim}}
\newcommand{\Hom}{\mathrm{Hom}}
\newcommand{\Ker}{\mathrm{Ker}\:}
\newcommand{\rk}{\mathrm{rk}\:}
\newcommand{\diag}{\mathrm{diag}}
\newcommand{\ort}{\mathrm{ort}}
\newcommand{\pr}{\mathrm{pr}}
\newcommand{\vol}{\mathrm{vol\:}}
\def\limref#1#2{{#1}\negmedspace\mid_{#2}}
\newcommand{\eps}{\varepsilon}

\renewcommand{\epsilon}{\varepsilon}
\renewcommand{\phi}{\varphi}
\newcommand{\e}{\mathbb{e}}
\renewcommand{\l}{\lambda}
\renewcommand{\C}{\mathbb{C}}
\newcommand{\R}{\mathbb{R}}
\newcommand{\E}{\mathbb{E}}

\newcommand{\vvector}[1]{\begin{pmatrix}{#1}_1 \\\vdots\\{#1}_n\end{pmatrix}}
\renewcommand{\vector}[1]{({#1}_1, \ldots, {#1}_n)}

%Теоремы
%11.01.2016
\newtheorem*{standartbase}{Теорема о стандартном базисе}
\newtheorem*{fulllemma}{Лемма}
\newtheorem*{sl1}{Следствие 1}
\newtheorem*{sl2}{Следствие 2}
\newtheorem*{monotonousbase}{Теорема о монотонном базисе}
\newtheorem*{scheme}{Утверждение 1}
\newtheorem*{n2}{Утверждение 2}
\newtheorem*{usp-rais}{Теорема Успенского-Райса}
\newtheorem*{rec}{Свойство рекурсии}
\newtheorem*{point}{Теорема о неподвижной точке}
\newtheorem*{zhegalkin}{Теорема Жегалкина}
\newtheorem*{poste}{Теорема Поста}
\newtheorem*{algo1}{Первое свойство алгоритмов}

%18.01.2016
\newtheorem*{theorem}{Теорема}

\renewcommand{\qedsymbol}{\textbf{Q.E.D.}}
\newcommand{\definition}{\underline{Определение:} }
\newcommand{\definitions}{\underline{Определения:} }
\newcommand{\definitionone}{\underline{Определение 1:} }
\newcommand{\definitiontwo}{\underline{Определение 2:} }
\newcommand{\statement}{\underline{Утверждение:} }
\newcommand{\note}{\underline{Замечание:} }
\newcommand{\sign}{\underline{Обозначения:} }
\newcommand{\statements}{\underline{Утверждения:} }

\newcommand{\Z}{\mathbb{Z}}
\newcommand{\N}{\mathbb{N}}
\newcommand{\Q}{\mathbb{Q}}
\begin{document}
	\section{Домашнее задание 13\\ Шумилкин Андрей, группа 163} 
	\subsection{Задача 1}
	Пусть  множество исходов -- это последовательность, в которой родились дети и тогда все исходы равновероятны, так как рождение мальчика и девочки равновероятны. То, что мальчик родился в понедельник никак не влияет на вероятность, поскольку в тот же понедельник могла за ним родиться девочка или наоборот(двойняшки) и по условию рождение каждого -- равновероятно. Обозначим М -- мальчик, Д -- девочка. Тогда множество исходов имеет вид $\{ММ, МД, ДМ, ДД\}$ и вероятность каждого исхода равна 1/4. С учетом того, что по условию один из детей должен быть мальчиком, а другой -- девочкой, нам пододят два из этих исходов -- МД и ДМ, и тогда итоговая вероятность равна 1/2.\\
	\textbf{Ответ:} $\frac{1}{2}$.
	
	\subsection{Задача 3}
	Пусть множество исходов -- это пятерки выбранных чисел и все исходы равновероятны. \\
	Количество всех исходов равно $C_{36}^{5}$.
	Вероятность выбрать пятерку, среди элементов которой есть число 2 будет равна $\frac{C_{35}^{4}}{C_{36}^{5}}$, так как пятерка должна быть выбрана по условию и еще четыре элемента мы выбираем из оставшихся 35-и элементов множества. Вероятность выбрать пятерку, среди элементов которой будет число 5 так же равна $\frac{C_{35}^{4}}{C_{36}^{5}} = \frac{5}{36}$.\\
	Теперь посчитаем вероятность выбрать элемент 5 при том, что двойка уже выбрана. Вероятность выбрать пятерку в которой есть элементы и 5, и 2 равна $\frac{C_{34}^{3}}{C_{36}^{5}}$. \\
	Тогда вероятность выбрать элемент 5 при том, что элемент 2 уже выбран равна $\frac{C_{34}^{3}}{C_{36}^{5}} \cdot \frac{C_{36}^{5}}{C_{35}^{4}} = \frac{C_{34}^{3}}{C_{35}^{4}} = \frac{4}{35}$.\\
	И, так как $\frac{5}{36} \not= \frac{4}{35}$, то есть вероятность события не равна вероятности его же при условии какого-то другого делаем вывод, что события зависимы. \\
	\textbf{Ответ:} Данные события зависимы.
	
	\subsection{Задача 4}
	Пусть множество исходов -- это какая-то определенная функция и все исходы равновероятны. 
	Всего вариантов составить какую-либо функцию у нас будет $n^n$, поскольку она всюду определена и каждому из $n$ элементов одного множества мы можем и должны сопоставить один из $n$ элементов другого множества.\\
	Составить же инъективную функцию у нас $n!$ способов, то есть мы будем элементам первого множества сопоставлять некую перестановку из элементов второго. Значит вероятность того, что функция инъективна равна $\frac{n!}{n^n}$.  \\
	Составить функцию так, чтобы $f(1) = 1$ у нас будет $(n-1)^n$ способов, то есть один элемент мы определяем изначально, а для остальных $n-1$ выбираем один из $n$ элементов другого множества. Тогда вероятность того, что $f(1) = 1$ будет $\frac{1}{n}$\\
	Если же хотим получить инъективную функцию у которой $f(1) = 1$, то у нас будет $(n-1)!$ способов сделать это, поскольку мы определяем $f(1) = 1$  и нам нужно для оставшихся $n-1$ элемента сопоставить различные элементы другого множества, в которое уже не входит 1, так как мы уже составили с ней пару. \\
	Тогда вероятность того, что $f(1) = 1$, при условии, что функция инъективна равна $\frac{(n-1)!}{n^n} \cdot \frac{n!}{n^n} = \frac{1}{n}$. \\
	Можно заметить, что она равна вероятности того, что $f(1) = 1$, а значит данные события независимы. \\
	\textbf{Ответ:} Данные события независимы.
	
	\subsection{Задача 5}
	Пусть множество исходов -- \\
	Обозначим правильное решение как 1, а неправильное -- как 0. Рассмотрим все возможные исходы выбора первыми двумя членами жюри. Это: $\{00, 01, 10, 11\}$. \\
	Теперь посчитаем вероятность того, что третий член жюри сделает правильный выбор для каждого случая:
	\begin{enumerate}
		\item 00. Итоговая вероятность равна 0, так как правильного решения тут нет.
		\item 01. Вероятность такого случая равна $(1 - p) \cdot p$, тогда вероятость правильного выбора третьим членом жюри равна $\frac{p \cdot (1 -p)}{2}$.
		\item 10. Вероятность такого случая равна $p \cdot (1-p)$, тогда вероятость правильного выбора третьим членом жюри равна $\frac{p \cdot (1 - p)}{2}$.
		\item 11. Вероятность такого случая равна $p^2$, тогда вероятость правильного выбора третьим членом жюри равна $p^2$.
	\end{enumerate}
	А общая вероятность выбора правильного решения третьим из судей будет равна сумме данных вариантов: \\
	$p \cdot (1 - p) + p^2 = p^2 + p - p^2 = p$. \\
	Видно, что эта вероятность равна $p$ -- вероятности правильного решения, принимаемого одним добросовестным членом жюри.
	\textbf{Ответ:} Вероятность выбора верного решения равна $p$ и равна вероятности правильного решения, принимаемого одним добросовестным членом жюри.
	
	\subsection{Задача 7} 
	Посчитаем вероятность как бы спускаясь вниз, при этом вероятность выигрыша при счете 10:9 возьмем, конечно же, за 1. \\
	Тогда вероятность выигрыша при 9:9 = 1/2. \\
	9:8 -- это будет сумма вероятностей вариантов (выиграть сразу) и (проиграть, а потом выиграть) и она равна  $1/2 \cdot 1/2 + 1/2 = 3/4$. \\
	9:7 -- это будет сумма вероятностей вариантов (выиграть сразу) и (перейти в случай 9:8), для которого мы уже посчитали вероятность и тогда она будет равна $1/2 + 1/2 \cdot 3/4 = 7/8/$.
	8:9 -- для выигрыша можем перейти в вариант 9:9 для которго уже посчитали, тогда равна $1/2 \cdot 1/2 = 1/4$.
	8:8 -- перйти в вариант 9:8, либо 8:9, тогда равна $1/2 \cdot 3/4 + 1/2 \cdot 1/4 = 1/2$. 
	8:7 -- перейти в вариант 9:7, либо в вариант 8:8. Тогда вероятность равна $1/2 \cdot 1/2 + 1/2 \cdot 7/8 = 11/16$.\\
	\textbf{Ответ:} $11/16$.\\\\\\\\
	Вероятность $A$ при условии $B$:
	\[
		Pr[A|B] = \frac{Pr[A \cap B]}{Pr[B]}.\\
	\]
	\[
	Pr[A|B] = Pr[A] \cdot \frac{Pr[B|A]}{Pr[B]}.
	\]
	
		
\end{document}