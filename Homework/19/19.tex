\documentclass[a4paper,12pt]{article}
\usepackage{amsmath}
\usepackage{cmap}					% поиск в PDF
\usepackage{mathtext} 				% русские буквы в формулах
\usepackage[T2A]{fontenc}			% кодировка
\usepackage[utf8]{inputenc}			% кодировка исходного текста
\usepackage[english,russian]{babel}	% локализация и переносы

% Изменим формат \section и \subsection:
\usepackage{titlesec}
\titleformat{\section}
{\vspace{1cm}\centering\LARGE\bfseries}	% Стиль заголовка
{}										% префикс
{0pt}									% Расстояние между префиксом и заголовком
{} 										% Как отображается префикс
\titleformat{\subsection}				% Аналогично для \subsection
{\Large\bfseries}
{}
{0pt}
{}

%% Отступы между абзацами и в начале абзаца 
\setlength{\parindent}{0pt}
\setlength{\parskip}{\medskipamount}

% Перенос знаков в формулах (по Львовскому)
\newcommand*{\hm}[1]{#1\nobreak\discretionary{}
	{\hbox{$\mathsurround=0pt #1$}}{}}

%% Изменяем размер полей
\usepackage[top=1in, bottom=1in, left=1in, right=1in]{geometry}
\begin{document}
	\section{Домашнее задание 19\\ Шумилкин Андрей, группа 163} 
	\subsection{Задача 1}
	Если граф задан матрицей смежности, то нам достаточно просто для каждой вершины подсчитать дизъюнкцию всех переменных в ее строке, кроме той, что стоит на диагонали и тогда мы получим для каждой вершины переменную равную единице только тогда, когда данная вершина соединенена с какой-то другой. \\
	Тогда вторым шагом нам достаточно подсчитать конъюкцию всех этих вершин, которая будет равна нулю, когда хотя бы одна вершина не соединена ни с какими другими вершинами. Значит нам нужно просто взять отрицание подсчитанной конъюкции. \\
	И так как мы просто один раз просматриваем матрицу смежности, размер которой $n^2$, то схема получится полиномиальной. 
	
	\subsection{Задача 2}
	Заметим, что выбрать в графе три вершины мы можем $C_n^3$ способами, т.е. $\frac{n(n-1)(n-2)}{6}$ и данное выражение является полиномом. \\
	Тогда будем выбирать такие тройки, делать конъюнкцию элементов, стоящих в них, которая будет равна единице тогда, когда они образуют треугольник. Тогда отрицание дизъюнкции конъюнкций всех троек как раз и вернет ответ, поскольку просто дизъюнкция  конъюкций всех троек будет равна единице как раз тогда, когда в графе есть один треугольник. \\
	И так как троек всего полиномиальное количество, то и схема получится полиномаилального размера.
	
	\subsection{Задача 3}
	Чтобы в графе существовал Эйлеров цикл он должен быть связен и все вершины в нем должны быть четной степени. \\
	Четность степени всех вершин мы можем проверить для каждой вершины подсчитав xor(который так же называют сложением по модулю 2) всех переменных в ее строке матрицы смежности, кроме той, что стоит на диагонали. Заметим, что это значение будет равно нулю, если кол-во единиц будет четным. Тогда нам достаточно взять конъюкцию отрицаний данных значений для всех вершин и значение данного выражения будет равно единице, когда степени всех вершин в графе четны. Размерность этой схемы так же будет полиномиальна, так как мы просто раз просматриваем матрицу смежности. \\
	А то, что можно проверить связность графа схемой, глубиной не больше О($log^2 n$) и то, как это сделать с помощью булевых степеней матрицы смежности мы рассматривали на лекции, тогда достаточно просто сделать конъюкцию получившихся из двух данных схем значений и, так как они обе имеют полиномиальный размер, то и итоговая схема так же будет полиномиальна. 
	
	\subsection{Задача 4}
	Мы знаем, что любую функцию можно записать с помощью конъюкций и дизъюнкций просто представив ее в ДНФ. \\
	Так же заметим, что всего функций будет $2^n$. Тогда остается заметить, что СДНФ минимальной функции  будет представима как дизъюнкция конъюкций, без отрицаний и размер такой схемы будет $n$. \\
	Тогда размер общей схемы будет как раз О($n*2^n$).
		
\end{document}