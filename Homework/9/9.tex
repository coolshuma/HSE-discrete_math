\documentclass[a4paper,12pt]{article}
\usepackage{amsmath}
\usepackage{cmap}					% поиск в PDF
\usepackage{mathtext} 				% русские буквы в формулах
\usepackage[T2A]{fontenc}			% кодировка
\usepackage[utf8]{inputenc}			% кодировка исходного текста
\usepackage[english,russian]{babel}	% локализация и переносы

% Изменим формат \section и \subsection:
\usepackage{titlesec}
\titleformat{\section}
{\vspace{1cm}\centering\LARGE\bfseries}	% Стиль заголовка
{}										% префикс
{0pt}									% Расстояние между префиксом и заголовком
{} 										% Как отображается префикс
\titleformat{\subsection}				% Аналогично для \subsection
{\Large\bfseries}
{}
{0pt}
{}

%% Отступы между абзацами и в начале абзаца 
\setlength{\parindent}{0pt}
\setlength{\parskip}{\medskipamount}

% Перенос знаков в формулах (по Львовскому)
\newcommand*{\hm}[1]{#1\nobreak\discretionary{}
	{\hbox{$\mathsurround=0pt #1$}}{}}

%% Изменяем размер полей
\usepackage[top=1in, bottom=1in, left=1in, right=1in]{geometry}
\begin{document}
	\section{Домашнее задание 9\\ Шумилкин Андрей, группа 163} 
	\subsection{Задача 1}
	Только для бинарных отношений над пустыми множествами.\\
	Поскольку над любым непустым множством $A$ отношение вида $P(x, x), x \in A$ будет принадлежать либо $\bar{P}$, либо $P^{-1}$, поскольку: \\
	Пусть изначальное отношение $P$ включает в себя пары $(x,x)$, тогда его дополнение не будет их включать, а вот обратное отношение будет. \\
	Если же изначальное отношение не содержит данные пары, то его дополнение будет их включать, а вот обратное отношение нет. 
	\subsection{Задача 2}
	\textit{a)} Нет, не будет. Контрпример: \\
	$A = {a_1, a_2, a_3}$. $P_1 = {(a_1, a_3)}$. $\bar{P_1} = {(a_1, a_1), (a_1, a_2), (a_2, a_1), (a_2, a_2), (a_2, a_3), (a_3, a_1), (a_3, a_2), (a_3, a_3)}$.
	Как видим, $P_1$ транзитивно, а вот $\bar{P_1}$ нет, поскольку $\bar{P_1}(a_1, a_2) \land \bar{P_1}(a_2, a_3) \not\Rightarrow \bar{P_1}(a_1, a_3)$.\\
	\textit{б)} Назовем пересечение множеств $I$ -- оно будет транзитивно, поскольку пусть $(a, b) \in I \land (b, c) \in I$ -- из этого по определению пересечения множест следует, что $(a, b) \in P_1 \land (b, c) \in P_1$ и $(a, b) \in P_2 \land (b, c) \in P_2$, а так как и $P_1\ и\ P_2$ транзитивны, то из этого следует $(a,c) \in P_1, P_2$, а так как $I$ -- пересечение множеств, то и $(a,c) \in I$.\\
	\textit{в)} Контрпример. $A = {a, b, c, d}, P_1 = {(a, b), (b, c), (a, c)}, P_2 = {(c, d)}$. Тогда $P_1 \cup P_2 = {(a, b), (b, c), (a, c), (c, d)}$ и оно не транзитивно, поскольку $P_1 \cup P_2(a,c) \land P_1 \cup P_2(c,d) \not\Rightarrow P_1 \cup P_2(a,d)$.\\
	\textit{г)} Нет. Контрпример: $P_1 = {(c, b), (x, c), (x, b)}, P_2 = {(a, c), (b, x)}$. Оба отношения транзитивны и $P_1 \circ P_2 = {(a, b), (b, c), (b, b)}$ и, как видно, композиция не транзитивна, так как  $P_1 \circ P_2(a, b) \land P_1 \circ P_2(b, c) \not\Rightarrow P_1 \circ P_2(a,c)$.
	
	\subsection{Задача 3} 
	Обозначим отношение $R$, карты меньшие или равные десятке соответствующей цифрой и  карты больше десятки заглавной буквой, которая является первой в слове их обозначающем: Валет - В и т.д. Будем, не теряя общности, рассматривать карту произвольной масти и обозначать просто согласно выбранным обозначением без указания конкретной масти.\\
	\textit{a)} Да, будет, потому что сказано "одна из карт ... другая"$,$ а не "первая ... вторая"$,$ к примеру. То есть будет спаведливо как $R$(6, Валет), так и $R$(Валет, 6). \\
	\textit{б)} Нет, оно нерефликсивно. Потому что никакая карта не может одновременно быть как младше деятки, так и старше десятки, т.е. контрпример: $R$(8, 8) -- ложно. \\
	\textit{в)} Нет, оно не транзитивно. Контрпример $R$(9, Валет) $\land$ $R$(Дама, 8) -- истинно, но из $R$(9, Валет) $\land$ $R$(Дама, 8) $\not\Rightarrow$ $R$(9, 8), поскольку $R$(9, 8) ложно.\\
	Посчитаем кол-во возможных пар относительно карт меньших десятки, при этом уже рассматривая карты конкретных мастей. \\
	Карт меньших десятки у нас 4 и у каждой из них четыре масти. Ставим их на место слева в нашем отношении. Тогда справа может стоять любая карта, большая десятки, которых так же четыре и у каждой четыре масти. Также нужно не забыть домножить все это на 2, так как каждую пару мы можем "перевернуть"\ и она так же будет истинна. \\
	$4 \cdot 4 \cdot 4 \cdot 4 \cdot 2 = 512$.
	\subsection{Задача 4} 
	\textit{a)} Да, может, если операция нерефлексивна. Тогда, к примеру, мы можем взять все возможные пары для первого элемента, притом так, чтобы он был как слева, так и справа. Всего получится 30 пар. Потом возьмем пару второго третьим, и наоборот -- третьего со вторым. Выходит 32 пары. И последней парой возьмем (первый элемент, первый элемент). Выходит 33 элемента и полученное отношение симметрично. \\
	
	\subsection{Задача 5} 
	\textit{a)} Любоая комбинация пар элементов $A$(даже пустая) определяется неким отношением на множестве. Посчитаем общее количество пар возможных для составления из элементов множества $A$. \\
	Мы каждый элеммент можем поставить в пару с каждым, значит это $2^n$. 
	Как известно, количество подмножест множества $C$ равно $2^{|C|}$. Тогда количество всех возможных комбинаций пар равно $2^{n^2}$. \\
	\textit{б)} Чтобы отношение было рефлексивным оно должно содержать в себе $(x, x), \forall x \in A$. Тогда, можем сказать, что нужно посчитать количество подмножеств множества, n элементов которого принадлежат любому из подмножеств и это $2^{n^2 - n}$. \\
	\textit{в)} Построим для каждого отношения матрицу, состоящую из единиц и нулей. Единица будет стоять в клетке с индексом $(x, y)$ тогда, когда в отношение входит пара $(x, y)$. Тогда заметим, что мы можем ставить единицы на главной диагонали или выше нее и, чтобы соблюдалось условие симметричности отношения, так же ставить единицу в клетке ниже главной диагонали, симметричной относительно нами выбранной относительно главной диагонали. А всего клеток выше главной диагонали(считаем идя по столбцам или строкам) $1+2+ \ldots + n$ -- арифметическая прогрессия, сумма которой равна $\frac{n^2 + n}{2}$. И, по уже известной формуле количества подмножеств множества получаем $2^{\frac{n^2 + n}{2}}$.\\
	
	
	\subsection{Задача 6}
	\textit{a)}Из данных отношений мы можем посмотреть как отношение $P$ делит множество $A$ на классы эквивалентности. Мы можем заметить, что $a,b,c$ лежат в одном классе, а $d,e$ в другом. \\
	Тогда весь вопрос к какому классу относится f, при этом f точно не принадлежит ко второму классу, где $d,e$ по условию. Тогда она может либо принадлежать тому же классу, что и $a,b,c$, либо входить в новый класс, куда будет относиться только этот элемент -- $f$.\\
	Выходит, что у нас будет либо два, либо три класса эквивалентности. \\
	\textit{б)} Данный элемент, так как про него ничего не известно может либо относится к одному из уже существующих классов эквивалентности, либо входить в свой собственный класс, куда относится только он. 
	И вариантов какие классы могут получится при наличествовании такого элемента в $A$ получится семь, поскольку: \\
	Если $f$ принадлежит классу эквивалентности, где лежат $a,b,c$, то $g$ может либо так же принадлежать этому классу, либо принадлежать классу, где находятся $d,e$, либо образовывать свой собственный новый класс -- всего 3 варианта. \\
	Если же $f$ образует свой собственный класс, то $g$ может либо относиться к одному из трех существующих классов, либо образовывать свой собственный класс -- всего 4 варианта.
	
\end{document}
