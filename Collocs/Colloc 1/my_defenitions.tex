\documentclass[a4paper,12pt]{article}
\usepackage{amsmath}
\usepackage{cmap}					% поиск в PDF
\usepackage{mathtext} 				% русские буквы в формулах
\usepackage[T2A]{fontenc}			% кодировка
\usepackage[utf8]{inputenc}			% кодировка исходного текста
\usepackage[english,russian]{babel}	% локализация и переносы

% Изменим формат \section и \subsection:
\usepackage{titlesec}
\titleformat{\section}
{\vspace{1cm}\centering\LARGE\bfseries}	% Стиль заголовка
{}										% префикс
{0pt}									% Расстояние между префиксом и заголовком
{} 										% Как отображается префикс
\titleformat{\subsection}				% Аналогично для \subsection
{\Large\bfseries}
{}
{0pt}
{}

%% Отступы между абзацами и в начале абзаца 
\setlength{\parindent}{0pt}
\setlength{\parskip}{\medskipamount}

% Перенос знаков в формулах (по Львовскому)
\newcommand*{\hm}[1]{#1\nobreak\discretionary{}
	{\hbox{$\mathsurround=0pt #1$}}{}}

%% Изменяем размер полей
\usepackage[top=1in, bottom=1in, left=1in, right=1in]{geometry}

\title{Дискретная математика\\Определения\\ (!!!) -- неуверенность в определении}
\begin{document}
\maketitle

\section{Принцип математической индукции.}

\textbf{Математическая индукция} -- метод математического доказательства, который используется, чтобы доказать истинность некоторого утверждения для всех натуральных чисел. Для этого сначала проверяется истинность утверждения с номером 1 — \textbf{база} индукции, а затем доказывается, что, если верно утверждение с номером $n$, то верно и следующее утверждение с номером $n + 1$ — \textbf{шаг} индукции, или индукционный переход.\\

Строгая формулировка \textbf{принципа математической индукции}:\\
Пусть у нас имеется последовательность утверждений $P_1, P_2, P_3, \ldots$. И пусть утверждение $Y_1$ верно \textit{(база индукции)}, и мы умеем доказывать, что, из верности утверждения $P_{n-1}$, следует верность утверждения $P_{n}$\textit{(шаг индукции)}. Тогда все утверждения в данной последовательности верны. \\

Также существует так называемый \textbf{принцип \textit{полной} математической индукции:}\\
Пусть у нас имеется последовательность утверждений $P_1, P_2, P_3, \ldots$. Если для любого натурального $n$ из того, что истинны все утверждения до него, т.е. $P_1, P_2, \ldots P_{n-1}$ следует истинность самого $P_n$, то все утверждения в это последовательности истинны . \\
В этой вариации база индукции оказывается лишней, поскольку оказывается частным случаем индукционного перехода.\\

Примеры:
\begin{enumerate}
	\item Доказательство формулы суммы арифметической прогрессии. 
	\item Доказательство существования в графе-турнире гамильтонова пути.
	\item Доказательство представимости любой подстановки в виде произведения транспозиций.
\end{enumerate}


\section{Правила суммы, произведения, дополнения. Конечные слова в алфавите. Перестановки, формулы для числа перестановок. Двоичные слова, подмножества конечного множества.}

\textbf{Правило суммы} -- если элемент A можно выбрать n способами, а элемент B можно выбрать m способами и данные способы никак не пересекаются, то выбрать A \textit{или} B можно n + m способами. \\
Из учебника: Другими словами, если надо подсчитать количество объектов какого-то вида,
и это объекты можно поделить на непересекающиеся типы, то общее количество
объектов равно сумме количеств объектов каждого типа.

\textbf{Правило произведения} -- если элемент $A$ можно выбрать $n$ способами, а элемент $B$ можно выбрать $m$ способами, то выбрать $A$ \textit{и} $B$ можно $n + m$ способами. \\
Из учебника: Если объект интересующего нас вида строится в несколько шагов, и на каждом шаге есть выбор из какого-то числа вариантов, то общее количество объектов равно произведению количеств вариантов выбора для каждого из шагов. 

\textbf{Правило дополнения(!!!)} -- Если $X$ является подмножеством $Y$, то разность множеств $X$ и $Y$ называется дополнением множества $Y$ в множестве $X$. \\

\textbf{Конечные слова в алфавите(!!!)} -- не нашел.\\

\textbf{Перестановка из $n$ элементов} --  это упорядоченный набор чисел $1,2, \ldots, n$,обычно трактуемый как биекция на множестве $\{1, 2, \ldots, n\}$, которая числу $i$ ставит в соответствие $i$-й элемент из набора. Число $n$ при этом называется порядком перестановки. \\
Число всех перестановок степени $n$ равняется $n!$.\\


\textbf{Двоичное слово длины n(!!!)} -- последовательность символов 0 и 1 длины $n$. \\

\textbf{Подмножества конечных множеств}: \\
\textbf{Множество} -- понятие множества обычно принимается за одно из исходных (аксиоматических) понятий, то есть несводимое к другим понятиям, а значит, и не имеющее определения. Для его объяснения используются описательные формулировки, характеризующие множество как совокупность различных элементов, мыслимую как единое целое. \\
\textbf{Подмножество} -- множество $A$ является подмножеством множества $B$, если все элементы, принадлежащие $A$, также принадлежит $B$.

Если конечное множество состоит из $n$ элементов, то оно имеет ровно $2^n$ подмножеств.\\

\section{Формула включений и исключений. Примеры использования.} 

\textbf{Формула включений и исключений} -- комбинаторная формула, позволяющая определить мощность объединения конечного числа конечных множеств, которые в общем случае могут пересекаться друг с другом.\\
Для двух элементов:\\
$|A \cup B| = |A| + |B| - |A \cap B|$.	

Для трех элементов: \\
$|A \cup B \cup C| = |A| + |B| +|C| - |A \cap B| - |A \cap C| - |B \cap C| + |A \cap B \cap C|$.
Её называют формулой включений-исключений для трёх множеств — сначала вклю-
чаем все множества, потом исключаем попарные пересечения, потом снова вклю-
чаем пересечение всех трёх.\\

В общем случае: 
\[
|\bigcup_{i=1}^{n} A_i = \sum_{i}^{}|A_i| - \sum_{i<j}|A_i \cap A_j| + \sum_{i<j<k}|A_i \cap A_j \cap A_k| - \ldots + (-1)^{n-1}|A_1 \cap A_2 \cap \ldots \cap A_n|.
\]


Примеры:
\begin{enumerate}
	\item Сколько есть перестановок чисел от 0 до 9 таких, что первый элемент больше 1, а последний меньше 8?
	
	Посчитаем количество "плохих" перестановок, то есть таких, у которых первый элемент меньше либо равен единице (множество таких перестановок обозначим $X$) и/или последний больше либо равен 8 (множество таких перестановок обозначим $Y$).
	
	Тогда количество "плохих" перестановок по формуле включений-исключений равно:
	
	$|X|+|Y|-|X \cap Y|$
	
	Проведя несложные комбинаторные вычисления, получим:
	
	$2 \cdot 9!+2 \cdot 9! - 2 \cdot 2 \cdot 8!$
	
	Отнимая это число от общего числа перестановок $10!$, получим ответ. 
\end{enumerate}


\section{Биноминальные коэффициенты, основные свойства. Бином Ньютона.} 


\end{document}
